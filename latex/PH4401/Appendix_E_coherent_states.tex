\documentclass[pra,12pt]{revtex4}
\usepackage{amsmath}
\usepackage{amssymb}
\usepackage{graphicx}
\usepackage{color}
\usepackage{mathrsfs}
\usepackage[pdfborder={0 0 0},colorlinks=true,linkcolor=blue,urlcolor=blue]{hyperref}

\def\ket#1{\left|#1\right\rangle}
\def\bra#1{\left\langle#1\right|}
\def\braket#1{\left\langle#1\right\rangle}

\usepackage{fancyhdr}
\fancyhf{}
\lhead{\tiny Y.~D.~Chong (2018)}
\rhead{\scriptsize PH4401: Quantum Mechanics III}
\lfoot{}
\rfoot{\thepage}
\pagestyle{fancy}

\setlength{\parindent}{0pt}

\renewcommand{\baselinestretch}{1.0}
\setlength{\parskip}{0.07in}

\begin{document}

\begin{center}
{\large \textbf{Appendix E: Coherent States}}
\end{center}

Coherent states are special states of bosonic systems (like the
quantum harmonic oscillator, whose excitation quanta can be regarded
as bosonic ``particles''), whose dynamics are closely related to the
oscillatory trajectories of a classical oscillator.  They provide an
important link between the behaviors of quantum and classical
oscillators.

\section{Definition}

The Hamiltonian of a simple harmonic oscillator is (taking $\hbar = m
= \omega_0 = 1$ for simplicity)
\begin{equation}
  \hat{H} = \frac{\hat{p}^2}{2} + \frac{\hat{x}^2}{2},
  \label{hsho}
\end{equation}
where $\hat{x}$ and $\hat{p}$ are the position and momentum operators
respectively.  As we know, we can define ladder operators
\begin{align}
  \hat{a} &= \frac{1}{\sqrt{2}} \left(\hat{x} + i\hat{p}\right) \\
  \hat{a}^\dagger &= \frac{1}{\sqrt{2}} \left(\hat{x} - i\hat{p}\right).
\end{align}
These obey the commutation relation
\begin{equation}
  \left[\hat{a}, \hat{a}^\dagger\right] = 1,
  \label{commutator}
\end{equation}
which is the same as the commutation relation obeyed by creation and
annihilation operators of bosonic particles.

The Hamiltonian for the harmonic oscillator, Eq.~\eqref{hsho}, can be
written as $\hat{H} = \hat{a}^\dagger\hat{a} + 1/2$.  Hence, it can be
argued that the annihilation operator $\hat{a}$ kills off the ground
state $|\varnothing\rangle$:
\begin{equation}
  \hat{a} |\varnothing\rangle = 0.
  \label{annihilation}
\end{equation}
This is, once again, generalizable to the case of bosonic particles,
where the annihilation operator kills off the vacuum state.

Returning to the Hamiltonian \eqref{hsho}, suppose we add a term
proportional to $\hat{x}$:
\begin{equation}
  \hat{H}' = \frac{\hat{p}^2}{2} + \frac{\hat{x}^2}{2} - \sqrt{2}\alpha_1\hat{x}.
  \label{hshift}
\end{equation}
The coefficient of $-\sqrt{2}\alpha_1$, where $\alpha_1 \in
\mathbb{R}$, is for later convenience.  By completing the square, we
can show that this additional term corresponds to a
\textit{displacement} of the center of the potential, plus an energy
shift:
\begin{equation}
  \hat{H}' = \frac{\hat{p}^2}{2} + \frac{1}{2}\left(\hat{x}^2 - \sqrt{2}\alpha_1\right)^2 - \alpha_1^2.
\end{equation}
Therefore, $\hat{H}'$ still describes a harmonic oscillator.  Let
$|\alpha_1\rangle$ denote the ground state for this displaced and
shifted oscillator.  We now want to study $|\alpha_1\rangle$ from the
point of view of the excitations (bosons) of the \textit{original}
Hamiltonian $\hat{H}$---i.e., the excitations created/annihilated by
the $\hat{a}^\dagger$ and $\hat{a}$ operators.

To derive $|\alpha_1\rangle$, we define a new annihilation operator
with a displacement in $x$:
\begin{align}
  \hat{a}' = \frac{1}{\sqrt{2}}\left(\hat{x} - \sqrt{2}\alpha_1 + i \hat{p}\right).
\end{align}
This is related to the original annihilation operator by
\begin{equation}
  \hat{a}' = \hat{a} - \alpha_1.
  \label{aprime}
\end{equation}
We can easily show that $[\hat{a}',\hat{a}'^\dagger] = 1$, and that
$\hat{H}' = \hat{a}'^\dagger \hat{a}' + 1/2 - \alpha_1^2$.  Hence,
\begin{equation}
  \hat{a}' \, |\alpha_1 \rangle = 0.
\end{equation}
But Eq.~\eqref{aprime} then implies that for the \textit{original}
annihilation operator,
\begin{equation}
  \hat{a}\, |\alpha_1\rangle = \alpha_1 \,|\alpha_1\rangle.
  \label{aeigenv}
\end{equation}
In other words, $|\alpha_1\rangle$ is an eigenstate of the original
harmonic oscillator's annihilation operator!  The corresponding
eigenvalue is the displacement parameter $\alpha_1$.

We call the state $|\alpha_1\rangle$, which satisfies
Eq.~\eqref{aeigenv}, a \textbf{coherent state}.  To derive an explicit
expression for it, we use the translation operator $T(\Delta x) =
\exp(-i\hat{p}\Delta x)$.  Evidently, $|\alpha_1\rangle$ can be
generated by displacing the original oscillator's ground state by
$\Delta x = \sqrt{2}\alpha_1$.  Hence,
\begin{align}
  |\alpha_1\rangle &= \exp\left[-\sqrt{2}i\alpha_1\hat{p}\right] |\varnothing\rangle  \\
  &= \exp\left[\alpha_1\left(\hat{a}^\dagger - \hat{a}\right)\right] |\varnothing\rangle.
\end{align}
This can be further simplified by using the Baker-Campbell-Hausdorff
formula for operator exponentials:
\begin{equation}
  \mathrm{If} \;\; [[\hat{A},\hat{B}],\hat{A}] = [[\hat{A},\hat{B}],\hat{B}] = 0
  \;\;\Rightarrow \;\; e^{\hat{A}+\hat{B}}
  = e^{-[\hat{A},\hat{B}]/2}\, e^{\hat{A}} e^{\hat{B}}.
\end{equation}
The result is
\begin{equation}
  |\alpha_1\rangle = e^{-\alpha_1^2/2} \, e^{\alpha_1 \hat{a}^\dagger} |\varnothing\rangle.
\end{equation}
If we write the exponential in its series form,
\begin{equation}
  |\alpha_1\rangle = e^{-\alpha_1^2/2} \, \left(1 + \alpha_1 \hat{a}^\dagger
  + \frac{\alpha_1^2}{2} \left(\hat{a}^\dagger\right)^2 + \cdots\right)
 |\varnothing\rangle,
\end{equation}
then we see that from the point of view of the bosonic excitations of
the original Hamiltonian $\hat{H}$, the state $|\alpha_1\,\rangle$
contains an \textit{indeterminate number of bosons}.  It consists of a
superposition of the zero-boson state (the vacuum), a one-boson state,
a two-boson state, etc.

We can generalize the coherent state by performing a shift not just in
space, but also in momentum.  Instead of Eq.~\eqref{hshift}, let us
define
\begin{align}
  \hat{H}' &= \frac{1}{2}\left(\hat{p} - \sqrt{2}\alpha_2\right)^2
  + \frac{1}{2}\left(\hat{x} - \sqrt{2}\alpha_1\right)^2 \\
  &= \hat{H} - \left(\alpha \hat{a}^\dagger + \alpha^*\hat{a}\right)
  + \textrm{constant},
\end{align}
where
\begin{equation}
  \alpha \equiv \alpha_1 + i \alpha_2 \;\in \;\mathbb{C}.
\end{equation}
It can then be shown that the ground state of $\hat{H}'$, which we
denote by $|\alpha\rangle$, satisfies
\begin{equation}
  \hat{a} \, |\alpha\rangle = \alpha \,|\alpha\rangle.
\end{equation}
(Note that $\hat{a}$ is not Hermitian, so its eigenvalue $\alpha$ need
not be real.)  In explicit terms,
\begin{equation}
  |\alpha\rangle
  \,=\, \exp\left[\alpha\hat{a}^\dagger - \alpha^*\hat{a}\right] |\varnothing\rangle
  \,=\, e^{-|\alpha|^2/2} e^{\alpha\hat{a}^\dagger} |\varnothing\rangle.
  \label{alphaexp}
\end{equation}

\section{Basic properties}

There is one coherent state $|\alpha\rangle$ for each complex number
$\alpha \in \mathbb{C}$.  They have the following properties:
\begin{enumerate}
\item They are normalized: $\langle\alpha|\alpha\rangle = 1$.  This is
  obvious from the way they are defined (as ground states of displaced
  harmonic oscillators).

\item They form a complete set.  This means that the identity operator
  can be resolved as
  \begin{equation}
    \hat{I} = C \int d^2\alpha |\alpha\rangle\langle\alpha|,
  \end{equation}
  where $C$ is some numerical constant, and $\int d^2\alpha$ denotes
  an integral over the complex plane.  However, the coherent states do
  \textit{not} form an orthonormal set, as they are over-complete:
  $\langle\alpha|\alpha'\rangle \ne 0$ for $\alpha \ne \alpha'$.

\item The expected number of particles in a coherent state is
  \begin{equation}
    \langle\alpha| \hat{a}^\dagger\hat{a} | \alpha\rangle
    = |\alpha|^2.
  \end{equation}

\item The probability distribution of the number of particles follows
  a \textit{Poisson distribution}:
  \begin{equation}
    P(n) = |\langle n | \alpha\rangle|^2 = e^{-|\alpha|^2} \frac{|\alpha|^{2n}}{n!}.
  \end{equation}
  The mean and variance of this distribution are both $|\alpha|^2$.

\item The mean position and momentum are
  \begin{align}
    \langle \alpha | \hat{x}|\alpha\rangle &= \sqrt{2} \, \mathrm{Re}(\alpha) 
    \label{x} \\
    \langle \alpha | \hat{p}|\alpha\rangle &= \sqrt{2} \, \mathrm{Im}(\alpha).
    \label{p}
  \end{align}
\end{enumerate}

\section{Dynamical properties}

Consider the Hamiltonian
\begin{equation}
  \hat{H} = \hat{a}^\dagger \hat{a}.
\end{equation}
This is the harmonic oscillator Hamiltonian, with the zero-point
energy omitted for convenience.  Suppose we initialize the system in a
coherent state $|\alpha_0\rangle$, for some arbitrary $\alpha_0 \in
\mathbb{C}$.  This is not an energy eigenstate of $\hat{H}$; how will
it subsequently evolve?

It turns out that the dynamical state will take the form
\begin{equation}
  |\psi(t)\rangle = |\alpha(t)\rangle, \;\;\;\mathrm{where}\;\;\alpha(0) = \alpha_0.
\end{equation}
In other words, the system remains in a coherent state, but the
complex $\alpha(t)$ will vary with time.  To determine $\alpha(t)$, we
plug the ansatz into the time-dependent Schr\"odinger equation:
\begin{align}
  i \frac{d}{dt} |\alpha(t)\rangle &= \hat{a}^\dagger \hat{a} |\alpha(t)\rangle\\
  \Rightarrow \;\;\;
  i \Big\langle\alpha(t)\Big| \frac{d}{dt} \Big|\alpha(t)\Big\rangle &=
  |\alpha(t)|^2.
\end{align}
Using Eq.~\eqref{alphaexp}, it can be shown (after some algebra) that
this reduces to
\begin{equation}
  \frac{i}{2}\left(\dot{\alpha}\alpha^* - \alpha\dot{\alpha}^*\right) = |\alpha|^2.
  \label{eom}
\end{equation}
This complex differential equation looks complicated, but the solution
is actually simple.  Let us define
\begin{equation}
  \alpha(t) = A(t) \, e^{i\phi(t)},
\end{equation}
where $A(t)$ and $\phi(t)$ are real.  Substituting this into
Eq.~\eqref{eom} gives just
\begin{equation}
  \frac{d\phi}{dt} = -1.
\end{equation}
Thus, the solution to Eq.~\eqref{eom} for the intial condition
$\alpha(t) = \alpha_0$ is
\begin{equation}
  \alpha(t) = \alpha_0 \, e^{-it} = |\alpha_0| e^{-i[t - \mathrm{arg}(\alpha_0)]}.
\end{equation}
Referring back to Eqs.~\eqref{x}--\eqref{p}, this implies that the
mean position and momentum satisfies
\begin{align}
  \langle x\rangle &=\, \;\;\sqrt{2} |\alpha_0| \cos\left[t - \mathrm{arg}(\alpha_0)\right] \\
  \langle p\rangle &= -\sqrt{2} |\alpha_0| \sin\left[t - \mathrm{arg}(\alpha_0)\right].
\end{align}
Hence, we conclude that that dynamics of a coherent state reproduces
the motion of a classical simple harmonic oscillator with $m =
\omega_0 = 1$.

\end{document}
