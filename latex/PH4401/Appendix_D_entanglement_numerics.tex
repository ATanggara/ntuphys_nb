\documentclass[pra,12pt]{revtex4}
\usepackage{amsmath}
\usepackage{amssymb}
\usepackage{graphicx}
\usepackage{color}
\usepackage{mathrsfs}
\usepackage[pdfborder={0 0 0},colorlinks=true,linkcolor=blue,urlcolor=blue]{hyperref}

\def\ket#1{\left|#1\right\rangle}
\def\bra#1{\left\langle#1\right|}
\def\braket#1{\left\langle#1\right\rangle}

\usepackage{fancyhdr}
\fancyhf{}
\lhead{\tiny Y.~D.~Chong (2018)}
\rhead{\scriptsize PH4401: Quantum Mechanics III}
\lfoot{}
\rfoot{\thepage}
\pagestyle{fancy}

\setlength{\parindent}{0pt}

\renewcommand{\baselinestretch}{1.0}
\setlength{\parskip}{0.07in}

\begin{document}

\begin{center}
{\large \textbf{Appendix D: Numerical Tensor Products}}
\end{center}

As described in Chapter 2, multi-particle quantum states are described
mathematically using tensor products.  This appendix discusses how
tensor product states, and the operators acting upon them, are handled
in numerical linear algebra software like
\href{https://scipy.org/}{Scientific Python}, GNU Octave, or Matlab.

Numerical linear algebra is always performed in an explicit basis, so
that states are represented by 1D arrays of numbers (vectors), and
linear operators are represented by 2D arrays (matrices).  Tensor
products can be represented by an operation known as the ``Kronecker
product'', which is implemented by the
\href{https://docs.scipy.org/doc/numpy/reference/generated/numpy.kron.html}{\texttt{numpy.kron}}
function in Python, and the \texttt{kron} function in GNU Octave and
Matlab.

The Kronecker product operation takes two array inputs, and returns a
new array consisting of blocks of the second array scaled by the
components of the first array.  What this means is best understood in
terms of two distinct cases: (i) products of 1D arrays, and (ii)
products of 2D arrays.

\section{Products of vectors}

Suppose that $a$ and $b$ are both 1D arrays, of length $M$ and $N$
respectively; let their components be $(a_0, a_1, \dots, a_{M-1})$ and
$(b_0, b_1, \dots, b_{N-1})$.  Following the conventions of Scientific
Python, we will not distinguish between ``row vectors'' and ``column
vectors'', and components are indexed from 0.  The Kronecker product
has the following effect:
$$\texttt{kron}(a, b) = \Big(a_0b_0,\,\dots,\, a_0 b_{N-1},\, a_1 b_0,\, \dots,\, a_1 b_{N-1},\, \dots,\, a_{M-1}b_{N-1}\Big).$$
This is a 1D array of length $MN$, and its first $N$ components are
$a_0 (b_0, \dots,b_{N-1})$, the next $N$ components are $a_1 (b_0,
\dots, b_{N-1})$, etc.  We can describe this compactly in index
notation:
$$\big[\, \texttt{kron}(a, b) \,\big]_{\mu} = a_m \, b_n \;\;\;\mathrm{where}\;\;\;\mu = mN+n.$$
Note that the index $\mu$ sweeps through the values $\mu =
0,1,\dots,MN-1$, without duplication, as we take $m = 0,\dots,M-1$ and
$n = 0,\dots,N-1$.

This operation can be used to describe the tensor product of two
quantum state vectors.  Suppose we have a state vector $|a\rangle$ in
an $M$-dimensional Hilbert space $\mathscr{H}_A$, and a state vector
$|b\rangle$ in an $N$-dimensional Hilbert space $\mathscr{H}_B$.  Let
the two Hilbert spaces have bases $\{|m\rangle\}$ and $\{|n\rangle\}$
respectively.  The state vectors can be written using the basis
decompositions
$$|a\rangle = \sum_{m=0}^{M-1} a_m |m\rangle, \quad |b\rangle = \sum_{n=0}^{N-1} b_n |n\rangle.$$
The tensor product $|a\rangle\otimes|b\rangle$ is a vector in the
space $\mathscr{H}_A\otimes \mathscr{H}_B$, which is $MN$-dimensional.
We can adopt a natural basis for this product space,
$$\Big\{|\mu\rangle \equiv |m\rangle\otimes |n\rangle\Big\} \;\;\;\mathrm{where} \;\begin{cases}\mu\!\!\!\! &= mN+n \\ m \!\!\!\!&= 0,1,\dots,M-1 \\ n \!\!\!\!&= 0,1, \dots, N-1.\end{cases}$$
Then the components of $|A\rangle\otimes|B\rangle$ in this basis are
given by the components of the Kronecker product:
$$|a\rangle\otimes|b\rangle = \sum_{\mu=0}^{MN-1} \big[\texttt{kron}(a,b)\big]_\mu \; |\mu\rangle.$$

\section{Products of matrices}

Consider the case where $A$ and $B$ are 2D arrays (matrices), whose
sizes are $M\times M$ and $N\times N$ respectively:
$$A = \begin{bmatrix}A_{00} & \cdots & A_{0,M-1} \\ \vdots & \ddots & \vdots \\ A_{M-1,0} & \cdots & A_{M-1,M-1} \end{bmatrix}, \;\; B = \begin{bmatrix}B_{00} & \cdots & B_{0,N-1} \\ \vdots & \ddots & \vdots \\  B_{N-1,0} & \cdots & B_{N-1,N-1} \end{bmatrix}.$$
In this case, the Kronecker product of $A$ and $B$ is an $MN\times MN$
array of the form
$$\texttt{kron}(A,B) = \begin{bmatrix} A_{00}B & \cdots & A_{0,M-1}B \\ \vdots & \ddots & \vdots \\ A_{M-1,0}B & \cdots & A_{M-1,M-1}B\end{bmatrix}.$$
In component terms,
$$\big[\,\texttt{kron}(A,B)\,\big]_{\mu\mu'} = A_{mm'} B_{nn'}\;\;\;\mathrm{where}\;\;\;\mu = mN+n, \; \mu' = m'N+n'.$$
The index $\mu$ sweeps through the values $\mu = 0,1,\dots,MN-1$,
without duplication, as we take $m = 0,\dots,M-1$ and $n =
0,\dots,N-1$; and likewise for the primed indices.

This Kronecker product can be used to describe the tensor product of
two quantum operators.  Let $\hat{A}$ be a quantum operator acting on
the $M$-dimensional Hilbert space $\mathscr{H}_A$, and $\hat{B}$ a
quantum operator acting on the $N$-dimensional Hilbert space
$\mathscr{H}_B$.  As before, let the bases for the two spaces be
$\{|m\rangle\}$ and $\{|n\rangle\}$ respectively, so that
$$\hat{A} = \sum_{m,m'=0}^{M-1}  |m\rangle A_{mm'} \langle m'|, \quad \hat{B} = \sum_{n,n'=0}^{N-1} |n\rangle B_{nn'}\langle n'|.$$
The tensor product $\hat{A}\otimes\hat{B}$ acts on the
$MN$-dimensional Hilbert space $\mathscr{H}_A\otimes \mathscr{H}_B$.
Again, we adopt the basis $\big\{|\mu\rangle \equiv |m\rangle\otimes
|n\rangle \big\}$ where $\mu = mN+n$.  Then the matrix elements in
this basis are given by the components of the Kronecker product:
$$\hat{A}\otimes\hat{B} = \sum_{\mu,\mu'=0}^{MN-1} |\mu\rangle \; \big[\,\texttt{kron}(A,B)\,\big]_{\mu\mu'} \; \langle\mu'|.$$

Thus, we see that the Kronecker product takes pairs of vectors, or
pairs of matrices, and returns the tensor product corresponding to a
particular straightforward basis choice for the product space.  Note
that the order of inputs matters: $\texttt{kron}(A,B)$ is not the same
as $\texttt{kron}(B,A)$, since the inputs determine the ordering of
the basis vectors for the product space (e.g., in the former case, we
define $\mu = mN+n$ rather than $\mu = nM+m$).  We have to be careful
to stick to one choice throughout a program (e.g., vectors and
matrices for $\mathscr{H}_A$ always go in the first slot, and those
for $\mathscr{H}_B$ in the second slot).

\section{Projections}

Let $|\psi\rangle \in \mathscr{H}_A\otimes \mathscr{H}_B$ be the quantum state
of a two-particle system.  Let its vector components, in the basis
used by the Kronecker product, be $\{\psi_0, \dots, \psi_{MN-1}\}$:
$$|\psi\rangle = \sum_{\mu} \psi_\mu |\mu\rangle = \sum_{mn} \psi_{\mu(m,n)} |m\rangle\otimes|n\rangle, \;\;\;\mathrm{where}\;\;\mu(m,n) = mN+n.$$
To keep the notation from getting too crowded, we will stop explicitly
indicating the summation limits; each sum is assumed to run over the
full range of the index concerned.

Suppose we perform a measurement on subsystem $A$, and collapse that
subsystem into the state $|a\rangle = \sum_m a_m |m\rangle \in
\mathscr{H}_A$.  According to the discussion in Chapter 2, the
probability of achieving this result is
$$P = \big|\langle b|b\rangle\big|^2 \;\;\; \mathrm{where} \;\; |b\rangle = \langle a | \psi\rangle \in \mathscr{H}_B.$$
Hence, we need to compute the \textit{projection} of $|\psi\rangle$
onto the Hilbert space $\mathscr{H}_B$ (which is a vector in
$\mathscr{H}_B$).  In index notation,
$$\begin{aligned}|b\rangle = \langle a | \psi\rangle &= \left(\sum_{m'} \langle m'| a_{m'}^* \right) \left(\sum_{mn}\psi_{\mu(m,n)} |m\rangle\otimes|n\rangle \right) \\ &= \sum_{n} \left(\sum_m a_{m}^*\, \psi_{\mu(m,n)}\right) |n\rangle\end{aligned}$$
The components of the projected vector are
$$b_n = \sum_m a_{m}^*\, \psi_{mN+n}.$$
To calculate this, it is convenient to first re-arrange $\psi$ from a
1D array to a 2D array, using the
\href{https://docs.scipy.org/doc/numpy/reference/generated/numpy.resize.html}{\texttt{numpy.resize}}
function in Python (or the equivalent \texttt{resize} function in GNU
Octave or Matlab).  If \texttt{psi} is a 1D array of length $MN$ whose
components are $\psi_{\mu(m,n)}$, then \texttt{resize(psi, (M,N))}
returns an $M\times N$ matrix whose components are
$\psi_{mn}^{(\mathrm{resized})} = \psi_{\mu(m,n)}$.  Therefore,
$$b_n = \sum_m a_{m}^*\, \psi_{mn}^{(\mathrm{resized})}.$$
The $b$ array can thus be computed with the code
\begin{verbatim}
    b = dot(conj(a), resize(psi, (M,N)))     # Projecting out space A
\end{verbatim}
Here, \texttt{conj} returns the complex conjugate, and the
\href{https://docs.scipy.org/doc/numpy/reference/generated/numpy.dot.html}{\texttt{dot}}
function performs a sum product over the last index of the first input
and the next-to-last index of the second input (in this case, this
means summing over the index $m$).

For projections onto the other Hilbert space, the procedure is very
slightly different.  Suppose we perform a measurement on subsystem
$B$, and collapse that subsystem into the state $|b\rangle = \sum_n
b_n |n\rangle \in \mathscr{H}_B$.  In that case, we would need to
calculate the projection
$$\begin{aligned}|a\rangle &= \langle b | \psi\rangle  \;\; \in \;\; \mathscr{H}_A \\ &= 
\left(\sum_{n'} \langle n'| b_{n'}^* \right) \left(\sum_{mn}\psi_{\mu(m,n)} |m\rangle\otimes|n\rangle \right) \\ &= \sum_{m} \left(\sum_n b_{n}^*\, \psi_{\mu(m,n)}\right) |m\rangle
\end{aligned}$$
The components of the projected vector are
$$a_n = \sum_n b_n^* \, \psi_{mN+n} = \sum_n \psi_{mn}^{(\mathrm{resized})}\, b_{n}^*.$$
To compute this, we supply the arrays to the \texttt{dot} function
in the opposite order:
\begin{verbatim}
    a = dot(resize(psi, (M,N)), conj(b))     # Projecting out space B
\end{verbatim}

\section{Partial traces}

Another common computation that we need to perform is the partial
trace, which is used to in constructing density matrices, computing
the entanglement entropy, etc.  Consider an operator $\hat{C}$ acting
on the space $\mathscr{H}_A\otimes\mathscr{H}_B$, such that
$$\hat{C} = \sum_{\mu,\mu'} |\mu\rangle \, C_{\mu\mu'} \, \langle\mu'|.$$
As before, we take $|\mu\rangle \equiv |m\rangle\otimes|n\rangle$
where $\mu = mN+n$, which is the basis corresponding to the Kronecker
product.  Suppose we wish to trace over subsystem $A$:
$$\begin{aligned}\mathrm{Tr}_A \big[\hat{C}\big] &= \sum_m \langle m| \,\hat{C} \,|m\rangle \\ &= \sum_{nn'} |n\rangle \left(\sum_m C_{mN+n, mN+n'}\right) \langle n'| \end{aligned}$$
Such a matrix can be computed by array slicing:
\begin{verbatim}
    Ctrace = zeros((N,N))                    # Tracing out space A
    for m in range(M):
        Ctrace += C[m*N:(m+1)*N, m*N:(m+1)*N]
\end{verbatim}

If we instead perform a partial trace over subsystem $B$, then
$$\mathrm{Tr}_B \big[\hat{C}\big] = \sum_{mm'} |m\rangle \left(\sum_n C_{mN+n, m'N+n}\right) \langle m'|.$$
This can be computed as follows:
\begin{verbatim}
    Ctrace = zeros((M,M))                    # Tracing out space B
    for n in range(N):
        Ctrace += C[n:M*N+n:N, n:M*N+n:N]
\end{verbatim}

\end{document}
