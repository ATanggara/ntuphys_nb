\documentclass[pra,11pt]{revtex4}
\usepackage{amsmath}
\usepackage{amssymb}
\usepackage{graphicx}
\usepackage{color}
\usepackage{mathrsfs}
\usepackage[pdfborder={0 0 0},colorlinks=true,linkcolor=blue]{hyperref}

\def\ket#1{\left|#1\right\rangle}
\def\bra#1{\left\langle#1\right|}
\def\braket#1{\left\langle#1\right\rangle}

\setlength{\parindent}{0pt}

\renewcommand{\baselinestretch}{1.0}
\setlength{\parskip}{0.07in}

\begin{document}

\section{Quantum mechanics of multi-particle systems}

So far, we have studied quantum mechanical systems consisting of
single particles.  The next important step is to look at systems of
more than one particle.  As we shall see, the postulates of quantum
mechanics have interesting implications that only show up in
multi-particle systems, particularly the phenomenon of \textbf{quantum
  entanglement}.

In the following discussion, you may assume we are dealing with
systems of ``distinguishable'' particles.  There is another set of
complications if the particles are ``identical'' or
``indistinguishable'', which we'll deal with in the next chapter.  (If
you're unsure what this means, just read on.)

Suppose we have a pair of distinguishable particles, labeled A and B.
If each individual particle is treated as a quantum system, then
according to the postulates of quantum mechanics, its state is
described by a vector in a complex Hilbert space.  Let $\mathscr{H}_A$
and $\mathscr{H}_B$ denote the respective single-particle Hilbert
spaces.

When the two particles are considered as a single system, the combined
Hilbert space is denoted
$$\mathscr{H} = \mathscr{H}_A\otimes \mathscr{H}_B.$$
The symbol $\otimes$ refers to a mathematical operation known as the
\textbf{tensor product}, which is a way of combining two Hilbert
spaces to form a larger Hilbert space.  Its meaning is as follows:
suppose $\mathscr{H}_A$ is spanned by an orthonormal basis
$\{|\mu_1\rangle, |\mu_2\rangle, |\mu_3\rangle, \dots\}$, and
$\mathscr{H}_B$ spanned by $\{|\nu_1\rangle, |\nu_2\rangle,
|\nu_3\rangle, \dots\}$.  Then the tensor product space $\mathscr{H}_A
\otimes \mathscr{H}_B$ is a space spanned by the basis vectors
$$\Big\{\;\,|\mu_i\rangle\otimes|\nu_j\rangle \;\;  \Big| \;\; \textrm{all}\;|\mu_i\rangle,\; |\nu_j\rangle \;\,\Big\}.$$
This reflects the intuitive notion that if particle A has some state
$|\mu_i\rangle$ and particle B has state $|\nu_j\rangle$, then the
state of the combined system is fully specified.  Note that if
$\mathscr{H}_A$ has dimension $d_A$ and $\mathscr{H}_B$ has dimension
$d_B$, then the combined Hilbert space $\mathscr{H}$ has dimension
$d_A d_B$.  Using such a basis, any two-particle state can be written as
$$|\psi\rangle = \sum_{i} \sum_{j} \, c_{ij}\; |\mu_i\rangle \otimes |\nu_j\rangle.$$

Inner products between the tensor product basis states is defined as follows:
$$\Big(|\mu_i\rangle \otimes |\nu_j\rangle\;,\; |\mu_p\rangle \otimes |\nu_q\rangle \Big) \;\equiv\; \Big(\langle\mu_i| \otimes \langle\nu_j| \Big) \Big(|\mu_p\rangle \otimes |\nu_q\rangle\Big) \;\equiv\; \langle\mu_i|\mu_p\rangle \, \langle\nu_j|\nu_q\rangle = \delta_{ip}\delta_{jq}.$$
In other words, we (i) calculate the inner product for space A, (ii)
calculate the inner product for space B, and (iii) multiply the two
resulting numbers together.  You can check that this satisfies the
formal requirements for an inner product in linear algebra.

As an example, suppose $\mathscr{H}_A$ and $\mathscr{H}_B$ are both 2D
Hilbert spaces describing spin-$1/2$ degrees of freedom.  Each space
can be spanned by an orthonormal basis $\{\,|\!+\!z\rangle,
\,|\!-\!z\rangle \, \}$, representing ``spin-up'' and ``spin-down''.
Then the tensor product space $\mathscr{H}$ is a 4D space spanned by
$$\Big\{\;|\!+\!z\rangle\otimes|\!+\!z\rangle\,,\; |\!+\!z\rangle\otimes|\!-\!z\rangle\,,\; |\!-z\!\rangle\otimes|\!+\!z\rangle\,,\; |\!-\!z\rangle\otimes|\!-\!z\rangle \;\Big\}.$$

We now make an important observation.  As previously noted, if
particle $A$ is in state $|\mu_i\rangle$ and particle $B$ is in state
$|\nu_j\rangle$, that specifies the state of the combined system.  But
the reverse is not generally true!  There are states of the combined
system that \textit{cannot} be expressed in terms of definite states
of the individual particles.  For example, take the previous example
of two spaces describing spin-$1/2$ degrees of freedom, and consider
$$|\psi\rangle = \frac{1}{\sqrt{2}} \Big(|\!+\!z\rangle\otimes|\!-\!z\rangle \,-\, |\!-z\!\rangle\otimes|\!+\!z\rangle\Big).$$

This two-particle state is constructed from two of the basis states of
the tensor product space, $|\!+\!z\rangle\otimes|\!-\!z\rangle$ and
$|\!-\!z\rangle\otimes|\!+\!z\rangle$; you can check that the factor
of $1/\sqrt{2}$ correctly normalizes the state so that
$\langle\psi|\psi\rangle = 1$.  Evidently, neither particle A nor
particle B has a definite $|\!+\!z\rangle$ or $|\!-\!z\rangle$ state.
Moreover, we shall show (in Section~\ref{sec:entropy}) that there's
\textit{no} choice of basis that allows the individual particles to be
expressed with definite quantum states.  In other words,
$$|\psi\rangle \ne |\psi_A\rangle\otimes|\psi_B\rangle \;\;\;\textrm{for}\;\textrm{any}\;\; |\psi_A\rangle \in \mathscr{H}_A, \;|\psi_B\rangle \in \mathscr{H}_B.$$
In such a situation, the two particles are said to be
\textbf{entangled}.  This means that they cannot be described by
individual single-particle quantum states, but are intrinsically
``mixed together''.

For systems of more than two particles, quantum states can be defined
using multiple tensor products.  In general, suppose a quantum system
contains $N$ subsystems described by the individual Hilbert spaces
$\{\mathscr{H}_1, \mathscr{H}_2, \dots, \mathscr{H}_N\}$, which have
dimensionality $\{d_1, \dots, d_N\}$.  Then the overall system is
described by the Hilbert space
$$\mathscr{H} = \mathscr{H}_1 \otimes \mathscr{H}_2 \otimes \cdots
\otimes \mathscr{H}_N.$$
This combined Hilbert space has dimensionality $d = d_1 d_2\cdots
d_N$.  It is interesting to note that the dimensionality scales
\textit{exponentially} with the number of subsystems.  For instance,
if a single particle has a 2D Hilbert space, a $20$-particle system
has a Hilbert space of dimensionality $2^{20} =1\,048\,576$.
Evidently, even for quantum systems containing a modest number of
particles, the quantum state carries huge amounts of information.
This is one of the motivations behind the active research field of
quantum computing.

\section{Partial measurements}
\label{sec:partialmeasurements}

Let us recall how measurements work in single-particle quantum theory.
Each observable quantity $Q$ is described by some Hermitian operator
$\hat{Q}$, which has an eigenbasis $\{|q_i\rangle\}$ such that
$$\hat{Q}|q_i\rangle = q_i |q_i\rangle.$$
For simplicity, let the eigenvalues $\{q_i\}$ be non-degenerate.
Suppose a particle initially has quantum state $|\psi\rangle$; this
can always be expanded in terms of the eigenbasis of $\hat{Q}$:
$$|\psi\rangle = \sum_i \psi_i\, |q_i\rangle, \;\;\mathrm{where}\;\;\,\textrm{and}\;\, \psi_i = \langle q_i|\psi\rangle.$$
The \textbf{measurement postulate of quantum mechanics} states that if
we measure $Q$, then (i) the probability of obtaining the measurement
outcome $q_i$ is $P_i = |\psi_i|^2$, the absolute square of the
coefficient of $|q_i\rangle$ in the basis expansion; and (ii) upon
obtaining this outcome, the system instantly ``collapses'' into state
$|q_i\rangle$.  Mathematically, the state collapse can be described
using the projection operation
$$|\psi\rangle \longrightarrow \frac{1}{\sqrt{\mathcal{N}}}\; |q_i\rangle\langle q_i|\psi\rangle,$$
where the prefactor $\mathcal{N} = |\psi_i|^2$ ensures that the new
state remains normalized.

For multi-particle systems, there is a new complication: what if a
measurement is performed on just one particle?

Consider once again a system of two particles A and B, with
two-particle Hilbert space $\mathscr{H}_A \otimes \mathscr{H}_B$.
Suppose we perform a measurement on particle $A$; the measurement
corresponds to a Hermitian operator $\hat{\mu}$ that acts upon
$\mathscr{H}_A$, and has eigenvectors $\{|\mu_1\rangle,
|\mu_2\rangle,\dots\}$.  We can write any state $|\psi\rangle$ using
the eigenbasis of $\hat{\mu}$ for the $\mathscr{H}_A$ part, and an
arbitrary basis $\{|\nu_1\rangle, |\nu_2\rangle,\dots\}$ for the
$\mathscr{H}_B$ part:
$$\begin{aligned}|\psi\rangle &= \sum_{ij} \psi_{ij}\, |\mu_i\rangle\otimes |\nu_j\rangle \\&= \sum_i |\mu_i\rangle\otimes |\varphi_i\rangle, \;\;\mathrm{where}\;\;|\varphi_i\rangle\equiv \sum_j \psi_{ij}\,|\nu_j\rangle.\end{aligned}$$
Unlike the single-particle case, the ``coefficient'' of
$|\mu_i\rangle$ in this basis expansion is not a complex number, but a
vector $|\varphi_i\rangle \in \mathscr{H}_B$.  Proceeding by analogy,
we postulate that the probability of obtaining the outcome $\mu_i$ is
the ``absolute square'' of this ``coefficient'':
$$P_i = \langle \varphi_i|\varphi_i\rangle = \sum_j |\psi_{ij}|^2.$$
After we obtain the measurement result $\mu_i$, the state should
collapse.  This can be described by the projection operation
$$|\psi\rangle \longrightarrow \frac{1}{\sqrt{\mathcal{N}}}\; \Big(|\mu_i\rangle\langle \mu_i| \otimes \hat{I}\Big) |\psi\rangle = \frac{1}{\sqrt{\mathcal{N}}}\; |\mu_i\rangle\otimes |\varphi_i\rangle.$$
Note that the projection operator $|\mu_i\rangle\langle \mu_i|$ acts
only upon the $\mathscr{H}_A$ part of the two-particle space.  The
prefactor $\mathcal{N}$ is, once again, defined so that the new state
is normalized.

Let's work through an explicit example.  Consider a system composed of
two spin-$1/2$ particles, with the two-particle state
$$|\psi\rangle = \frac{1}{\sqrt{2}} \Big(|\!+\!z\rangle\otimes|\!-\!z\rangle \,-\, |\!-\!z\rangle\otimes|\!+\!z\rangle\Big).$$
For each particle, $|\!+\!z\,\rangle$ and $|\!-\!z\,\rangle$ denote
eigenstates of the operator $\hat{S}_z$, with eigenvalues $+\hbar/2$
and $-\hbar/2$ respectively.  Suppose we measure $S_z$ on particle A.
Then:
\begin{itemize}
\item The outcome $+\hbar/2$ occurs with probability $P_+ = \langle
  n'|n'\rangle = 1/2$, where $|n'\rangle =
  (1/\sqrt{2})\,|\!-\!z\rangle$.  After the measurement, the state
  collapses to $|\!+\!z\rangle \otimes|\!-\!z\rangle$.

\item The outcome $-\hbar/2$ occurs with probability $P_- = \langle
  n'|n'\rangle = 1/2$, where $|n'\rangle =
  (1/\sqrt{2})\,|\!+\!z\rangle$.  After the measurement, the state
  collapses to $|\!-\!z\rangle \otimes|\!+\!z\rangle$.
\end{itemize}
In other words, the two possible results $\pm \hbar/2$ occur with
equal probability.  In either case, the two-particle state collapses
so that particle $A$ is in the observed spin eigenstate, while
particle $B$ has the opposite spin.  After the collapse, the
two-particle state is no longer entangled.

\section{The Einstein-Podolsky-Rosen ``paradox''}

In 1935, \hyperref[cite:epr]{Einstein, Podolsky, and Rosen (EPR)} used
the counter-intuitive features of quantum entanglement to formulate a
thought experiment known as the \textbf{EPR paradox}.  They tried to
use this thought experiment to argue that quantum theory cannot
ultimately be a correct description of reality.  Subsequently,
however, it was shown that the EPR paradox is not an actual paradox;
physical systems \textit{really do} behave in the strange way
described in this thought experiment.

The EPR paradox begins with a two-particle entangled state, such as
the following state of two spin-$1/2$ particles:
$$|\psi\rangle = \frac{1}{\sqrt{2}} \Big(|\!+\!z\rangle\otimes|\!-\!z\rangle \,-\, |\!-\!z\rangle\otimes|\!+\!z\rangle\Big).$$
As discussed in the previous section, measuring $S_z$ on particle $A$
causes the system to collapse into a two-particle state with definite
spins.  If the measurement outcome is $+\hbar/2$, the collapsed state
is $|\!+\!z\rangle \otimes|\!-\!z\rangle$, whereas if we measure
$-\hbar/2$, the collapsed state is
$|\!-\!z\rangle\otimes|\!+\!z\rangle$.

According to quantum theory, the state collapse happens
instantaneously, regardless of the distance separating the particles.
We could prepare the two-particle state in a laboratory and hang on to
particle $A$, while sending particle $B$ to the Betelgeuse star
system, 642 light years away.  In principle, this can be done
carefully enough to avoid disturbing the two-particle quantum state.
Once ready, we measure $\hat{S}_z$ on particle $A$, which induces an
instantaneous collapse of the two-particle state.  Immediately
afterwards, if our alien colleague at Betelgeuse measures $\hat{S}_z$
on particle $B$, xhe will obtain (with 100\% certainty) the opposite
of our Earth result.  Yet in the time between these two measurements,
no classical signal could have traveled between Earth and Betelgeuse,
not even at the speed of light.

\textcolor{red}{[Fig.]}

There are three noteworthy aspects of this phenomenon.

First, it dispels certain intuitive but ultimately mistaken
``explanations'' of quantum state collapse.  For instance, it is
sometimes explained that if we want to measure a particle's position,
we need to shine a light beam on it, or disturb it in some way; due to
this disturbance, the particle's momentum becomes uncertain after the
measurement.  The EPR paradox shows that such stories don't capture
the full weirdness of quantum state collapse, since we can collapse
the state of a particle by doing a measurement on \textit{another}
particle, hundreds of light years away!

Second, the experimentalist can have a certain amount of control over
the state collapse, by choosing what measurement to perform.  So far,
we have considered $S_z$ measurements performed on particle $A$.  But
the Earth experimentalist can choose to measure the spin of $A$ along
another axis, say $S_x$.  In the basis of spin-up and spin-down states,
the operator $\hat{S}_x$ has matrix representation
$$\hat{S}_x = \frac{\hbar}{2}\, \begin{pmatrix}0&1\\1&0\end{pmatrix}.$$
The eigenvalues and eigenvectors are
$$\begin{aligned}s_x = \;\;\frac{\hbar}{2},\; &\;\;\; |\!+\!x\rangle = \frac{1}{\sqrt{2}}\Big(|\!+\!z\rangle + |\!-\!z\rangle\Big) \\ s_x = -\frac{\hbar}{2}, &\;\;\; |\!-\!x\rangle = \frac{1}{\sqrt{2}}\Big(|\!+\!z\rangle - |\!-\!z\rangle\Big).\end{aligned}$$
Conversely, we can write the $\hat{S}_z$ eigenstates in the $\{|\!+\!x\rangle,|\!-\!x\rangle\}$ basis:
$$\begin{aligned}|\!+\!z\rangle &= \frac{1}{\sqrt{2}}\Big(|\!+\!x\rangle + |\!-\!x\rangle\Big) \\ |\!-\!z\rangle &= \frac{1}{\sqrt{2}}\Big(|\!+\!x\rangle - |\!-\!x\rangle\Big).\end{aligned}$$
This allows us to write the two-particle entangled state in the
$\hat{S}_x$ basis:
$$|\psi\rangle = \frac{1}{\sqrt{2}} \Big(|\!-\!x\rangle\otimes|\!+\!x\rangle \,-\, |\!+\!x\rangle\otimes|\!-\!x\rangle\Big).$$
The Earth measurement still collapses the particles into definite spin
states with opposite spins---but now spin states of ${S}_x$ rather
than ${S}_z$.

Third, the choice of measurement cannot be used for superluminal
communication.  The Earth experimentalist can choose whether to (i)
measure $S_z$ or (ii) measure $S_x$, and this choice has an
instantaneous effect on the state of particle $B$.  If the Betelgeuse
experimentalist can find a way to distinguish between case (i) and
case (ii), even statistically, this could serve as a method of
instantaneous communication.  Yet this turns out to be impossible!
This is because quantum states themselves cannot be measured; only
observables can be measured.  Suppose the Earth measurement is
$\hat{S}_z$, which collapses $B$ to either $|\!+\!z\rangle$ or
$|\!-\!z\rangle$ (each with probability 0.5).  The Betelgeuse
experimentalist can now choose which measurement to perform.  If $S_z$
is measured, the outcome is $+\hbar/2$ or $-\hbar/2$ with equal
probabilities.  If $S_x$ is measured, the probabilities are:
$$\begin{aligned}P(S_x = +\hbar/2) &= \frac{1}{2}\, \Big|\langle\!+x|\!+\!z\rangle\Big|^2 \;+\;\, \frac{1}{2}\, \Big|\langle\!+x|\!-\!z\rangle\Big|^2 = \frac{1}{2}\\P(S_x = -\hbar/2) &= \frac{1}{2}\, \Big|\langle\!-x|\!+\!z\rangle\Big|^2 \;+\;\, \frac{1}{2}\, \Big|\langle\!-x|\!-\!z\rangle\Big|^2 = \frac{1}{2}.\end{aligned}$$
This analysis can be repeated for any other choices of measurement
axis.  No matter what, we find that the outcomes of any spin
measurement at Betelgeuse always have 50/50 probability!  Therefore,
the Betelgeuse measurement does not yield any information about the
choice of measurement axis made on Earth.

Since quantum state collapse does not allow for superluminal
communication, it is consistent \textit{in practice} with the theory
of relativity.  However, state collapse is still \textbf{nonlocal}, in
the sense that certain unobservable ingredients of the theory (quantum
states) change faster than light can travel between two points.  For
this reason, EPR argued that quantum theory is
\textit{philosophically} inconsistent with relativity.

They then suggested an alternative: maybe quantum mechanics is an
approximation of some deeper theory, whose details are currently not
known, but which is both deterministic and local.  Such a
``\textbf{hidden variables theory}'' can give rise to the appearance
of quantum state collapse in the following way.  Suppose each particle
has a definite but ``hidden'' value of $S_z$, either $S_z = +\hbar/2$
or $S_z = -\hbar/2$; for conciseness, we denote these as $[+]$ or
$[-]$.  We then hypothesize that the two-particle quantum state
$$|\psi\rangle = \frac{1}{\sqrt{2}} \Big(|\!+\!z\rangle\otimes|\!-\!z\rangle \,-\, |\!-\!z\rangle\otimes|\!+\!z\rangle\Big)$$
is not an actual description of reality.  Instead, it corresponds to a
\textit{statistical} distribution of ``hidden variable'' states, which
we denote by $[+;-]$ (i.e., $S_z = +\hbar/2$ for particle $A$ and $S_z
= -\hbar/2$ for particle $B$), and $[-;+]$ (the other way around).
When the Earth experimentalist measures $S_z$, we uncover the value of
the hidden variables: a $+z$ result implies $[+;-]$, and a $-z$ result
implies $[-;+]$.  When the Betelgeuse experimentalist subsequently
measures $S_z$, the result obtained is the opposite of the Earth
result.  But this happens simply because those were the values of the
hidden variables all along---there is no physical influence traveling
instantly from Earth to Betelguese.

Of course, there is a huge amount of detail missing about the actual
contents of the ``hidden variables'' theory.  In particular, such a
theory would have to be consistent, with all the many successful
predictions of quantum mechanics.  This seems difficult, but with
enough hard work, one might imagine that the missing details can be
filled in.

\section{Bell's theorem}

In 1964, however, John S.~Bell published a
\hyperref[cite:bell]{bombshell paper} showing that the predictions of
quantum theory are \textit{inherently inconsistent} with hidden
variable theories.  The amazing thing about this result, which is
known as \textbf{Bell's theorem}, is that it requires no knowledge
about the details of the hidden variable theory, except that it is
deterministic and local.  In this section, we study a simplified
version of Bell's theorem, based on the presentation of
\hyperref[cite:mermin]{Mermin (1981)}.

We again consider spin-1/2 particle pairs, with particle $A$ on Earth
and particle $B$ in the Betelgeuse star system.  At each location, the
experimentalist can measure the local particle spin along three
distinct choices of spin axis.  These spin observables are denoted by
$S_1$, $S_2$, and $S_3$.  We will not specify the actual directions of
these spin axes until later in the proof.  (Note, however, that they
need not correspond to orthogonal spatial directions.)

We now imagine repeatedly preparing two-particle systems, and sending
the particles to the appropriate laboratories on Earth and Betelgeuse.
Each time, the prepared two-particle state is
$$|\psi\rangle = \frac{1}{\sqrt{2}} \Big(|\!+\!z\rangle\otimes|\!-\!z\rangle \,-\, |\!-\!z\rangle\otimes|\!+\!z\rangle\Big).$$
We then let the experimentalists perform measurements on their
particle.  In each instance of the experiment, each experimentalist
independently and randomly chooses one of the three possible axes for
measuring spin.  (It doesn't matter which experimentalist performs the
measurement first.)  The procedure is repeated many times, and both
experimentalists' choices of spin axis are recorded, along with their
measurement results.

\textcolor{red}{[Fig.]}

At the end of the run of experiments, the records are examined.  We
assume that the results are consistent with the predictions of quantum
theory.  Among other things, this means that whenever the
experimentalists happen to choose the same measurement axis (e.g.,
both measuring $S_1$), they always find opposite spins.

Can a hidden variables theory reproduce the results predicted by
quantum theory?  In a hidden variables theory, each particle must have
a definite value for each spin observable.  For example, particle $A$
might have $S_1 = +\hbar/2, \, S_2 = +\hbar/2, \, S_3 = -\hbar/2$.
For conciseness, we denote this by $[++-]$.  To be consistent with the
predictions of quantum theory, the hidden spin variables for the two
particles must have opposite values along each direction.  This means
that there are $8$ distinct possibilities, which we can denote as
$$\begin{aligned}{[}{+++};{---}], \;\;\; [{++-};{--+}], \;\;\; [{+-+};{-+-}], \;\;\; [{+--};{-++}],\\ [{-++};{+--}], \;\;\; [{-+-};{+-+}], \;\;\; [{--+};{++-}], \;\;\; [{---};{+++}].\end{aligned}$$
For instance, $[{++-};{--+}]$ indicates that for particle $A$, $S_1 =
S_2 = +\hbar/2$ and $S_3 = -\hbar/2$, while particle $B$ has the
opposite spin values, $S_1 = S_2 = -\hbar/2$ and $S_3 = +\hbar/2$.
So far, however, we don't know anything about the relative
probabilities of these 8 cases.

Let's now focus on the subset of experiments in which the two
experimentalists happened to choose \textit{different} spin axes
(e.g., Earth chose $S_1$ and Betelgeuse chose $S_2$).  Within this
subset, what is the probability \textit{for the two measurement
  results to have opposite signs} (i.e., one $+$ and one $-$)?  To
answer this question, we first look at the following 6 cases:
$$\begin{aligned}{[}{++-};{--+}], \;\;\; [{+-+};{-+-}], \;\;\; [{+--};{-++}],\\ [{-++};{+--}], \;\;\; [{-+-};{+-+}], \;\;\; [{--+};{++-}].\end{aligned}$$
These are the cases which do not have all $+$ or all $-$ for each
particle.  Consider one of these, say ${[}{++-};{--+}]$.  The two
experimentalists picked their measurement axes at random each time,
and amongst the experiments where they picked different axes, there
are only two ways for the measurement results to have opposite signs:
$(S_1,S_2)$ or $(S_2,S_1)$.  There are four ways to get the same sign:
$(S_1,S_3)$, $(S_2,S_3)$, $(S_3,S_1)$ and $(S_3, S_2)$.  Thus,
for this particular set of hidden variables, the probability for
measurement results with opposite signs is 1/3.  If we go through all
6 of the cases listed above, we find the same conclusion: the
probability for opposite signs is 1/3.

Now look at the remaining 2 cases:
$${[}{+++};{---}], \;\;\; [{---};{+++}].$$ For these, the
experimentalists will find results that have opposite signs with
probability 1.  Combining this with the findings from the previous
paragraph, we obtain the following statement:

\textit{Given that the two experimentalists choose different spin
  axes, the probability that their results have opposite signs is $P
  \ge 1/3$.}

This is called \textbf{Bell's inequality}.  If there is ever a
situation where quantum theory predicts a probability $P < 1/3$ (i.e.,
a violation of Bell's inequality), we will be forced to conclude that
quantum theory is inconsistent with local deterministic hidden
variables.  This conclusion holds regardless of the ``inner workings''
of the hidden variables theory, because the above derivation made no
assumptions about any such details (e.g., the probabilities of the
hidden variable states).

To complete the proof, we need to find a set of $S_1$, $S_2$, and
$S_3$ such that the predictions of quantum mechanics violate Bell's
inequality.  One simple choice is to align $S_1$ with the $z$ axis,
and align $S_2$ and $S_3$ along the $x$-$z$ plane at $120^\circ$
($2\pi/3$ radians) from $S_1$, as shown in the figure:

\textcolor{red}{[Fig.]}

Thus, the spin operators can be written in the eigenbasis of $\hat{S}_z$:
$$\begin{aligned}\hat{S}_1 &= \frac{\hbar}{2} \, \sigma_3 \\ \hat{S}_2 &= \frac{\hbar}{2} \, \left[\cos(2\pi/3) \sigma_3 + \sin(2\pi/3)\sigma_1\right]  \\   \hat{S}_3 &= \frac{\hbar}{2} \, \left[\cos(2\pi/3) \sigma_3 - \sin(2\pi/3)\sigma_1\right].\end{aligned}$$

Suppose the Earth experimentalist (measuring $A$) chooses $S_1$, and
obtains $+\hbar/2$.  The state of particle $A$ collapses to
$|\!+\!z\rangle$, and the state of particle $B$ collapses to
$|\!-\!z\rangle$.  The Betelgeuse experimentalist is assumed to
choose a different spin axis.  If the choice is $S_2$, the expectation
value of the measurement is
$$\begin{aligned}\langle\, - z \, | \, S_2 \,|-\!z\,\rangle &= \frac{\hbar}{2} \Big[\cos(2\pi/3) \langle\,- z\,|\sigma_3| - \!z\,\rangle + \sin(2\pi/3)\langle\,- z\,|\sigma_1|-\!z\,\rangle\Big]\\ &= \frac{\hbar}{2} \cdot \frac{1}{2} \end{aligned}$$
If $P_+$ and $P_-$ respectively denote the probability of measuring
$+\hbar/2$ and $-\hbar/2$ in this measurement, the above equation
implies that $P_+ - P_- = + 1/2$.  Moreover, $P_+ + P_- = 1$ by
probability conservation.  It follows that the probability of
obtaining a negative value (the opposite sign from the Earth
measurement) is $P_- = 1/4$.  All the other possible scenarios are
worked out in a similar way.  The overall conclusion is that if the
two experimentalists choose different measurement axis, they obtain
results of opposite signs with probability $1/4$, violating Bell's
inequality.

Last of all, we must consult Nature itself.  Is it possible to
observe, in an actual experiment, probabilities that violate Bell's
inequality?  In the decades following Bell's 1964 paper, many
experiments were performed to answer this question.  These experiments
are all substantially more complicated than the simple two-particle
spin-$1/2$ model that we've studied, and they are subject to various
uncertainties and ``loopholes'' that are beyond the scope of our
discussion.  But in the end, the experimental consensus appears to be
a clear \textit{yes}: Nature really does behave according to quantum
mechanics, and in a manner that cannot be replicated by deterministic
local hidden variables!  For a review of the experimental evidence,
see \hyperref[cite:aspect]{Aspect (1999)}.

\section{Entanglement entropy}
\label{sec:entropy}

In previous sections, we said that a multi-particle system is
``entangled'' if the individual particles do not have definite quantum
states.  It would be nice to make this statement more precise, by
formulating an expression for the level of entanglement in any given
system.  In fact, physicists have come up with several ways of
quantifying entanglement; in this section, we will describe the most
commonly-used one, called \textbf{entanglement entropy}, which is
closely related to the concept of entropy in thermodynamics,
statistical mechanics, and information theory.

Consider a quantum system with a $d$-dimensional Hilbert space
$\mathscr{H}$.  Given an arbitrary state $|\psi\rangle \in
\mathscr{H}$, we can define an operator
$$\hat{\rho}(\psi) = |\psi\rangle\, \langle\psi|.$$
This is simply the projection operator associated with $|\psi\rangle$,
but in this context we call it a \textbf{density matrix}.  (In the
language of linear algebra, $\hat{\rho}(\psi)$ is formed by taking the
``matrix outer product'' of the vector $|\psi\rangle$ with its
Hermitian conjugate.)  We use the notation $\hat{\rho}(\psi)$ to
emphasize that the density matrix is dependent on the chosen state
$|\psi\rangle$.  Evidently, it is a Hermitian operator; one eigenvalue
is 1 (with eigenvector $|\psi\rangle$), and the other $d-1$
eigenvalues are all $0$ (with eigenvectors given by $d-1$ vectors
spanning the subspace orthogonal to $|\psi\rangle$).

The expectation values of the density matrix have a special meaning.
Suppose the system is in state $|\psi\rangle$, and we perform a
measurement corresponding to a Hermitian operator $\hat{Q}$, which has
eigenvalues $\{q_1,q_2,\dots\}$ and eigenvectors
$\{|q_1\rangle,|q_2\rangle,\dots\}$.  The probability of
measuring $q_i$ is
$$P_i \;=\; |\langle q_i| \psi\rangle|^2 \;=\; \langle q_i |\psi\rangle \langle \psi|q_i\rangle \;=\; \langle q_i |\, \hat{\rho}(\psi)\, |q_i \rangle.$$
In other words, the expectation values of $\hat{\rho}(\psi)$
correspond to the probabilities for the outcomes of measurements
performed on $|\psi\rangle$.

Now suppose $\mathscr{H} = \mathscr{H}_A \otimes \mathscr{H}_B$, where
$\mathscr{H}_A$ and $\mathscr{H}_B$ are the Hilbert spaces of two
subsystems.  Starting from a density matrix of the combined system,
$\hat{\rho}(\psi)$, we can define a \textbf{``reduced'' density
  matrix} for subsystem A:
$$\hat{\rho}_A(\psi) = \mathrm{Tr}_B \,\big[\,\hat{\rho}(\psi)\,\big].$$
Here, $\mathrm{Tr}_B[\cdots]$ refers to a \textbf{partial trace},
tracing over the Hilbert space of subsystem B.  What's left after
this partial trace is an operator acting solely on the Hilbert space
$\mathscr{H}_A$.

To understand the meaning of $\hat{\rho}_A(\psi)$, let us work in
an explicit basis.  Let $\{|\mu_1\rangle, |\mu_2\rangle,\cdots\}$ be a
basis for $\mathscr{H}_A$, and $\{|\nu_1\rangle,
|\nu_2\rangle,\cdots\}$ a basis for $\mathscr{H}_B$.  A given state of
the combined system can be written in the tensor product basis:
$$|\psi\rangle = \sum_{ij} \psi_{ij} \, |\mu_i\rangle \otimes |\nu_j\rangle, \;\;\; \mathrm{where}\;\; \psi_{ij} \in \mathbb{C}.$$
The density matrix of the combined system is
$$\hat{\rho}(\psi) = |\psi\rangle \langle\psi| = \sum_{ii'jj'} \Big( |\mu_i\rangle \otimes |\nu_j\rangle\Big) \psi_{ij} \psi_{i'j'}^* \, \Big(\langle \mu_{i'}| \otimes \langle \nu_{j'}|\Big).$$
The density matrix for subsystem A is then obtained by tracing over
the states of subsystem B:
$$\begin{aligned}\hat{\rho}_A(\psi) &= \sum_j \langle \nu_j | \hat{\rho}(\psi) | \nu_j \rangle \\ &= \sum_{ii'} |\mu_i\rangle \,\rho_{ii'}^A \, \langle \mu_{i'}| \;\;\;\mathrm{where} \;\; \rho_{ii'}^A \equiv \sum_j \psi_{ij} \psi_{i'j}^*.\end{aligned}$$
Note that $\hat{\rho}_A(\psi)$ is clearly Hermitian.

The density matrix for subsystem A contains information about the
outcome probabilities for \textit{partial} measurements performed on
subsystem A.  Suppose the overall system is in state $|\psi\rangle$,
and we perform a measurement on subsystem A corresponding to a
Hermitian operator $\hat{\mu}$, which has eigenvectors
$\{|\mu_1\rangle,|\mu_2\rangle,\dots\}$.  Then the probability of
measuring $\mu_i$ is
$$\langle \mu_i | \hat{\rho}_A |\mu_i \rangle \,=\, \rho_{ii}^A \,=\, \sum_{j} |\psi_{ij}|^2.$$
(Note that this is consistent with the discussion of partial
measurements in Section~\ref{sec:partialmeasurements}.)  In other
words, the diagonal matrix elements of $\hat{\rho}_A$, in any basis,
specify the outcome probabilities for measurements corresponding to
that basis.  Conservation of probability implies that $\sum_i
\rho^A_{ii} = 1$, and hence that
$$\mathrm{Tr}\big[\rho_A\big] = 1.$$

In entangled systems, the density matrix essentially takes over the
theoretical role of the quantum state.  In a single-particle or
unentangled system, we can define the quantum state and use it to
calculate the probabilities of various measurement outcomes.  However,
a system has no definite quantum state if it is entangled with other
systems.  Probabilities are instead calculated from the density
matrix.

Given a density matrix, we can define a quantity called the
\textbf{entanglement entropy}:
$$S_{\mathrm{ent.}} = - k_B \mathrm{Tr} \Big\{ \hat{\rho}\, \ln\!\big[\hat{\rho}\big]\Big\}.$$
Here, $S_{\mathrm{ent.}}$ refers to the entropy of the system or
subsystem described by the density matrix $\hat{\rho}$.  For instance,
if we use the density matrix $\hat{\rho}_A$ in the above formula, we
get the entropy of subsystem A.  In this formula, $\ln[\cdots]$
denotes the logarithm of an operator, which is the inverse of
the exponential: $\ln(\hat{A}) = \hat{B} \Rightarrow \exp(\hat{B}) =
\hat{A}$.  The prefactor $k_B$ is Boltzmann's constant, and is present
to ensure that $S_{\mathrm{ent.}}$ has the same units as thermodynamic
entropy.

The definition of the entanglement entropy is based on an analogy with
the entropy concept from classical thermodynamics, statistical
mechanics and information theory.  In those classical contexts,
entropy is a quantitative measure of uncertainty (i.e, lack of
information) about a system's underlying microscopic state, or
``microstate''.  Suppose a system has $W$ possible microstates that
occur with probabilities $\{p_1, p_2, \dots, p_W\}$, satisfying
$\sum_i p_i = 1$.  Then we define the classical entropy
$$S_{\mathrm{cl.}} = - k_B \sum_{i=1}^W p_i \ln(p_i).$$
In a situation of complete certainty where the system is known to be
in a specific microstate $k$ ($p_i = \delta_{ik}$), the formula gives
$S_{\mathrm{cl.}} = 0$.  (Note that $x \ln(x)\rightarrow 0$ as
$x\rightarrow 0$).  In a situation of complete uncertainty where all
microstates are equally probable ($p_i = 1/W$), we get
$S_{\mathrm{cl.}} = k_B \ln W$, the entropy of a microcanonical
ensemble in statistical mechanics.  For any other distribution of
probabilities, it can be shown that the entropy lies between these two
extremes: $0 \le S_{\mathrm{cl.}}  \le k_B\ln W$.  For a review of the
properties of entropy, see \hyperref[appendixa]{Appendix A}.

Entanglement entropy quantifies the uncertainty arising from an
entangled system's lack of a definite quantum state.  But we need to
be careful when extending classical notions of probability to quantum
systems.  As described above, when performing a measurement whose
possible outcomes are $\{\mu_1, \mu_2, \dots\}$, the probability of
getting $\mu_i$ is $p_i = \langle \mu_i | \hat{\rho}|\mu_i\rangle$.
However, it is problematic to directly substitute these probabilities
$\{p_i\}$ into the classical entropy formula, since they are
basis-dependent (i.e., the set of probabilities is dependent on the
choice of measurement).  The above definition for the entanglement
entropy, $S_{\mathrm{ent.}} = - k_B \mathrm{Tr} \big\{ \hat{\rho}\, \ln[\hat{\rho}]
\big\}$, bypasses this problem by using the trace, which is
basis-independent.  This also means that if we look specifically at
the eigenbasis for $\hat{\rho}$, denoted by $\{|\mu'_i\rangle\}$, then
$$\begin{aligned}S_{\mathrm{ent.}} &= -k_B \sum_i \langle \mu'_i | \hat{\rho}\ln(\hat{\rho}) | \mu'_i\rangle  \\ &= - k_B \sum_i p_i' \ln(p_i') \;\;\;\mathrm{where}\;\;\hat{\rho}\,|\mu'_i\rangle = p_i' |\mu'_i\rangle\end{aligned}$$
In this specific basis, where the eigenvalues $p_i =
\langle\mu'_i|\hat{\rho}|\mu'_i\rangle$ are also the measurement
probabilities, the definition of the entanglement entropy is
consistent with the definition of the classical entropy.

For an unentangled system, we can write $\hat{\rho} =
|\psi\rangle\langle\psi|$ for some state $|\psi\rangle$, so
$|\psi\rangle$ is itself an eigenstate of $\hat{\rho}$ with eigenvalue
1; hence, $S_{\mathrm{ent.}} = 0$.  Conversely, if we find that a
system has $S_{\mathrm{ent.}} \ne 0$, that implies that it is
entangled, and lacks a definite quantum state.

A system is \textbf{maximally entangled} if the density matrix
eigenvalues (measurement probabilities) are all equal: i.e., $p_i' =
1/d$, where $d$ is the dimension of the Hilbert space.  In that case,
the entropy is $S_{\mathrm{ent.}} = k_B \ln(d)$.  In general, the
entanglement entropy lies between these two extremes, $0 \le
S_{\mathrm{ent.}} \le k_B\ln(d)$.

As a concrete example, let us apply the entanglement entropy formula
to the system of two spin-$1/2$ systems and the state
$$|\psi\rangle = \frac{1}{\sqrt{2}} \Big(|\!+\!z\rangle\otimes|\!-\!z\rangle \,-\, |\!-\!z\rangle\otimes|\!+\!z\rangle\Big).$$
The density matrix for the two-particle system is
$$\hat{\rho}(\psi) = \frac{1}{2} \Big(|\!+\!z\rangle\otimes|\!-\!z\rangle \,-\, |\!-\!z\rangle\otimes|\!+\!z\rangle\Big) \Big(\langle+z|\otimes\langle-z| \,-\, \langle-z|\otimes\langle+z|\Big).$$
Tracing over the B subsystem gives the reduced density matrix:
$$\hat{\rho}_A(\psi) = \frac{1}{2} \Big(|\!+\!z\rangle \langle+z| \,+\, |\!-\!z\rangle \langle-z|\Big).$$
This can be expressed as a matrix in the
$\{|\!+z\rangle,|\!-z\rangle\}$ basis:
$$\hat{\rho}_A(\psi) = \begin{pmatrix}\frac{1}{2} & 0 \\ 0 & \frac{1}{2}\end{pmatrix}.$$
Note that this satisfies $\mathrm{Tr}(\hat\rho_A) = 1$, as expected.
Next, we can use $\hat{\rho}_A$ to compute the entropy of subsystem A:
$$S_{\mathrm{ent.}}^A = -k_B\mathrm{Tr}\left\{\hat{\rho}_A\ln(\rho_A)\right\} = -k_B\mathrm{Tr}\begin{pmatrix}\frac{1}{2}\ln\left(\frac{1}{2}\right) & 0 \\ 0 & \frac{1}{2}\ln\left(\frac{1}{2}\right)\end{pmatrix} = k_B\ln(2).$$
This is the maximum possible entanglement entropy for a system with a
2D Hilbert space.  Hence, subsystems A and B are maximally entangled.

\section{The Many-Worlds Interpretation}

We conclude this chapter by discussing a set of compelling but
controversial ideas arising from the phenomenon of quantum
entanglement: the \textbf{Many-Worlds Interpretation} of quantum
theory.

So far, when describing the phenomenon of state collapse, we have
relied on the measurement postulate (see
Section~\ref{sec:partialmeasurements}), which is part of the
\textbf{Copenhagen Interpretation} of quantum mechanics.  This is how
quantum mechanics is typically taught, and how physicists think about
the theory when doing practical, everyday calculations.

However, the measurement postulate has two bad features:

First, it seems to stand apart from the other postulates of quantum
mechanics, in being the only place where randomness (or
``indeterminism'') creeps into quantum theory.  The other postulates
do not refer to probabilities.  In particular, the Schr\"odinger equation
$$i\hbar\frac{\partial}{\partial t}|\psi(t)\rangle = \hat{H}(t) |\psi(t)\rangle$$
is completely deterministic, in the sense that if you are told the
Hamiltonian $\hat{H}(t)$ and the state vector $|\psi(t_0)\rangle$ at a
certain time $t_0$, then you can (in principle) solve the
Schr\"odinger equation to determine $|\psi(t)\rangle$ for all $t$.
This time-evolution consists of a smooth, non-random rotation of the
state vector within its Hilbert space.  A measurement process, on the
other hand, has a completely different effect: it causes the state
vector to instantaneously jump to a randomly-selected value.  It is
strange that the theory contains two fundamentally different ways for a
state to change.

The second noteworthy feature of the measurement postulate is that it
is silent on what exactly constitutes a measurement.  Does a
measurement require a conscious observer?  Surely not: in the words of
Einstein, could we really believe that the Moon exists only when we
look at it?  But then if a given device interacts with a particle, how
do we figure out whether the interaction amounts to a measurement, or
whether it affects the particle ``merely'' via the Schr\"odinger equation?

The Many-Worlds Interpretation seeks to resolve these problems by
positing that \textit{the measurement postulate is \underline{not} a
  fundamental postulate of quantum mechanics}.  Rather, what we call
``measurement'' (including state collapse and the apparent randomness
of measurement results) is an emergent phenomenon that can be derived
from the behavior of complex many-particle quantum systems obeying the
Schr\"odinger equation.

The key idea is to model a measurement process by applying the
Schr\"odinger equation to a multi-particle quantum system, one that
contains both the particle being measured \textit{and} the measurement
apparatus itself.  Suppose, for example, that we have a spin-$1/2$
particle, and an apparatus designed to measure $S_z$.  Let
$\mathscr{H}_p$ denote the 2D Hilbert space for the particle's
spin-$1/2$ degree of freedom.  Let $\mathscr{H}_a$ denote the Hilbert
space of the apparatus, which is \textit{much} larger and more
complicated than $\mathscr{H}_p$ (since it has to describe the
$10^{23}$ or more atoms in the apparatus).  The Hilbert space of the
combined system can be written as
$$\mathscr{H} = \mathscr{H}_p \otimes \mathscr{H}_a.$$
The system is prepared in an initial state of the form
$$|\psi(0)\rangle = |\!+x\rangle \otimes |\textbf{START}\rangle,$$
where $|\!+x\rangle \in \mathscr{H}_p$ is an initial state of the
particle that does not have a definite value of $S_z$, while
$|\textbf{START}\rangle \in \mathscr{H}_a$ denotes the initial state
of the apparatus.

The combined system then undergoes time evolution via the
Schr\"odinger equation, with an extremely complicated Hamiltonian that
describes how the apparatus acts upon the particle to perform an $S_z$
measurement.  After some time $T$ has passed, the state evolves to
$$|\psi(T)\rangle \;=\; |\!+z \rangle \otimes |\textbf{RESULT}\uparrow\rangle \;+\; |\!-z\rangle\otimes |\textbf{RESULT}\downarrow\rangle, $$
where $|\textbf{RESULT}\uparrow\rangle, \,
|\textbf{RESULT}\downarrow\rangle \in \mathscr{H}_a$ are complicated
multi-particle state vectors of the apparatus.  For example, the
apparatus might have a centimeter-size pointer (of $\sim 10^{23}$
atoms) that indicates the measurement result;
$|\textbf{RESULT}\uparrow\rangle$ describes the apparatus with the
pointer pointing up (indicating a $+z$ result), while
$|\textbf{RESULT}\downarrow\rangle$ describes the apparatus with the
pointer pointing down (indicating a $-z$ result).

In the above expression for $|\psi(T)\rangle$, the two terms can be
interpreted as two ``alternative histories'' of the system, or two
different ``worlds''.  In one world, the pointer on the apparatus
indicates a $+z$ measurement result, and the quantum state of the
particle has ``collapsed'' to $|\!+z\rangle$.  In the other world, the
pointer indicates a $-z$ result, and the particle state has
``collapsed'' to $|\!-z\rangle$.  Moreover, with a bit more thought we
can work out that the relative weights of the two terms correspond to
the probabilities of obtaining each measurement result.  Thus, the
effects of the measurement postulate (i.e., random measurement results
and the phenomenon of state collapse) arise from restricting our
attentions to a single world.

This description can be generalized to the universe as a whole.  In
the viewpoint of the Many-Worlds Interpretation, the entire universe
can be described by a mind-bogglingly complicated quantum state,
evolving deterministically according to the Schr\"odinger equation.
This evolution involves repeated ``branchings'' of the universal
quantum state, which continuously produces more and more worlds.  The
classical world that we appear to inhabit is just one amidst a vast
multitude of worlds.  It is up to you to decide whether this
conception of reality seems reasonable.  It is essentially a matter of
preference, because the Copenhangen Interpretation and the Many-Worlds
Interpretation have identical physical consequences (which is why they
are referred to as different ``interpretations'' of quantum mechanics,
rather than different theories).

\section*{Appendix A: Entropy in classical physics}
\label{appendixa}

In classical physics, ``entropy'' quantifies our lack of knowledge
about a complex system.  In information theory, \textbf{information
  entropy} describes our uncertainty about the contents of a message
before receiving it.  In thermodynamics and statistical mechanics,
\textbf{thermodynamic entropy} describes our uncertainty about the
microscopic details (``microstate'') of a complex many-body system.
In fact, these two formulations are consistent with each other; for a
description of how they are related, see Feynman
\textcolor{red}{(???)}.  Here, we will use the statistical mechanics
formulation.

Suppose a system has discrete microstates labeled by integers
$\{1,2,3,\dots\}$ , which have probabilities $\{p_1, p_2, p_3,
\dots\}$.  Then the entropy of the system is defined as
$$S_{\mathrm{cl.}} = - k_B \sum_{i} p_i \ln(p_i).$$
The probabilities $p_i$ are to be understood as \textit{conditional}
probabilities, conditioned on the known facts about the macroscopic
features of the system (e.g., we might know the total energy $E$).

If we know exactly which microstate the system is in (i.e., $p_k = 1$
for some state $k$), the entropy formula gives $S _{\mathrm{cl.}} =
0$.  The opposite situation, of ``complete uncertainty'', is provided
by a \textbf{micro-canonical ensemble}: a system that is at
equilibrium, has a fixed total energy $E$, and does not interact with
the rest of the universe.  In this case, the \textbf{ergodicity
  postulate} of statistical mechanics states that all possible
microstates with energy $E$ are equally probable.  If there are $W$
possible microstates, the probabilities are
$$p_i = \frac{1}{W} \;\;\forall i \in \{1,2,\dots,W\}.$$
Therefore, the entropy formula gives
$$S_{\mathrm{cl.}} \,=\, -k_B W \frac{1}{W} \ln(1/W) \;=\; k_B \ln W,$$
the famous result carved into the gravestone of Ludwig Boltzmann
(\textcolor{red}{???}).  Note that this expression contains an
implicit energy dependence; changing $E$ also varies $W$, and hence
$S_{\mathrm{cl.}}$.

The entropy formula is designed so that any other probability
distribution, which describes a situation of \textit{partial}
uncertainty, yields an entropy $S_{\mathrm{cl.}}$ lying between $0$
and $k_B \ln W$.  To see that zero is the lower bound, first note that
for $0 \le p_i \le 1$, each term in the entropy formula satisfies
$-k_B\, p_i\ln(p_i) \ge 0$, and the equality holds if and only if $p_i
= 0$ or $p_i = 1$ (see the figure below).  This implies that
$S_{\mathrm{cl.}}\ge 0$, and moreover that $S_{\mathrm{cl.}} = 0$ if
and only if $p_i = \delta_{ik}$ for some $k$ (i.e., there is no
uncertainty).  \textcolor{red}{The proof that $k_B \ln W$ is the upper
  bound (which implies the second law of thermodynamics) is more
  complicated, and we refer the interested reader to [????].}

\textcolor{red}{[Fig.]}

Another important feature of the entropy formula is that
$S_{\mathrm{cl.}}$ is \textbf{extensive}, meaning that it scales
(``extends'') proportionally with system size.  To see this, consider
two independent systems A and B, which have microstate probabilities
$\{p_i^A\}$ and $\{p_j^B\}$.  If we regard the combination of A and B
as a single system, each microstate of the combined system is
specified by the microstate of A and the microstate of B, and is thus
indexed by integers $(i,j)$, with probability $p_{ij} = p^A_ip^B_j$.
The entropy of the combined system is therefore
$$\begin{aligned}S_{\mathrm{cl.}} &= - k_B \sum_{ij} p_i^Ap^B_j \ln\left(p^A_ip^B_j\right) \\
  &= - k_B \Big(\sum_{i} p^A_i \ln p^A_i\Big)\Big(\sum_j p^B_j\Big)
  - k_B \Big(\sum_{i} p^A_i \Big) \Big(\sum_j p^B_j \ln p^B_j\Big) \\
  &= S_{\mathrm{cl.}}^A + S_{\mathrm{cl.}}^B,
\end{aligned}$$
where $S_{\mathrm{cl.}}^A$ and $S_{\mathrm{cl.}}^B$ are the individual
entropies of the A and B subsystems.

This has important consequences for the behavior of $W(E)$, the number
of microstates at each energy $E$.  Suppose we extend a system by
adding micro-canonical subsystems (which are insulated from each
other).  In the process, both $E$ and $S_{\mathrm{cl.}}$ increase
proportionally.  Since $S_{\mathrm{cl.}} \propto \ln[W(E)]$,
$$E \propto \ln W \;\;\;\Rightarrow \;\;\;W(E) \propto e^{\beta_0 E} \;\;\; \mathrm{for\;some}\; \beta_0 > 0.$$
If we relax the restriction that the additional subsystems are
micro-canonical, the number of microstates grows even faster with $E$,
as energy can now be distributed in different ways between the
subsystems.  It is reasonable to assume that the scaling is a
faster-growing exponential,
$$W(E) \propto e^{\beta E} \;\;\; \mathrm{for\;some}\; \beta \ge \beta_0.$$
This implies that the constant of proportionality relating
$S_{\mathrm{cl.}}$ and $E$ is
$$\frac{\partial E}{\partial S_{\mathrm{cl.}}} = \frac{1}{k_B \beta} = T,$$
where $\beta \equiv (k_BT)^{-1}$ \textit{defines} the temperature $T$.
This is the first law of thermodynamics.

A \textbf{canonical ensemble} is a system held in equilibrium with a
larger system, called a ``heat bath''.  We can model this with a
micro-canonical ensemble of energy $E$, divided two subsystems, A (the
canonical ensemble) and B (the heat bath), which are not insulated
from each other.  Using the ergodicity postulate, and the
aforementioned exponential scaling of $W$ with $E$, one can show that
the probability for subsystem A to have energy $E_A$ is
$$p_A(E_A) \propto W_A(E_A) \, e^{-\beta E_A},$$
where $W_A(E_A)$ is the number of microstates of energy $E_A$ for
subsystem A, and $\beta$ is the inverse temperature of the heat bath.
This is the celebrated \textbf{Boltzmann law} for the distribution of
microstates at fixed temperature.  It implies that each microstate
$i$, of energy $E_i$, has probability
$$p_i = \frac{\exp(-\beta E_i)}{Z}, \;\;\;\mathrm{where}\;\;Z \equiv \sum_j \exp(-\beta E_j).$$
Here, $Z(\beta,E_1, E_2,\dots)$ is the \textbf{partition function}.
Note that $p_i$ satisfies probability conservation, $\sum_i p_i = 1$,
and that the sum involves all microstates of all possible energies.

The probability distribution for a canonical ensemble represents a
partial-uncertainty situation, since lower-energy microstates are
more probable than higher-energy microstates.  Plugging the above
expression for $p_i$ into the entropy formula gives:
$$S_{\mathrm{cl.}} = \frac{1}{T} \frac{\sum_i E_i e^{-\beta E_i}}{\sum_i e^{-\beta E_i}} + k_B \ln Z \;=\; \frac{\langle E\rangle}{T} + k_B \ln Z,$$
where $\langle E\rangle = \sum_i E_i p_i$ denotes the average energy.
We can then define
$$F \,\equiv\, - k_B T \ln Z \,=\, \langle E \rangle - TS_{\mathrm{cl.}},$$
and show that this satisfies $\partial F/\partial T = -
S_{\mathrm{cl.}}$.  This quantity can thus be identified as the
thermodynamic \textbf{free energy}.



\section*{Exercises}

\begin{enumerate}
\item \textcolor{red}{Prove that the inner product for tensor product basis states satisfies inner product axiom.}
\item \textcolor{red}{Calculate entropy for Bell states.}
\item \textcolor{red}{Explicit model for MWI state collapse.}
\end{enumerate}

\section*{Further Reading}

\begin{itemize}
\item A.~Einstein, B.~Podolsky, and N.~Rosen,
  \textit{Can Quantum-Mechanical Description of Physical Reality Be
    Considered Complete?}, Physical Review \textbf{47}, 777 (1935).
  \label{cite:epr}

\item J.~S.~Bell, \textit{On the Einstein-Podolsky-Rosen paradox},
  Physics \textbf{1}, 195 (1964). \label{cite:bell}
  
\item N.~D.~Mermin, \textit{Bringing home the atomic world: Quantum
  mysteries for anybody}, American Journal of Physics \textbf{49}, 940
  (1981). \label{cite:mermin}

\item A.~Aspect, \textit{Bell's inequality test: more ideal than ever},
  Nature (News and Views) \textbf{398}, 189 (1999). \label{cite:aspect}

\item \textcolor{red}{Feynman lectures on computation}
  \label{cite:feynman}
\end{itemize}

\end{document}


%% For decades after the discovery of quantum mechanics, the quantum
%% double-slit experiment was just a ``thought experiment'', meant to
%% illustrate the features of quantum mechanics that had been uncovered
%% by other, more complicated experiments.  Nowadays, the most convenient
%% way to do the experiment is with light, using single-photon sources
%% and single-photon detectors.  Quantum interference has also been
%% demonstrated experimentally using electrons, neutrons, and even
%% large-scale particles such as buckyballs.
