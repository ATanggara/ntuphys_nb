\documentclass[pra,11pt]{revtex4}
\usepackage{amsmath}
\usepackage{amssymb}
\usepackage{graphicx}
\usepackage{color}
\usepackage{mathrsfs}

\def\ket#1{\left|#1\right\rangle}
\def\bra#1{\left\langle#1\right|}
\def\braket#1{\left\langle#1\right\rangle}

\setlength{\parindent}{0pt}

\renewcommand{\baselinestretch}{1.0}
\setlength{\parskip}{0.07in}

\begin{document}

\section{Entangled quantum states}

So far, we have studied quantum mechanical systems consisting of
single particles.  The next important step is to look at systems
containing more than one particle.  As we shall see, the postulates of
quantum mechanics have interesting implications which only show up in
multi-particle systems, such as the phenomenon of quantum
entanglement.

In the following discussion, you may assume that we are dealing with
systems of ``distinguishable'' particles.  A simple example of such a
system is a two-particle system containing different types of
particles, like a proton and a neutron.  There is a separate set of
considerations to take into account whenever we're dealing with
systems of ``identical'' particles, such as a system composed of two
electrons, where the electrons cannot be distinguished from each other
even in principle.  We will spend the next chapter dealing with those
special complications.  If you're not sure what this paragraph is
talking about, don't worry; just read on.

We begin by considering a system of two particles, labeled $A$ and
$B$.  If each individual particle is treated as a quantum system, then
according to the usual postulates of quantum mechanics, its state is
described by a vector in some complex Hilbert space.  Let
$\mathscr{H}_A$ and $\mathscr{H}_B$ denote the respective Hilbert
spaces for these two particles.  When the two particles are considered
as a single quantum system, that system has its own Hilbert space,
denoted by
$$\mathscr{H} = \mathscr{H}_A\otimes \mathscr{H}_B.$$
The symbol $\otimes$ refers to a mathematical operation known as the
\textbf{tensor product}, which is a way of combining two Hilbert
spaces to form a larger Hilbert space.

The mathematical meaning of the tensor product is as follows.  Suppose
$\mathscr{H}_A$ can be spanned by an orthonormal basis
$\{|\mu_1\rangle, |\mu_2\rangle, |\mu_3\rangle, \dots\}$, and
$\mathscr{H}_B$ can be spanned by $\{|\nu_1\rangle, |\nu_2\rangle,
|\nu_3\rangle, \dots\}$.  Then the tensor product space $\mathscr{H}_A
\otimes \mathscr{H}_B$ can be defined in terms of the basis set
$$\Big\{\;\,|\mu_i\rangle\otimes|\nu_j\rangle \;\;  \Big| \;\; \textrm{all}\;|\mu_i\rangle,\; |\nu_j\rangle \;\,\Big\}.$$
Intuitively, this means that if you know particle $A$ is in state
$|\mu_i\rangle$, and particle $B$ is in state $|\nu_j\rangle$, that
specifies a state of the combined system.  A complete basis set for
the combined system can be constructed in this way.  Note that if
$\mathscr{H}_A$ has dimension $d_A$ and $\mathscr{H}_B$ has dimension
$d_B$, then $\mathscr{H}_A \otimes \mathscr{H}_B$ has dimension $d_A
d_B$.

Any two-particle state can then be written as
$$|\psi\rangle = \sum_{i} \sum_{j} \, c_{ij}\; |\mu_i\rangle \otimes |\nu_j\rangle.$$
The inner product for tensor product basis states is defined in terms
of the inner product of the individual spaces:
$$\Big(|\mu_i\rangle \otimes |\nu_j\rangle\;,\; |\mu_p\rangle \otimes |\nu_q\rangle \Big) \;\equiv\; \big(\langle\mu_i| \otimes \langle\nu_j| \big) \big(|\mu_p\rangle \otimes |\nu_q\rangle\big) \;\equiv\; \langle\mu_i|\mu_p\rangle \, \langle\nu_j|\nu_q\rangle.$$
In other words, if you need to multiply a tensor product bra with a
tensor product ket, the procedure is to (i) do the bra-ket product for
the first slot, (ii) do the bra-ket product for the second slot, and
(iii) multiply the results together.  It is straightforward to check
that this satisfies the usual requirements for an inner product in
linear algebra (assuming that the inner products for the
single-particle states are already well-defined).

As an example, suppose that both $\mathscr{H}_A$ and $\mathscr{H}_B$
are two-dimensional Hilbert spaces describing spin-$\frac{1}{2}$
degrees of freedom, such that each can be spanned by an orthonormal
basis $\{\,|\!+\!z\rangle, \,|\!-\!z\rangle \, \}$, representing
``spin-up'' and ``spin-down'' states.  Then the tensor product space
$\mathscr{H}$ is spanned by
$$\Big\{\;|\!+\!z\rangle\otimes|\!+\!z\rangle\,,\; |\!+\!z\rangle\otimes|\!-\!z\rangle\,,\; |\!-z\!\rangle\otimes|\!+\!z\rangle\,,\; |\!-\!z\rangle\otimes|\!-\!z\rangle \;\Big\}.$$
The dimension of $\mathscr{H}$ is $2 \times 2 = 4$.  

At this point, we must make an extremely crucial observation.  As
mentioned, if we specify that particle $A$ is in state $|\mu_i\rangle$
and particle $B$ is in state $|\nu_j\rangle$, that specifies a state
of the combined system.  But the reverse is not true!  \textit{Some
  states of the combined system cannot be expressed in terms of
  definite states of the individual particles.}  For example, take the
previous example of two spaces describing spin-$\frac{1}{2}$ degrees
of freedom, and consider
$$|\psi\rangle = \frac{1}{\sqrt{2}} \Big(|\!+\!z\rangle\otimes|\!-\!z\rangle \,-\, |\!-z\!\rangle\otimes|\!+\!z\rangle\Big).$$

This two-particle state is constructed from two of the basis states of
the tensor product space,
$|\!+\!z\rangle\otimes|\!-\!z\rangle$ and
$|\!-\!z\rangle\otimes|\!+\!z\rangle$; you can check that
the factor of $1/\sqrt{2}$ correctly normalizes the state so that
$\langle\psi|\psi\rangle = 1$.  Evidently, neither particle A nor
particle B has a definite $|\!+\!z\rangle$ or
$|\!-\!z\rangle$ state.  Furthermore, we shall show rigorously,
in a later section, that there's \textit{no} choice of basis in which this
particular quantum state $|\psi\rangle$ can be expressed with
definite quantum states for each particle.  In other words,
$$|\psi\rangle \ne |\psi_A\rangle\otimes|\psi_B\rangle \;\;\;\textrm{for}\;\textrm{any}\;\; |\psi_A\rangle, \;|\psi_B\rangle.$$
In such a situation, the two particles are said to be
\textbf{entangled}.  This means that they cannot individually be
described by single-particle quantum states; their quantum
mechanical descriptions are intrinsically ``mixed up'' with one another.

For systems of more than two particles, the multi-particle quantum
states are defined using multiple tensor products.  Suppose we have
$N$ particles, with single-particle Hilbert spaces $\mathscr{H}_1,
\mathscr{H}_2, \dots, \mathscr{H}_N$, of dimensionality $d_1, \dots,
d_N$.  Then the multi-particle Hilbert space is $\mathscr{H} =
\mathscr{H}_1 \otimes \mathscr{H}_2 \otimes \cdots \otimes
\mathscr{H}_N$, which has dimensionality $d = d_1 d_2\cdots d_N$.  It
is interesting to note that the dimensionality of the multi-particle
Hilbert space scales \textit{exponentially} with the number of
particles.  For instance, if each single-particle Hilbert space is
two-dimensional, a $20$-particle system will have a Hilbert space of
dimensionality $2^{20} =1\,048\,576$.  This means that multi-particle
quantum systems are able to ``carry'' huge amounts of information,
compared to classical systems.  That is one of the reasons people are
excited by the prospect of quantum computing.

\section{Partial measurements}

Let us recall how measurements work in single-particle quantum
theory.  Suppose a particle is in a quantum state $|\psi\rangle$
and we perform some measurement $\hat{Q}$.  To figure out the
measurement outcome probabilities, we first write $|\psi\rangle$ in
the eigenstate basis of $\hat{Q}$:
$$|\psi\rangle = \sum_n |n\rangle\,\psi_n, \;\;\mathrm{where}\;\;\hat{Q}|n\rangle = q_n |n\rangle \;\,\textrm{and}\;\, \psi_n = \langle n|\psi\rangle.$$
For simplicity, let's suppose the eigenvalues are all non-degenerate.
Then the probability of obtaining the measurement outcome $q_n$ is
$|\psi_n|^2$, the absolute square of the coefficient of
$|n\rangle$ in the above formula.  After the measurement, the system
``collapses'' into state $|n\rangle$, which can be regarded as the
projection operation
$$|\psi\rangle +\!x \frac{1}{\sqrt{\mathcal{N}}}\; |n\rangle\langle n|\psi\rangle.$$
The prefactor $\mathcal{N} = |\psi_n|^2$ simply ensures that the new
state remains normalized; the overall phase has no physical
significance.

For multi-particle systems, the measurement postulates of quantum
theory are faced with a new complication: how do we describe a
measurement that is performed on just one particle?

The generalization turns out to be not so difficult.  Let's again
consider a system of two particles, with Hilbert spaces
$\mathscr{H}_A$ and $\mathscr{H}_B$; the two-particle Hilbert space is
$\mathscr{H} = \mathscr{H}_A \otimes \mathscr{H}_B$.  Suppose we
perform a measurement on particle $A$, described by the operator
$\hat{Q}$ which acts upon $\mathscr{H}_A$.  We can write any two-particle
state $|\psi\rangle$ using the eigenvectors of $\hat{Q}$ as a basis
set for $\mathscr{H}_A$, and an arbitrary basis set for
$\mathscr{H}_B$:
$$\begin{aligned}|\psi\rangle &= \sum_{n}\sum_{\mu} |n\rangle\otimes |\mu\rangle \;\psi_{n\mu} \\&= \sum_n |n\rangle\otimes |n'\rangle, \;\;\mathrm{where}\;\,|n'\rangle\equiv \sum_\mu |\mu\rangle \;\psi_{n\mu} \in \mathscr{H}_B.\end{aligned}$$
Note that unlike the single-particle case, the ``coefficient'' of
$|n\rangle$ in the basis expansion is not merely a complex number, but
a vector in the space $\mathscr{H}_B$.  We can proceed by
analogy.  The probability of obtaining the measurement outcome $q_n$
should be the ``absolute square'' of this:
$$\langle n'|n'\rangle = \sum_\mu |\psi_{n\mu}|^2.$$
After the measurement, the ``collapse'' of the state is described by
the projection operation
$$|\psi\rangle +\!x \frac{1}{\sqrt{\mathcal{N}}}\; \Big(|n\rangle\langle n|\Big) |\psi\rangle = \frac{1}{\sqrt{\mathcal{N}}}\; |n\rangle\otimes |n'\rangle.$$
Note that the projection operator $|n\rangle\langle n|$ only acts on
the $\mathscr{H}_A$ part of the two-particle space.  The factor
$\mathcal{N}$ is again defined so that the resulting state is
normalized to unity.

Let's work through an explicit example, involving the
previously-discussed system of two spin-$\frac{1}{2}$ particles, and
the two-particle entangled state
$$|\psi\rangle = \frac{1}{\sqrt{2}} \Big(|\!+\!z\rangle\otimes|\!-\!z\rangle \,-\, |\!-\!z\rangle\otimes|\!+\!z\rangle\Big).$$
Here, $|\!+\!z\,\rangle$ and $|\!-\!z\,\rangle$ denote eigenstates
of the operator $\hat{S}_z$, with the eigenvalues $+\hbar/2$ and $-\hbar/2$
respectively.  Let us perform an $S_z$ measurement on particle A.  The
results are as follows:
\begin{itemize}
\item A measurement outcome of $+\hbar/2$ has probability $P_+ = \langle
  n'|n'\rangle$, where $|n'\rangle =
  (1/\sqrt{2})\,|\!-\!z\rangle$.  Hence, $P_+ = 1/2$.  After the
  measurement, the state collapses to $|\!+\!z\rangle
  \otimes|\!-\!z\rangle$.

\item A measurement outcome of $-\hbar/2$ has probability $P_- = \langle
  n'|n'\rangle$, where $|n'\rangle =
  (1/\sqrt{2})\,|\!+\!z\rangle$.  Hence, $P_- = 1/2$.  After the
  measurement, the state collapses to $|\!-\!z\rangle
  \otimes|\!+\!z\rangle$.
\end{itemize}
In other words, we obtain the two possible results $\pm \hbar/2$ with
equal probability; in either case, the two-particle state collapses so
that particle $A$ is in the spin eigenstate corresponding to our
measurement result, while particle $B$ has the opposite spin.  After
collapse, the two-particle state is no longer entangled.

\section{The Einstein-Podolsky-Rosen ``paradox''}

In 1935, Einstein, Podolsky, and Rosen (EPR) formulated a thought
experiment known as the \textbf{EPR paradox}, which highlighted the
exceptionally counter-intuitive features of entangled quantum states.
They argued that because of these features, quantum theory had to be
an incomplete theory of reality.  Subsequently, however, it was shown
that the EPR paradox is not an actual paradox; physical systems
\textit{really do} exhibit the strange behaviors showcased in this
thought experiment.

The EPR thought experiment begins with an entangled state, such as the
entangled spin-$1/2$ two-particle state described in the previous section:
$$|\psi\rangle = \frac{1}{\sqrt{2}} \Big(|\!+\!z\rangle\otimes|\!-\!z\rangle \,-\, |\!-\!z\rangle\otimes|\!+\!z\rangle\Big).$$
As previously discussed, we can measure $\hat{S}_z$ on particle $A$.
The system collapses into a two-particle state with definite spins: if
we measured $+\hbar/2$, the collapsed state is $|\!+\!z\rangle
\otimes|\!-\!z\rangle$, and if we measured $-\hbar/2$, the collapsed
state is $|\!-\!z\rangle\otimes|\!+\!z\rangle$.

According to quantum theory, the state collapse happens
instantaneously, regardless of the distance between the particles.  We
could prepare the two-particle state in a laboratory and hang on to
particle $A$, while sending particle $B$ to the Betelgeuse star
system, 642 light years away.  In principle, this can be done
carefully enough to avoid disturbing the two-particle quantum state.
Once ready, we measure $\hat{S}_z$ on particle $A$, which induces an
instantaneous collapse of the two-particle state.  Immediately
afterwards, if our alien colleague at Betelgeuse measures $\hat{S}_z$
on particle $B$, xhe will obtain (with 100\% certainty) the opposite
of our Earth result.  Yet in the time between these two measurements,
no classical signal could have traveled between Earth and Betelgeuse,
not even at the speed of light.

\textcolor{red}{[Fig.]}

There are three noteworthy aspects of this state-collapse phenomenon.

First, it dispels popular intuitive ``explanations'' of quantum state
collapse that involve experimental probes physically disturbing the
system being measured.  For instance, one story that's sometimes told
is that measuring a particle's position requires shining a light beam
on it, or disturbing it in some way; because of this disturbance, the
particle's momentum becomes uncertain.  The EPR paradox shows that
such stories don't capture the full weirdness of quantum state
collapse.  We can collapse the state of a particle at Betelgeuse by
doing a measurement on another particle, hundreds of light years away!

Second, the experimentalist has control over what kind of measurement
to perform.  So far, we have considered $\hat{S}_z$ measurements
performed on particle $A$.  But the Earth experimentalist can choose
to measure the spin of $A$ along any axis, say $\hat{S}_x$.  In the
basis of spin-up and spin-down states, the $\hat{S}_x$ operator has
the matrix representation
$$\hat{S}_x = \frac{\hbar}{2}\, \begin{pmatrix}0&1\\1&0\end{pmatrix}.$$
The eigenvalues and eigenvectors are
$$\begin{aligned}s_x = \;\;\frac{\hbar}{2},\; &\;\;\; |\!+\!x\rangle = \frac{1}{\sqrt{2}}\Big(|\!+\!z\rangle + |\!-\!z\rangle\Big) \\ s_x = -\frac{\hbar}{2}, &\;\;\; |\!-\!x\rangle = \frac{1}{\sqrt{2}}\Big(|\!+\!z\rangle - |\!-\!z\rangle\Big).\end{aligned}$$
Conversely, we can write the $\hat{S}_z$ eigenstates in the $\{|\!+\!x\rangle,|\!-\!x\rangle\}$ basis:
$$\begin{aligned}|\!+\!z\rangle &= \frac{1}{\sqrt{2}}\Big(|\!+\!x\rangle + |\!-\!x\rangle\Big) \\ |\!-\!z\rangle &= \frac{1}{\sqrt{2}}\Big(|\!+\!x\rangle - |\!-\!x\rangle\Big).\end{aligned}$$
This allows us to write the two-particle entangled state in the
$\hat{S}_x$ basis:
$$|\psi\rangle = \frac{1}{\sqrt{2}} \Big(|\!-\!x\rangle\otimes|\!+\!x\rangle \,-\, |\!+\!x\rangle\otimes|\!-\!x\rangle\Big).$$
The Earth measurement causes the particles to collapse into definite
spin states with opposite spins---but these are now spin states of
definite ${S}_x$, rather than definite ${S}_z$.

Third, it is important to note that the choice of measurement cannot
be used for superluminal communication.  The Earth experimentalist can
choose whether to (i) measure $\hat{S}_z$ or (ii) measure $\hat{S}_x$,
and this choice has an instantaneous effect on the quantum state of
particle $B$.  If the Betelgeuse experimentalist can find a way to
distinguish between case (i) and case (ii), even statistically, this
could serve as a method of instantaneous communication.  Yet this
turns out to be impossible!  This is because quantum states cannot be
observed directly, but only via measurements.  Suppose the Earth
measurement is $\hat{S}_z$, which collapses $B$ to either
$|\!+\!z\rangle$ or $|\!-\!z\rangle$ (each with probability
0.5).  The Betelgeuse experimentalist now has a choice of which
measurement to perform.  If xhe measures $\hat{S}_z$, the outcome is
$+\hbar/2$ or $-\hbar/2$ with equal probabilities.  Whereas for an
$\hat{S}_x$ measurement, the probabilities are:
$$\begin{aligned}P(S_x = +\hbar/2) &= \frac{1}{2}\, \Big|\langle\!+x|\!+\!z\rangle\Big|^2 \;+\;\, \frac{1}{2}\, \Big|\langle\!+x|\!-\!z\rangle\Big|^2 = \frac{1}{2}\\P(S_x = -\hbar/2) &= \frac{1}{2}\, \Big|\langle\!-x|\!+\!z\rangle\Big|^2 \;+\;\, \frac{1}{2}\, \Big|\langle\!-x|\!-\!z\rangle\Big|^2 = \frac{1}{2}.\end{aligned}$$
This analysis can be repeated for the case where the Earth measurement
is $\hat{S}_x$.  No matter what, we find that the outcomes of any spin
measurement at Betelgeuse always have 50/50 probability.  Thus, even
though the choice of Earth measurement instantaneously affects the
quantum state at Betelgeuse, there's no \textit{measurement} that can
be done at Betelgeuse to deduce that choice.

Although quantum state collapse does not allow for superluminal
communication, it is a ``\textbf{nonlocal}'' phenomenon, in the sense
that certain ingredients of the theory (quantum states) are changing
faster than the speed of light classically allows.  EPR argued that
quantum theory is therefore \textit{philosophically} inconsistent with
relativity.  They suggested, as an alternative, that quantum mechanics
is an approximation of some deeper theory, whose details are currently
not known to us, but which is both deterministic and local.

Such a ``\textbf{hidden variables theory}'' might give rise to the
appearance of quantum state collapse in the following way.  Suppose
each particle has a definite but ``hidden'' value of $S_z$, either
$S_z = +\hbar/2$ or $S_z = -\hbar/2$; for conciseness, we denote these
as $+$ or $-$.  We then hypothesize that the two-particle quantum state
$$|\psi\rangle = \frac{1}{\sqrt{2}} \Big(|\!+\!z\rangle\otimes|\!-\!z\rangle \,-\, |\!-\!z\rangle\otimes|\!+\!z\rangle\Big)$$
is not an actual description of reality.  Instead, it corresponds to a
\textit{statistical} distribution of ``hidden variable'' states, which
we denote by $[+;-]$ (i.e., $S_z = +\hbar/2$ for particle $A$ and $S_z
= -\hbar/2$ for particle $B$), and $[-;+]$ (the other way around).
When the Earth experimentalist measures $S_z$, we uncover the value of
the hidden variables: a $+z$ result implies $[+;-]$, and a $-z$ result
implies $[-;+]$.  When the Betelgeuse experimentalist subsequently
measures $S_z$, the result obtained is the opposite of the Earth
result---because that's what hidden variables were all along.  No
physical influence actually travels from Earth to Betelguese between
the two experiments, instantaneously or otherwise.

Of course, there is a huge amount of detail missing about the actual
contents of the ``hidden variables'' theory, especially considering
the fact that it has to be consistent with the many successful
predictions of quantum mechanics.  For example, how does it mimick the
time-dependent Schr\"odinger equation?  But we could imagine that if
we work hard enough, it might be possible to fill in the missing
details.

\section{Bell's theorem}

In 1964, John S.~Bell published a bombshell paper showing that the
predictions of quantum theory are \textit{not} consistent with hidden
variable theories.  The amazing thing about this theorem, which is
known as \textbf{Bell's theorem}, is that it requires no knowledge
about the details of the hidden variable theory, except that it is
deterministic and local (as discussed in the previous section).  In
this section, we study a simplified version of Bell's theorem due to
D.~Mermin (Mermin 1981).

We again consider spin-1/2 particle pairs, with particle $A$ on Earth
and particle $B$ in the Betelgeuse star system.  The experimentalists
at the two locations can measure the spin along three distinct choices
of spin axis, whose spin observables are denoted by $S_1$, $S_2$,
$S_3$.  We will not specify the actual directions of these spin axes
until later in the proof.  (Note, however, that they need not
correspond to orthogonal spatial directions.)

In a hidden variables theory, each particle must have a definite value
of each spin observable.  For example, particle $A$ might have spin
values $S_1 = +\hbar/2, \, S_2 = +\hbar/2, \, S_3 = -\hbar/2$.  For
conciseness, we denote this by $[++-]$.

We now imagine repeatedly preparing two-particle systems, sending the
particles to Earth and Betelgeuse, and letting the two
experimentalists measure the spin of each particle.  Within the
framework of quantum mechanics, the system is prepared each time in
the two-particle state
$$|\psi\rangle = \frac{1}{\sqrt{2}} \Big(|\!+\!z\rangle\otimes|\!-\!z\rangle \,-\, |\!-\!z\rangle\otimes|\!+\!z\rangle\Big).$$
Moreover, in each instance of the experiment, each experimentalist
independently and randomly chooses one of the three possible axes for
measuring spin.  (It doesn't matter which experimentalist performs the
measurement first.)  The procedure is repeated many times, and both
experimentalists' choices of spin axis are recorded, along with their
measurement results.

\textcolor{red}{[Fig.]}

At the end of the run of experiments, the records are examined.  The
results are found to be consistent with the predictions of quantum
theory.  In particular, whenever the experimentalists happen to
choose the same spin axis (e.g., both measuring $S_1$), they always
find opposite spins.

We are interested in the hypothesis that quantum theory is a
statistical description of an underlying hidden variables theory.  To
be consistent with the aforementioned predictions of quantum theory,
the hidden spin states of the two particles in this experiment must
have opposite values along each direction.  This means that there are
$8$ distinct possibilities, which we can denote as
$$\begin{aligned}{[}{+++};{---}], \;\;\; [{++-};{--+}], \;\;\; [{+-+};{-+-}], \;\;\; [{+--};{-++}],\\ [{-++};{+--}], \;\;\; [{-+-};{+-+}], \;\;\; [{--+};{++-}], \;\;\; [{---};{+++}].\end{aligned}$$
For instance, $[{++-};{--+}]$ indicates that for particle $A$, $S_1 =
S_2 = +\hbar/2$ and $S_3 = -\hbar/2$, while particle $B$ has the
opposite spin values, $S_1 = S_2 = -\hbar/2$ and $S_3 = +\hbar/2$.
However, we don't know anything about the relative probabilities of
these 8 possibilities.

Let's now focus on the subset of experiments in which the two
experimentalists happened to choose \textit{different} spin axes
(e.g., Earth chose $S_1$ and Betelgeuse chose $S_2$).  Within this
subset, what is the probability \textit{for the two measurement
  results to have opposite signs} (i.e., one $+$ and one $-$)?  To
answer this question, we first look at the following 6 cases:
$$\begin{aligned}{[}{++-};{--+}], \;\;\; [{+-+};{-+-}], \;\;\; [{+--};{-++}],\\ [{-++};{+--}], \;\;\; [{-+-};{+-+}], \;\;\; [{--+};{++-}].\end{aligned}$$
These are the cases which do not have all $+$ or all $-$ for each
particle.  Consider one of these, say ${[}{++-};{--+}]$.  The two
experimentalists picked their measurement axes at random each time,
and amongst the experiments where they picked different axes, there
are only two ways for the measurement results to have opposite signs:
$(S_1,S_2)$ or $(S_2,S_1)$.  There are four ways to get the same sign:
$(S_1,S_3)$, $(S_2,S_3)$, $(S_3,S_1)$ and $(S_3, S_2)$.  Thus,
\textit{for this particular set of spin values}, the probability for
measurement results with opposite signs is 1/3.  If we go through all
6 of the cases listed above, we find the same conclusion: the
probability for opposite signs is 1/3.

Now look at the remaining 2 cases:
$${[}{+++};{---}], \;\;\; [{---};{+++}].$$
For these, the experimentalists will find results that have opposite
signs with probability 1.  Putting these results together, we arrive
at the following statement:

\textit{Given that the two experimentalists choose different spin
  axes, the probability that their results have opposite signs is $P
  \ge 1/3$.}

This inequality statement is called \textbf{Bell's inequality}.  If
there is ever a situation where quantum theory predicts a probability
$P < 1/3$ (i.e., a violation of Bell's inequality), then we will be
forced to conclude that quantum theory is inherently inconsistent with
local deterministic hidden variables.  Note that we don't need to know
anything about the ``inner workings'' of the hidden variables theory;
in the above derivation of Bell's inequality, we did not assume
anything about the relative probabilities for the various spin values.

To complete the proof, we need to find a set of $S_1$, $S_2$, and
$S_3$ such that the predictions of quantum mechanics violate Bell's
inequality.  One simple choice is to align $S_1$ with the $\hat{z}$
axis, and align $S_2$ and $S_3$ along the $\hat{x}$-$\hat{z}$ plane at
$120^\circ$ ($2\pi/3$ radians) from $S_1$, as shown in the figure:

\textcolor{red}{[Fig.]}

Thus, the spin operators can be written in the $\hat{S}_z$ eigenbasis as
$$\begin{aligned}\hat{S}_1 &= \frac{\hbar}{2} \, \sigma_3 \\ \hat{S}_2 &= \frac{\hbar}{2} \, \left[\cos(2\pi/3) \sigma_3 + \sin(2\pi/3)\sigma_1\right]  \\   \hat{S}_3 &= \frac{\hbar}{2} \, \left[\cos(2\pi/3) \sigma_3 - \sin(2\pi/3)\sigma_1\right].\end{aligned}$$

Suppose the Earth experimentalist (measuring $A$) chooses $S_1$, and
obtains $+\hbar/2$.  The state of particle $A$ collapses to
$|\!+\!z\rangle$, and the state of particle $B$ collapses to
$|\!-\!z\rangle$.  The Betelgeuse experimentalist is assumed to
choose a different spin axis.  If the choice is $S_2$, the expectation
value of the measurement is
$$\begin{aligned}\langle\, - z \, | \, S_2 \,|-\!z\,\rangle &= \frac{\hbar}{2} \Big[\cos(2\pi/3) \langle\,- z\,|\sigma_3| - \!z\,\rangle + \sin(2\pi/3)\langle\,- z\,|\sigma_1|-\!z\,\rangle\Big]\\ &= \frac{\hbar}{2} \cdot \frac{1}{2} \end{aligned}$$
If $P_+$ and $P_-$ respectively denote the probability of measuring
$+\hbar/2$ and $-\hbar/2$ in this measurement, the above equation
implies that $P_+ - P_- = + 1/2$.  Moreover, $P_+ + P_- = 1$ by
probability conservation.  It then follows that the probability of
obtaining a negative value (the opposite sign from the Earth
measurement) is $P_- = 1/4$.  All the other possible scenarios are
worked out in a similar way.  The overall conclusion is that if the
two experimentalists choose different measurement axis, they obtain
results of opposite signs with probability $1/4$.  This is smaller
than $1/3$, and hence violates Bell's inequality.

Finally, we must consult Nature by performing an experimental test.
Is it possible to observe, in an actual experiment, probabilities that
violate Bell's inequality?  In the decades following Bell's 1964
paper, many experiments were performed to answer this question.  These
experiments are all substantially more complicated than the simple
two-particle spin-$1/2$ model that we've studied, and they are subject
to various uncertainties and ``loopholes'' that are beyond the scope
of our discussion.  Overall, the experimental consensus is a clear
\textit{yes}: Nature really does behave as quantum mechanics predict,
in a manner that cannot be replicated using deterministic local hidden
variables!  For a review of the experimental evidence, see Aspect
(1999).


\section{Entanglement entropy}

\section{Quantum cryptography}

\section{The Many-Worlds Interpretation}

\section{Exercises}

%% \begin{enumerate}
%% \end{enumerate}




\section{Further Reading}

\begin{itemize}
\item A.~Einstein, B.~Podolsky, and N.~Rosen,
  \textit{Can Quantum-Mechanical Description of Physical Reality Be
    Considered Complete?}, Physical Review \textbf{47}, 777 (1935).

\item J.~S.~Bell, \textit{On the Einstein-Podolsky-Rosen paradox},
  Physics \textbf{1}, 195 (1964).
  
\item N.~D.~Mermin, \textit{Bringing home the atomic world: Quantum
  mysteries for anybody}, American Journal of Physics \textbf{49}, 940
  (1981).

\item A.~Aspect, \textit{Bell's inequality test: more ideal than ever},
  Nature (News and Views) \textbf{398}, 189 (1999).
\end{itemize}


\end{document}


%% For decades after the discovery of quantum mechanics, the quantum
%% double-slit experiment was just a ``thought experiment'', meant to
%% illustrate the features of quantum mechanics that had been uncovered
%% by other, more complicated experiments.  Nowadays, the most convenient
%% way to do the experiment is with light, using single-photon sources
%% and single-photon detectors.  Quantum interference has also been
%% demonstrated experimentally using electrons, neutrons, and even
%% large-scale particles such as buckyballs.
