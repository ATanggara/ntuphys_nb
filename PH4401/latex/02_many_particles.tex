\documentclass[pra,11pt]{revtex4}
\usepackage{amsmath}
\usepackage{amssymb}
\usepackage{graphicx}
\usepackage{color}
\usepackage{mathrsfs}

\def\ket#1{\left|#1\right\rangle}
\def\bra#1{\left\langle#1\right|}
\def\braket#1{\left\langle#1\right\rangle}

\setlength{\parindent}{0pt}

\renewcommand{\baselinestretch}{1.0}
\setlength{\parskip}{0.07in}

\begin{document}

\section{Entangled quantum states}

So far, we have studied quantum mechanical systems consisting of
single particles.  The next important step is to look at systems
containing more than one particle.  As we shall see, the postulates of
quantum mechanics have interesting implications which only show up in
multi-particle systems, such as the phenomenon of quantum
entanglement.

In the following discussion, you may assume that we are dealing with
systems of ``distinguishable'' particles.  A simple example of such a
system is a two-particle system containing different types of
particles, like a proton and a neutron.  There is a separate set of
considerations to take into account whenever we're dealing with
systems of ``identical'' particles, such as a system composed of two
electrons, where the electrons cannot be distinguished from each other
even in principle.  We will spend the next chapter dealing with those
special complications.  If you're not sure what this paragraph is
talking about, don't worry; just read on.

We begin by considering a system of two particles, labeled $A$ and
$B$.  If each individual particle is treated as a quantum system, then
according to the usual postulates of quantum mechanics, its state is
described by a vector in some complex Hilbert space.  Let
$\mathscr{H}_A$ and $\mathscr{H}_B$ denote the respective Hilbert
spaces for these two particles.  When the two particles are considered
as a single quantum system, that system has its own Hilbert space,
denoted by
$$\mathscr{H} = \mathscr{H}_A\otimes \mathscr{H}_B.$$
The symbol $\otimes$ refers to a mathematical operation known as the
\textbf{tensor product}, which is a way of combining two Hilbert
spaces to form a larger Hilbert space.

The mathematical meaning of the tensor product is as follows.  Suppose
$\mathscr{H}_A$ can be spanned by an orthonormal basis
$\{|\mu_1\rangle, |\mu_2\rangle, |\mu_3\rangle, \dots\}$, and
$\mathscr{H}_B$ can be spanned by $\{|\nu_1\rangle, |\nu_2\rangle,
|\nu_3\rangle, \dots\}$.  Then the tensor product space $\mathscr{H}_A
\otimes \mathscr{H}_B$ can be defined in terms of the basis set
$$\Big\{\;\,|\mu_i\rangle\otimes|\nu_j\rangle \;\;  \Big| \;\; \textrm{all}\;|\mu_i\rangle,\; |\nu_j\rangle \;\,\Big\}.$$
Intuitively, this means that if you know particle $A$ is in state
$|\mu_i\rangle$, and particle $B$ is in state $|\nu_j\rangle$, that
specifies a state of the combined system.  A complete basis set for
the combined system can be constructed in this way.  Note that if
$\mathscr{H}_A$ has dimension $d_A$ and $\mathscr{H}_B$ has dimension
$d_B$, then $\mathscr{H}_A \otimes \mathscr{H}_B$ has dimension $d_A
d_B$.

Any two-particle state can then be written as
$$|\psi\rangle = \sum_{i} \sum_{j} \, c_{ij}\; |\mu_i\rangle \otimes |\nu_j\rangle.$$
The inner product for tensor product basis states is defined in terms
of the inner product of the individual spaces:
$$\Big(|\mu_i\rangle \otimes |\nu_j\rangle\;,\; |\mu_p\rangle \otimes |\nu_q\rangle \Big) \;\equiv\; \big(\langle\mu_i| \otimes \langle\nu_j| \big) \big(|\mu_p\rangle \otimes |\nu_q\rangle\big) \;\equiv\; \langle\mu_i|\mu_p\rangle \, \langle\nu_j|\nu_q\rangle.$$
In other words, if you need to multiply a tensor product bra with a
tensor product ket, the procedure is to (i) do the bra-ket product for
the first slot, (ii) do the bra-ket product for the second slot, and
(iii) multiply the results together.  It is straightforward to check
that this satisfies the usual requirements for an inner product in
linear algebra (assuming that the inner products for the
single-particle states are already well-defined).

As an example, suppose that both $\mathscr{H}_A$ and $\mathscr{H}_B$
are two-dimensional Hilbert spaces describing spin-$\frac{1}{2}$
degrees of freedom, such that each can be spanned by an orthonormal basis
$\{\,|\!\uparrow\rangle, \,|\!\downarrow\rangle \, \}$.  Then the tensor
product space $\mathscr{H}$ is spanned by
$$\Big\{\;|\!\uparrow\rangle\otimes|\!\uparrow\rangle\,,\; |\!\uparrow\rangle\otimes|\!\downarrow\rangle\,,\; |\!\downarrow\rangle\otimes|\!\uparrow\rangle\,,\; |\!\downarrow\rangle\otimes|\!\downarrow\rangle \;\Big\}.$$
The dimension of $\mathscr{H}$ is $2 \times 2 = 4$.  

At this point, we must make an extremely crucial observation.  As
mentioned, if we specify that particle $A$ is in state $|\mu_i\rangle$
and particle $B$ is in state $|\nu_j\rangle$, that specifies a state
of the combined system.  But the reverse is not true!  \textit{Some
  states of the combined system cannot be expressed in terms of
  definite states of the individual particles.}  For example, take the
previous example of two spaces describing spin-$\frac{1}{2}$ degrees
of freedom, and consider
$$|\psi\rangle = \frac{1}{\sqrt{2}} \Big(|\!\uparrow\rangle\otimes|\!\downarrow\rangle \,-\, |\!\downarrow\rangle\otimes|\!\uparrow\rangle\Big).$$

This two-particle state is constructed from two of the basis states of
the tensor product space,
$|\!\uparrow\rangle\otimes|\!\downarrow\rangle$ and
$|\!\downarrow\rangle\otimes|\!\uparrow\rangle$; you can check that
the factor of $1/\sqrt{2}$ correctly normalizes the state so that
$\langle\psi|\psi\rangle = 1$.  Evidently, neither particle A nor
particle B has a definite $|\!\uparrow\rangle$ or
$|\!\downarrow\rangle$ state.  Furthermore, we shall show rigorously,
in a later section, that there's \textit{no} choice of basis in which this
particular quantum state $|\psi\rangle$ can be expressed with
definite quantum states for each particle.  In other words,
$$|\psi\rangle \ne |\psi_A\rangle\otimes|\psi_B\rangle \;\;\;\textrm{for}\;\textrm{any}\;\; |\psi_A\rangle, \;|\psi_B\rangle.$$
In such a situation, the two particles are said to be
\textbf{entangled}.  This means that they cannot individually be
described by single-particle quantum states; their quantum
mechanical descriptions are intrinsically ``mixed up'' with one another.

For systems of more than two particles, the multi-particle quantum
states are defined using multiple tensor products.  Suppose we have
$N$ particles, with single-particle Hilbert spaces $\mathscr{H}_1,
\mathscr{H}_2, \dots, \mathscr{H}_N$, of dimensionality $d_1, \dots,
d_N$.  Then the multi-particle Hilbert space is $\mathscr{H} =
\mathscr{H}_1 \otimes \mathscr{H}_2 \otimes \cdots \otimes
\mathscr{H}_N$, which has dimensionality $d = d_1 d_2\cdots d_N$.  It
is interesting to note that the dimensionality of the multi-particle
Hilbert space scales \textit{exponentially} with the number of
particles.  For instance, if each single-particle Hilbert space is
two-dimensional, a $20$-particle system will have a Hilbert space of
dimensionality $2^{20} =1\,048\,576$.  This means that multi-particle
quantum systems are able to ``carry'' huge amounts of information,
compared to classical systems.  That is one of the reasons people are
excited by the prospect of quantum computing.

\section{Partial measurements}

Let us recall how measurements work in single-particle quantum
theory.  Suppose a particle is in a quantum state $|\psi\rangle$
and we perform some measurement $\hat{Q}$.  To figure out the
measurement outcome probabilities, we first write $|\psi\rangle$ in
the eigenstate basis of $\hat{Q}$:
$$|\psi\rangle = \sum_n |n\rangle\,\psi_n, \;\;\mathrm{where}\;\;\hat{Q}|n\rangle = q_n |n\rangle \;\,\textrm{and}\;\, \psi_n = \langle n|\psi\rangle.$$
For simplicity, let's suppose the eigenvalues are all non-degenerate.
Then the probability of obtaining the measurement outcome $q_n$ is
$|\psi_n|^2$, the absolute square of the coefficient of
$|n\rangle$ in the above formula.  After the measurement, the system
``collapses'' into state $|n\rangle$, which can be regarded as the
projection operation
$$|\psi\rangle \rightarrow \frac{1}{\sqrt{\mathcal{N}}}\; |n\rangle\langle n|\psi\rangle.$$
The prefactor $\mathcal{N} = |\psi_n|^2$ simply ensures that the new
state remains normalized; the overall phase has no physical
significance.

For multi-particle systems, the measurement postulates of quantum
theory are faced with a new complication: how do we describe a
measurement that is performed on just one particle?

The generalization turns out to be not so difficult.  Let's again
consider a system of two particles, with Hilbert spaces
$\mathscr{H}_A$ and $\mathscr{H}_B$; the two-particle Hilbert space is
$\mathscr{H} = \mathscr{H}_A \otimes \mathscr{H}_B$.  Suppose we
perform a measurement on particle $A$, described by the operator
$\hat{Q}$ which acts upon $\mathscr{H}_A$.  We can write any two-particle
state $|\psi\rangle$ using the eigenvectors of $\hat{Q}$ as a basis
set for $\mathscr{H}_A$, and an arbitrary basis set for
$\mathscr{H}_B$:
$$\begin{aligned}|\psi\rangle &= \sum_{n}\sum_{\mu} |n\rangle\otimes |\mu\rangle \;\psi_{n\mu} \\&= \sum_n |n\rangle\otimes |n'\rangle, \;\;\mathrm{where}\;\,|n'\rangle\equiv \sum_\mu |\mu\rangle \;\psi_{n\mu} \in \mathscr{H}_B.\end{aligned}$$
Note that unlike the single-particle case, the ``coefficient'' of
$|n\rangle$ in the basis expansion is not merely a complex number, but
a vector in the space $\mathscr{H}_B$.  We can proceed by
analogy.  The probability of obtaining the measurement outcome $q_n$
should be the ``absolute square'' of this:
$$\langle n'|n'\rangle = \sum_\mu |\psi_{n\mu}|^2.$$
After the measurement, the ``collapse'' of the state is described by
the projection operation
$$|\psi\rangle \rightarrow \frac{1}{\sqrt{\mathcal{N}}}\; \Big(|n\rangle\langle n|\Big) |\psi\rangle = \frac{1}{\sqrt{\mathcal{N}}}\; |n\rangle\otimes |n'\rangle.$$
Note that the projection operator $|n\rangle\langle n|$ only acts on
the $\mathscr{H}_A$ part of the two-particle space.  The factor
$\mathcal{N}$ is again defined so that the resulting state is
normalized to unity.

Let's work through an explicit example, involving the
previously-discussed system of two spin-$\frac{1}{2}$ particles, and
the two-particle entangled state
$$|\psi\rangle = \frac{1}{\sqrt{2}} \Big(|\!\uparrow\rangle\otimes|\!\downarrow\rangle \,-\, |\!\downarrow\rangle\otimes|\!\uparrow\rangle\Big).$$
Here, $|\!\uparrow\,\rangle$ and $|\!\downarrow\,\rangle$ denote eigenstates
of the operator $\hat{S}_z$, with the eigenvalues $+\hbar/2$ and $-\hbar/2$
respectively.  Let us perform an $S_z$ measurement on particle A.  The
results are as follows:
\begin{itemize}
\item A measurement outcome of $+\hbar/2$ has probability $P_+ = \langle
  n'|n'\rangle$, where $|n'\rangle =
  (1/\sqrt{2})\,|\!\downarrow\rangle$.  Hence, $P_+ = 1/2$.  After the
  measurement, the state collapses to $|\!\uparrow\rangle
  \otimes|\!\downarrow\rangle$.

\item A measurement outcome of $-\hbar/2$ has probability $P_- = \langle
  n'|n'\rangle$, where $|n'\rangle =
  (1/\sqrt{2})\,|\!\uparrow\rangle$.  Hence, $P_- = 1/2$.  After the
  measurement, the state collapses to $|\!\downarrow\rangle
  \otimes|\!\uparrow\rangle$.
\end{itemize}
In other words, we obtain the two possible results $\pm \hbar/2$ with
equal probability; in either case, the two-particle state collapses so
that particle $A$ is in the spin eigenstate corresponding to our
measurement result, while particle $B$ has the opposite spin.  After
collapse, the two-particle state is no longer entangled.

\section{The Einstein-Podolsky-Rosen ``paradox''}

In 1935, several years after the formulation of quantum mechanics,
Einstein, Podolsky, and Rosen (EPR) formulated a thought experiment
known as the \textbf{EPR paradox}, which highlighted the exceptionally
counter-intuitive features possessed by entangled quantum states.  The
three authors believed these features meant that quantum theory had to
be incomplete.  Nowadays, it has been shown that the EPR paradox is
not actually a paradox; real systems \textit{really do} exhibit the
strange behaviors described in the thought experiment.

The EPR paradox begins with an entangled state, such as the entangled
spin-$1/2$ systems described in the previous section:
$$|\psi\rangle = \frac{1}{\sqrt{2}} \Big(|\!\uparrow\rangle\otimes|\!\downarrow\rangle \,-\, |\!\downarrow\rangle\otimes|\!\uparrow\rangle\Big).$$
As previously discussed, we can measure $\hat{S}_z$ on particle $A$.
The system collapses into a two-particle state with definite spins: if
we measured $+\hbar/2$, the collapsed state is $|\!\uparrow\rangle
\otimes|\!\downarrow\rangle$, and if we measured $-\hbar/2$, the collapsed
state is $|\!\downarrow\rangle\otimes|\!\uparrow\rangle$.

According to quantum theory, the state collapse happens
instantaneously, regardless of the distance between the particles.  We
could prepare the two-particle state in a laboratory and hang on to
particle $A$, while sending particle $B$ to the Betelgeuse star
system, 642 light years away.  In principle, this can be done
carefully to avoid disturbing the two-particle quantum state.  Once
ready, we can measure $\hat{S}_z$ on particle $A$.  Then the
two-particle state \textit{instantaneously} collapses.  Right after
our Earth measurement, if our alien colleague at Betelgeuse measures
$\hat{S}_z$ on particle $B$, xhe will obtain (with 100\% certainty)
the opposite of our measurement result.  Yet between the Earth and
Betelgeuse measurements, there has been no time for any signal to
travel between the particles, not even at the speed of light.

There are three noteworthy aspects of this state-collapse phenomenon.

First, it dispels popular intuitive ``explanations'' of quantum state
collapse that involve experimental probes physically disturbing the
system being measured.  For instance, one story that's sometimes told
is that measuring a particle's position requires shining a light beam
on it, or disturbing it in some way; because of this disturbance, the
particle's momentum becomes uncertain.  The EPR paradox shows that
such stories don't capture the full weirdness of quantum state
collapse.  We can collapse a state at Betelgeuse by doing a
measurement hundreds of light years away!

Second, the experimentalist has control over what kind of measurement
to perform.  So far, we have considered $\hat{S}_z$ measurements
performed on particle $A$.  But the Earth experimentalist can choose
to measure the spin of $A$ along any axis, and this choice affects the
post-measurement particle states.  For instance, the experimentalist
might choose to measure $\hat{S}_x$.  In the basis of spin-up and
spin-down states, this has the matrix representation
$$\hat{S}_x = \frac{\hbar}{2}\, \begin{pmatrix}0&1\\1&0\end{pmatrix}.$$
The eigenvalues and eigenvectors are
$$\begin{aligned}s_x = \;\;\frac{\hbar}{2},\; &\;\;\; |\!\rightarrow\rangle = \frac{1}{\sqrt{2}}\Big(|\!\uparrow\rangle + |\!\downarrow\rangle\Big) \\ s_x = -\frac{\hbar}{2}, &\;\;\; |\!\leftarrow\rangle = \frac{1}{\sqrt{2}}\Big(|\!\uparrow\rangle - |\!\downarrow\rangle\Big).\end{aligned}$$
Conversely, we can write the $\hat{S}_z$ eigenstates in the $\{|\!\rightarrow\rangle,|\!\leftarrow\rangle\}$ basis:
$$\begin{aligned}|\!\uparrow\rangle &= \frac{1}{\sqrt{2}}\Big(|\!\rightarrow\rangle + |\!\leftarrow\rangle\Big) \\ |\!\downarrow\rangle &= \frac{1}{\sqrt{2}}\Big(|\!\rightarrow\rangle - |\!\leftarrow\rangle\Big).\end{aligned}$$
This allows us to write the two-particle entangled state in the
$\hat{S}_x$ basis:
$$|\psi\rangle = \frac{1}{\sqrt{2}} \Big(|\!\leftarrow\rangle\otimes|\!\rightarrow\rangle \,-\, |\!\rightarrow\rangle\otimes|\!\leftarrow\rangle\Big).$$
After the Earth measurement, the two particles collapse into definite
spin states with opposite spins.  However, these are spin states of
definite $\hat{S}_x$, rather than definite $\hat{S}_z$.

Third, it is important to note that the choice of measurement cannot
be used to send superluminal signals.  The experimentalist on Earth
can choose whether to (i) measure $\hat{S}_z$ or (ii) measure
$\hat{S}_x$, and this choice affects which states particle $B$, at
Betelgeuse, can collapse into.  If there is a way for the
experimentalist at Betelgeuse to distinguish between case (i) and case
(ii), even statistically, this could serve as a method of
instantaneous communication.

Yet it turns out that the experimentalist at Betelgeuse
\textit{cannot} distinguish between the two cases.  This is because
quantum states cannot be observed directly, but only via measurements.
Suppose the Earth measurement is $\hat{S}_z$, which collapses $B$ to
either $|\!\uparrow\rangle$ or $|\!\downarrow\rangle$ (each with
probability 0.5).  If the Betelgeuse experimentalist likewise measures
$\hat{S}_z$, the outcome will be $+\hbar/2$ or $-\hbar/2$ with equal
probabilities.  Whereas if the Betelgeuse experimentalist measures
$\hat{S}_x$, the probabilities are:
$$\begin{aligned}P(S_x = +\hbar/2) &= \frac{1}{2}\, \Big|\langle\!\rightarrow|\!\uparrow\rangle\Big|^2 \;+\;\, \frac{1}{2}\, \Big|\langle\!\rightarrow|\!\downarrow\rangle\Big|^2 = \frac{1}{2}\\P(S_x = -\hbar/2) &= \frac{1}{2}\, \Big|\langle\!\leftarrow|\!\uparrow\rangle\Big|^2 \;+\;\, \frac{1}{2}\, \Big|\langle\!\leftarrow|\!\downarrow\rangle\Big|^2 = \frac{1}{2}.\end{aligned}$$
This analysis can be repeated for the case where the Earth measurement
is $\hat{S}_x$.  No matter what, we find that the outcomes of either
measurement at Betelgeuse always have 50/50 probability.  Thus, even
though the choice of Earth measurement instantaneously affects the
\textit{quantum state} at Betelgeuse, there's no \textit{measurement}
that can be done at Betelgeuse to deduce that choice.  Hence, there is
no usable communication signal.

\section{Bell's theorem}

\section{Entanglement entropy}

\section{Quantum cryptography}

\section{The Many-Worlds Interpretation}

\section{Exercises}

%% \begin{enumerate}
%% \item Phase shift under scattering resonance

%% \item Classical driven oscillator analogy  
%% \end{enumerate}




\section{Further Reading}

%% \begin{itemize}
%% \item Bransden \& Joachain, \S4.4, 9.2--9.3, 13.4
%% \item Sakurai, \S5.6, 7.7--7.8

%% \end{itemize}


\end{document}


%% For decades after the discovery of quantum mechanics, the quantum
%% double-slit experiment was just a ``thought experiment'', meant to
%% illustrate the features of quantum mechanics that had been uncovered
%% by other, more complicated experiments.  Nowadays, the most convenient
%% way to do the experiment is with light, using single-photon sources
%% and single-photon detectors.  Quantum interference has also been
%% demonstrated experimentally using electrons, neutrons, and even
%% large-scale particles such as buckyballs.
