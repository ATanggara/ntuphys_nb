\documentclass[pra,12pt]{revtex4}
\usepackage{amsmath}
\usepackage{amssymb}
\usepackage{graphicx}
\usepackage{color}
\usepackage{mathrsfs}
\usepackage{enumerate}
\usepackage{epigraph}
\usepackage[pdfborder={0 0 0},colorlinks=true,linkcolor=blue,urlcolor=blue]{hyperref}

\def\ket#1{\left|#1\right\rangle}
\def\bra#1{\left\langle#1\right|}
\def\braket#1{\left\langle#1\right\rangle}

\setlength{\parindent}{0pt}

\renewcommand{\baselinestretch}{1.0}
\setlength{\parskip}{0.07in}
\setlength{\epigraphwidth}{.6\textwidth}

\begin{document}

\begin{center}
{\Large \textbf{Chapter 4: Quantum Electrodynamics}}
\end{center}

This chapter describes \textbf{quantum electrodynamics}, the quantum
theory of the electromagnetic field and how it interacts with charged
particles such as electrons.  We start by formulating Hamiltonians
that describe how a quantum mechanical electron is affected by
classical electric and magnetic fields.  Next, we describe the
quantization of Maxwell's equations, which yields a quantum field
theory in which the elementary excitations are photons---particles of
light.  Finally, we can formulate a theory in which both electrons and
photons are treated on the same footing, as excitations of underlying
quantum fields.  Along the way, we will see how relativity is
accomodated into the framework of quantum mechanics.

Quantum electrodynamics is a rich and intricate theory, and we will
not be able to cover many important topics, such as diagrammatic
methods for performing field theoretical calculations.  If you are
interested in this topic, a good introductory textbook is Zee's
\textit{Quantum Field Theory in a Nutshell} (\hyperref[cite:zee]{Zee
  2010}).

\section{Particles in a classical electromagnetic field}

\subsection{Non-relativistic spinless particles}

Consider a non-relativistic charged particle in an electromagnetic
field.  As we are mainly interested in the physics of electrons
interacting with electromagnetic fields, we will henceforth take the
electric charge of the particle to be $-e$, where $e =
1.602\times10^{-19}\,\mathrm{C}$ is the elementary charge.  (If you
wish to describe particles with an arbitrary electric charge $q$,
simply perform the substitution $e \rightarrow -q$ in all formulas in
the rest of this chapter.)  For simplicity, we also assume (for now)
that the particle has charge and mass but is otherwise
``featureless''; in particular, unlike a real electron, it has no spin
angular momentum, and no magnetic dipole moment.

Suppose we treat the electromagnetic field as a classical field, in
the sense that the electric and magnetic field strengths are definite
quantities, not subject to quantum dynamics.  Our goal is to formulate
the Hamiltonian governing the quantum dynamics of this
non-relativistic charged particle.  To do this, let us first derive
the particle's \textit{classical} equations of motion.

In the classical regime, the action of an electromagnetic field on a
point charged particle is decribed by the Lorentz force law,
\begin{equation}
  \mathbf{F}(\mathrm{r},t) = -e\Big(\mathbf{E}(\mathrm{r},t)
  + \dot{\mathbf{r}}\times \mathbf{B}(\mathrm{r},t)\Big),
\end{equation}
where $\mathbf{r}$ and $\dot{\mathbf{r}}$ respectively denote the
position and velocity of the particle, $t$ is the time, and
$\mathbf{E}$ and $\mathbf{B}$ are the electric and magnetic fields.
If there are no other forces acting on the particle, then according to
Newton's second law, the equation of motion is
\begin{equation}
  m\ddot{\mathbf{r}} = -e\Big(\mathbf{E}(\mathrm{r},t)
  + \dot{\mathbf{r}} \times \mathbf{B}(\mathrm{r},t)\Big),
  \label{eom}
\end{equation}
where $m$ is the particle's mass.  To quantize this, we must convert
this equation of motion into the form of Hamilton's equations of
motion.

First, we introduce the electromagnetic scalar and vector potentials
$\phi(\mathrm{r},t)$ and $\mathbf{A}(\mathrm{r},t)$, where
\begin{align}
  \mathbf{E}(\mathbf{r},t) &= - \nabla \phi(\mathbf{r},t) - \frac{\partial\mathbf{A}}{\partial t} \\
  \mathbf{B}(\mathbf{r},t) &= \nabla \times \mathbf{A}(\mathbf{r},t).
  \label{Bfield}
\end{align}
We now postulate that the above equation of motion can be described by
the Lagrangian
\begin{equation}
  L(\mathbf{r},\dot{\mathbf{r}},t) = \frac{1}{2}m\dot{\mathbf{r}}^2
  + e \Big[\phi(\mathbf{r},t) - \dot{\mathbf{r}} \cdot \mathbf{A}(\mathbf{r},t)
    \Big].
\end{equation}
This is very similar to the usual prescription for the Lagrangian as
the kinetic energy minus the potential energy, with $-e\phi$ serving
as the potential energy function.  However, there is an extra
$-e\dot{\mathbf{r}} \cdot \mathbf{A}$ term; this will turn out to be
responsible for the magnetic force.  Let us plug the Lagrangian into
the Euler-Lagrange equations:
\begin{equation}
  \frac{\partial L}{\partial r_i} = \frac{d}{dt}
  \frac{\partial L}{\partial \dot{r}_i}.
\end{equation}
The partial derivatives of the Lagrangian are:
\begin{align}
  \begin{aligned}
    \frac{\partial L}{\partial r_i} &=
    e\Big[\partial_i \phi - \dot{r}_j \,\partial_i A_j \Big]\\
    \frac{\partial L}{\partial \dot{r}_i} &= m\dot{r}_i - e A_i.
  \end{aligned}
\end{align}
Next, we want to take the \textit{total} time derivative of $\partial
L /\partial \dot{r}_i$.  In doing so, we note that the $\mathbf{A}$
field has its own $t$-dependence, as well as varying with the
particle's $t$-dependent position:
\begin{align}
  \begin{aligned}
    \frac{d}{dt} \frac{\partial L}{\partial \dot{r}_i}
    &= m\ddot{r}_i - e\, \frac{d}{dt} A_i(\mathbf{r}(t),t) \\
    &= m\ddot{r}_i - e\, \partial_t A_i
    - e\, \dot{r}_j \partial_j A_i.
  \end{aligned}
\end{align}
(In the above equations, $\partial_i \equiv \partial/\partial r_i$,
where $r_i$ is the $i$-th component of the position vector, while
$\partial_t \equiv \partial/\partial t$.)  Putting these results into
the Euler-Lagrange equations yields
\begin{align}
  \begin{aligned}
    m\ddot{r}_i &=
    -e\Big[\Big(-\partial_i \phi - \partial_t A_i\Big)
      + \dot{r}_j \Big( \partial_i A_j - \partial_j A_i\Big) \Big] \\
    &= -e \Big[E_i(\mathbf{r},t) + \big(\dot{\mathbf{r}} \times
      \mathbf{B}(\mathbf{r},t) \big)_i\; \Big].
  \end{aligned}
\end{align}
(The last step can be derived by expressing the cross product using
the Levi-Cevita symbol, and using the identity $\varepsilon_{ijk}
\varepsilon_{lmk} = \delta_{il} \delta_{jm} - \delta_{im}
\delta_{jl}$.)  And indeed, this is the equation of motion
corresponding to the Lorentz force.

We can now derive the classical Hamiltonian.  First, we define the
canonical momentum:
\begin{equation}
  p_i = \frac{\partial L}{\partial \dot{r}_i} = m\dot{r}_i - e A_i.
\end{equation}
Now the Hamiltonian can be defined as $H(\mathbf{r},\mathbf{p}) =
\mathbf{p} \cdot \dot{\mathbf{r}} - L$, which needs to be expressed
using the $\mathbf{p}$ variables rather than $\dot{\mathbf{r}}$
variables:
\begin{align}
  \begin{aligned}
    H &= \mathbf{p}\cdot \left(\frac{\mathbf{p}+e\mathbf{A}}{m}\right)
    - \left(\frac{|\mathbf{p}+e\mathbf{A}|^2}{2m}
    + e\phi - \frac{e}{m}(\mathbf{p}+e\mathbf{A})\cdot \mathbf{A}\right) \\
    &= \frac{|\mathbf{p}+e\mathbf{A}|^2}{m}
    - \frac{e}{m}\mathbf{A}\cdot \left(\mathbf{p}+e\mathbf{A}\right)
    - \left(\frac{|\mathbf{p}+e\mathbf{A}|^2}{2m}
    + e\phi - \frac{e}{m}(\mathbf{p}+e\mathbf{A})\cdot \mathbf{A}\right)
  \end{aligned}
\end{align}
After cancelling various terms, we arrive at the result
$$\boxed{\qquad H = \frac{|\mathbf{p}+e\mathbf{A}(\mathbf{r},t)|^2}{2m} - e\phi(\mathbf{r},t).\qquad}$$

This looks much like the familiar Hamiltonian for a non-relativistic
particle that we have dealt with many times,
\begin{equation}
  H = \frac{|\mathbf{p}|^2}{2m} + V(\mathbf{r},t).
\end{equation}
The scalar potential $\phi(\mathbf{r},t)$ enters into the potential
energy term, as might be expected.  What may be more surprising is
that the vector potential appears via the substitution
\begin{equation}
  \mathbf{p} \rightarrow \mathbf{p} + e\mathbf{A}(\mathbf{r},t).  
\end{equation}
What does this mean?

To answer this, think about what ``momentum'' means in the context of
a charged particle in an electromagnetic field.  The meaning of
``momentum'' is rooted in Noether's theorem, which states that every
symmetry of a system (whether classical or quantum) is associated with
a conserved quantity.  Momentum is the quantity conserved when the
system is symmetric under spatial translations.  We can see this from
the Hamilton equation
\begin{equation*}
  \frac{dp_i}{dt} = \frac{\partial H}{\partial r_i},
\end{equation*}
which implies that if a Hamiltonian is $\mathbf{r}$-independent, then
$d\mathbf{p}/dt = 0$.  But when the electromagnetic potentials are
$\mathbf{r}$-independent, the quantity $m\dot{\mathbf{r}}$ (which we
usually call momentum) is \textit{not} necessarily conserved!
Consider the potentials
\begin{equation}
  \phi(\mathbf{r}, t) = 0, \;\;\; \mathbf{A}(\mathbf{r}, t) = Ct \hat{z},
\end{equation}
where $C$ is some constant.  These potentials are evidently
$\mathbf{r}$-independent, but the vector potential is time-dependent,
so the $-\dot{\mathbf{A}}$ term in Eq.~\eqref{Bfield} gives a
non-vanishing electric field:
\begin{equation}
  \mathbf{E}(\mathbf{r},t) = - C\hat{z}, \;\;\;\mathbf{B}(\mathbf{r},t) = 0.
\end{equation}
The Lorentz force law then says that
\begin{equation}
  \frac{d}{dt}(m\dot{\mathbf{r}}) = eC\hat{z},
\end{equation}
and thus $m\dot{\mathbf{r}}$ is not conserved.  On the other hand, the
quantity $\mathbf{p} = m\dot{\mathbf{r}} - e \mathbf{A}$ \textit{is}
conserved:
\begin{equation}
  \frac{d}{dt}(m\dot{\mathbf{r}} - e\mathbf{A}) =
  eC\hat{z} - eC\hat{z} = 0.
\end{equation}

To go from classical to quantum mechanics, we merely need to replace
$\mathbf{r}$ with the position operator $\hat{\mathbf{r}}$, and
$\mathbf{p}$ with the momentum operator $\hat{\mathbf{p}}$.  Hence,
the quantum Hamiltonian is
\begin{equation}
  \hat{H}(t) = \frac{|\hat{\mathbf{p}}+e\mathbf{A}(\hat{\mathbf{r}},t)|^2}{2m}
  - e\phi(\hat{\mathbf{r}},t).
  \label{quantumH}
\end{equation}
In the wavefunction representation, the momentum operator is
$\hat{\mathbf{p}} = -i\hbar\nabla$, as usual.

\subsection{Gauge symmetry}

The Hamiltonian \eqref{quantumH} possesses a subtle property known as
\textbf{gauge symmetry}.  Suppose we modify the scalar and vector
potentials via the substitutions
\begin{align}
  \begin{aligned}
    \phi(\mathbf{r},t) &\rightarrow \phi(\mathbf{r},t) + \dot{\lambda}(\mathbf{r},t)\\
    \mathbf{A}(\mathbf{r},t) &\rightarrow
    \mathbf{A}(\mathbf{r},t) - \nabla{\lambda}(\mathbf{r},t),
  \end{aligned}
\end{align}
where $\lambda(\mathbf{r},t)$ is an arbitrary scalar field called a
\textbf{gauge field}.  We know, from classical electrodynamics, that
such a transformation leaves the electric and mangetic fields
unchanged.  Applied to the Hamiltonian \eqref{quantumH}, the gauge
transformation generates a new Hamiltonian,
\begin{equation}
  \hat{H}_\lambda(t)
  = \frac{|\hat{\mathbf{p}}+e\mathbf{A}(\hat{\mathbf{r}},t) + e\nabla\lambda(\hat{\mathbf{r}},t)|^2}{2m}
  - e\phi(\hat{\mathbf{r}},t) - e\dot{\lambda}(\hat{\mathbf{r}},t).
\end{equation}
If $\psi(\mathbf{r},t)$ is any wavefunction obeying the Schr\"odinger
equation for the original Hamiltonian,
\begin{equation}
  i\hbar\frac{\partial\psi}{\partial t} =
  \hat{H}(t) \psi(\mathbf{r},t)
  = \left[\frac{|\hat{\mathbf{p}}+e\mathbf{A}(\hat{\mathbf{r}},t)|^2}{2m}
  - e\phi(\hat{\mathbf{r}},t) \right]\psi(\mathbf{r},t),
\end{equation}
then it can be shown that the wavefunction $\psi(\mathbf{r},t)
\exp(ie\lambda/\hbar)$ automatically satisfies the Schr\"odinger
equation for the transformed Hamiltonian:
\begin{equation}
  i\hbar\frac{\partial}{\partial t} \left[\psi(\mathbf{r},t) \, \exp\left(\frac{ie\lambda(\mathbf{r},t)}{\hbar}\right)\right] =
  \hat{H}_\lambda(t) \left[\psi(\mathbf{r},t) \, \exp\left(\frac{ie\lambda(\mathbf{r},t)}{\hbar}\right)\right].
  \label{gaugeschrod}
\end{equation}

To prove this, observe how time and space derivatives act on the new
wavefunction:
\begin{align}
  \begin{aligned}
    \frac{\partial}{\partial t} \left[\psi \, \exp\left(\frac{ie\lambda}{\hbar}\right)\right] &=
    \left[\frac{\partial\psi}{\partial t} \;+\; \frac{ie}{\hbar} \dot{\lambda} \psi
      \,\, \right] \exp\left(\frac{ie\lambda}{\hbar}\right)\\
    \nabla \left[\psi \, \exp\left(\frac{ie\lambda}{\hbar}\right)\right] &=
    \left[\nabla \psi + \frac{ie}{\hbar} \nabla \lambda \psi \right] \exp\left(\frac{ie\lambda}{\hbar}\right).
  \end{aligned}
\end{align}
When the extra terms generated by the $\exp(ie\lambda/\hbar)$ factor
are slotted into the Schr\"odinger equation, they cancel the gauge
terms in the scalar and vector potentials.  For example:
\begin{align}
  \begin{aligned}
    \Big(-i\hbar\nabla + e\mathbf{A} - e\nabla\lambda\Big)
    \left[\psi \, \exp\left(\frac{ie\lambda}{\hbar}\right)\right] &=
    \Big[\left(-i\hbar\nabla + e\mathbf{A}\right)\psi\Big]\;
    \exp\left(\frac{ie\lambda}{\hbar}\right) \\
    \Big|-i\hbar\nabla + e\mathbf{A} - e\nabla\lambda\;\Big|^2
    \left[\psi \, \exp\left(\frac{ie\lambda}{\hbar}\right)\right] &=
    \Big[\left|-i\hbar\nabla + e\mathbf{A}\right|^2\psi\Big]\;
    \exp\left(\frac{ie\lambda}{\hbar}\right).
  \end{aligned}
\end{align}
The remainder of the proof for Eq.~\eqref{gaugeschrod} is then
straightforward, and is left to the reader.

For static electromagnetic fields, the above result is simpler to
state.  Suppose
\begin{equation}
  \hat{H} = \frac{|\hat{\mathbf{p}}+e\mathbf{A}(\hat{\mathbf{r}})|^2}{2m}
  - e\phi(\hat{\mathbf{r}})
\end{equation}
is a time-independent electromagnetic Hamiltonian, with eigenenergies
$\{E_m \}$ and the corresponding energy eigenfunctions
$\{\psi_m(\mathbf{r})\}$.  Then the Hamiltonian
\begin{equation}
  \hat{H}_\lambda = \frac{|\hat{\mathbf{p}}+e\mathbf{A}(\hat{\mathbf{r}}) + e\nabla\lambda(\mathbf{r})|^2}{2m}
  - e\phi(\hat{\mathbf{r}})
\end{equation}
has the same energy spectrum $\{E_m\}$, with eigenfunctions
$\{\,\psi_m(\mathbf{r}) \exp[ie\lambda(\mathbf{r})/\hbar]\,\}$.

One particularly interesting class of gauge transformations involves
gauge fields of the form
\begin{equation}
  \lambda(\mathbf{r}) = \alpha\, \phi,
\end{equation}
where $\phi$ is the azimuthal coordinate in the cylindrical coordinate
system $(r,\phi,z)$.



\subsection{Non-relativistic electrons: the Pauli Hamiltonian}


\subsection{Relativistic electrons: the Dirac Hamiltonian}


\section{Quantizing the electromagnetic field}

\subsection{The electromagnetic field Hamiltonian}

\subsection{Relativity and quantum mechanics}

\section{The electron-photon interaction}

\subsection{The electromagnetic shift of energy levels}

Bethe's Lamb shift calculation.

\subsection{Photon emission and absorption}

Dirac's calculation of the coefficient of spontaneous emission.

\section{Looking ahead}

Need to treat both electrons and photons on the same footing, with QFT
language.  Renormalization.

\section*{Exercises}

\begin{enumerate}
\item Gauge transformations.
\end{enumerate}

\section*{Further Reading}

\begin{enumerate}[[1{]}]
\item A.~Zee, \textit{Quantum Field Theory in a Nutshell} (Princeton
  University Press, 2010).
\label{cite:zee}
\end{enumerate}

\end{document}


%% For decades after the discovery of quantum mechanics, the quantum
%% double-slit experiment was just a ``thought experiment'', meant to
%% illustrate the features of quantum mechanics that had been uncovered
%% by other, more complicated experiments.  Nowadays, the most convenient
%% way to do the experiment is with light, using single-photon sources
%% and single-photon detectors.  Quantum interference has also been
%% demonstrated experimentally using electrons, neutrons, and even
%% large-scale particles such as buckyballs.
