\documentclass[pra,11pt]{revtex4}
\usepackage{amsmath}
\usepackage{amssymb}
\usepackage{graphicx}
\usepackage{color}
\usepackage{mathrsfs}
\usepackage[pdfborder={0 0 0},colorlinks=true,linkcolor=blue]{hyperref}

\def\ket#1{\left|#1\right\rangle}
\def\bra#1{\left\langle#1\right|}
\def\braket#1{\left\langle#1\right\rangle}

\setlength{\parindent}{0pt}

\renewcommand{\baselinestretch}{1.0}
\setlength{\parskip}{0.07in}

\begin{document}

\section{Particle indistinguishability and exchange symmetry}

We saw, in the previous chapter, how the principles of quantum
mechanics are applied to systems of multiple particles.  However, that
discussion omitted an important feature of multi-particle systems,
namely the fact that particles of the same type are fundamentally
indistinguishable from each other.  It turns out that this
indistinguishability imposes strong constraints on the form of the
multi-particle quantum states, the study of which will lead us to a
fundamental re-interpretation of what ``particles'' are.

Suppose we have two particles of the same type, such as two electrons.
It is a fact of Nature that all electrons have identical physical
properties (the same mass, charge, spin, etc.).  The single-particle
Hilbert spaces of the two electrons must therefore be mathematically
identical.  Let us denote this space by $\mathscr{H}^{(1)}$.  The
Hilbert space for a two-electron system is a tensor product of two
single-electron Hilbert spaces, denoted by
$$\mathscr{H}^{(2)} = \mathscr{H}^{(1)} \otimes \mathscr{H}^{(1)}.$$
Because the electrons have identical physical properties, any
Hamiltonian for the two-electron system must act on the electrons in a
symmetrical way.  An example of such a Hamiltonian is
$$\hat{H} = \frac{1}{2m_e} \Big(|\hat{\mathbf{p}}_1|^2 + |\hat{\mathbf{p}}_2|^2\Big) + \frac{e^2}{4\pi\varepsilon_0|\hat{\mathbf{r}}_1 - \hat{\mathbf{r}}_2|},$$
which contains non-relativistic kinetic energies for the two
electrons, as well as the potential energy for the Coulomb interaction
between the electrons.  The operators $\hat{\mathbf{p}}_1$ and
$\hat{\mathbf{r}}_1$ act on electron 1, while $\hat{\mathbf{p}}_2$ and
$\hat{\mathbf{r}}_2$ act on electron 2.

It can be seen that this two-electron Hamiltonian is invariant under
an interchange of the operators acting on the two electrons (i.e.,
$\hat{\mathbf{p}}_1 \leftrightarrow \hat{\mathbf{p}}_2$ and
$\hat{\mathbf{r}}_1 \leftrightarrow \hat{\mathbf{r}}_2$).  This can be
regarded as a symmetry of the system, known as \textbf{exchange
  symmetry}.  In other contexts, we have seen how symmetries of
quantum systems can be represented by unitary (i.e., norm-conserving)
operators that commute with the Hamiltonian.  Exchange symmetry can
likewise be represented by an operator $\hat{P}$, defined as follows:
let $\{|\mu_i\rangle\}$ be a basis set for the single-electron Hilbert
space $\mathscr{H}^{(1)}$; then
$$P \Big (\sum_{ij} \psi_{ij} |\mu_i\rangle\otimes|\mu_j\rangle \Big)
\;\equiv\;  \sum_{ij} \psi_{ij} |\mu_j\rangle\otimes|\mu_i\rangle = \sum_{ij} \psi_{ji} |\mu_i\rangle\otimes|\mu_j\rangle.$$
Since two consecutive exchanges obviously leave the system unchanged,
the $\hat{P}$ operator satisfies
$$\hat{P}^2 = \hat{I},$$
where $\hat{I}$ is the identity operator.  It can be proven that
$\hat{P}$ is linear, unitary (as required for a symmetry operator), and
Hermitian; and moreover that the exchange operation performed by
$\hat{P}$ does not depend on the choice of basis (see
\hyperref[ex:1]{Exercise 1}).

It can also be shown that $\hat{P}$ commutes with the above
Hamiltonian $\hat{H}$, or indeed any Hamiltonian where the
single-particle operators appear in a symmetrical manner (see
\hyperref[ex:2]{Exercise 2}).

According to Noether's theorem, any symmetry implies a conservation
law.  In the case of exchange symmetry, since $\hat{P}$ is itself
Hermitian, we can take the conserved quantity to be its eigenvalue.
Given that $\hat{P}^2 = \hat{I}$, there are only two possibilities:
$$\hat{P} |\psi\rangle = p|\psi\rangle \;\;\;\Rightarrow\;\;\; p = \begin{cases}+1 & \textrm{(``symmetric\;state'')} \\ -1 & \textrm{(``antisymmetric\;state'').}\end{cases}$$
We call the eigenvalue $p$ the \textbf{exchange parity}.  If the
system is initially in either eigenstate of $\hat{P}$, then under time
evolution via Schr\"odinger's equation, it remains an eigenstate for
all other times.  (This follows from the fact that $\hat{P}$ commutes
with $\hat{H}$.)

The concept of exchange parity generalizes to systems of more than two
particles.  In a system of $N$ identical particles, we can define a
set of exchange operators $\hat{P}_{ij}$, where
$i,j\in\{1,2,\dots,N\}$ and $i\ne j$; this operator exchanges particle
$i$ and particle $j$.  The Hamiltonian must commute with \textit{all}
the exchange operators---in other words, their eigenvalues ($\pm 1$)
are all separately conserved.

We now come across the following important and profound facts:
\begin{enumerate}
\item A multi-particle state consisting of identical particles must be
  an eigenstate of every exchange operator $\hat{P}_{ij}$.

\item For every $\hat{P}_{ij}$, the eigenvalue (exchange parity)
  $p_{ij}$ has the same value (i.e., all $+1$ or all $-1$).

\item The exchange parity depends \textit{uniquely} on the type of
  particle.
\end{enumerate}
These facts are \textit{not} a logical result of our discussion thus
far!  Instead, they are to be regarded as empirical facts of Nature.
But later, in Section~\ref{sec:qft}, we will discuss why it may be
``natural'' for these facts to hold.

Particles that have symmetric states ($p_{ij} = +1$) are called
\textbf{bosons}.  The elementary particles that ``carry'' the
fundamental forces are all bosons: these are photons (the elementary
particles of light, which ``carry'' the electromagnetic force), gluons
(which carry the ``strong nuclear force'' that holds protons and
neutrons together), and $W$ and $Z$ bosons (which carry the ``weak
nuclear force'' that is responsible for beta decay).  Certain
composite particles, such as helium-4 nuclei (alpha particles), are
also bosons.

Particles that have antisymmetric states ($p_{ij} = -1$) are called
\textbf{fermions}.  The elementary particles of ``matter'' are all
fermions: these are electrons, muons, tauons, the various types of
quarks, and the various types of neutrinos, along with their
anti-particles (positrons, etc.).  Protons and neutrons are also
fermions, although these are not elementary particles, being each
composed of three quarks.  Certain other composite particles, such as
helium-3 nuclei, are also fermions.

We will discuss later, in Section~\ref{sec:spinstats}, what determines
if a certain particle is a fermion or a boson.  But first, let us
undertake an examination of the peculiar features possessed by
symmetric (``bosonic'') multi-particle states, and antisymmetric
(``fermionic'') multi-particle states.

\section{Boson states}

Consider system of $N$ bosons.  As described in the previous section,
any state of this system must be symmetric under every possible
exchange of two bosons:
$$\hat{P}_{ij}\; |\psi\rangle = |\psi\rangle \;\;\; \forall\, i\;\textrm{and}\;j,\;\; i\ne j.$$


\section{Fermion states}

\section{Second quantization}

\section{Measurements and state collapse}

\section{Quantum field theory}
\label{sec:qft}

\section{The spin-statistics connection}
\label{sec:spinstats}

\section*{Exercises}

\begin{enumerate}
\item Consider a system of two identical particles.  Each
  single-particle Hilbert space $\mathscr{H}^{(1)}$ is spanned by a
  basis $\{|\mu_i\}$.  The exchange operator is defined on
  $\mathscr{H}^{(2)} = \mathscr{H}^{(1)} \otimes \mathscr{H}^{(1)}$ by
$$P \Big (\sum_{ij} \psi_{ij} |\mu_i\rangle\otimes|\mu_j\rangle \Big)
  \;\equiv\;  \sum_{ij} \psi_{ij} |\mu_j\rangle\otimes|\mu_i\rangle.$$
  Prove that $\hat{P}$ is linear, unitary, and Hermitian.  Moreover,
  prove that the operation is basis-independent: i.e., given any other
  basis $\{\nu_j\}$ that spans $\mathscr{H}^{(1)}$, it is likewise
  true that
$$P \Big (\sum_{ij} \varphi_{ij} |\nu_i\rangle\otimes|\nu_j\rangle \Big)
  \;=\;  \sum_{ij} \varphi_{ij} |\nu_j\rangle\otimes|\nu_i\rangle.$$
  \label{ex:1}

\item
  Prove that the exchange operator commutes with the Hamiltonian
$$\hat{H} = - \frac{\hbar^2}{2m_e} \Big(\nabla_1^2 + \nabla^2_2\Big) + \frac{e^2}{4\pi\varepsilon_0|\mathbf{r}_1 - \mathbf{r}_2|}.$$ \label{ex:2}
  
\end{enumerate}

\section*{Further Reading}

%% \begin{itemize}
%% \item
%% \end{itemize}

\end{document}


%% For decades after the discovery of quantum mechanics, the quantum
%% double-slit experiment was just a ``thought experiment'', meant to
%% illustrate the features of quantum mechanics that had been uncovered
%% by other, more complicated experiments.  Nowadays, the most convenient
%% way to do the experiment is with light, using single-photon sources
%% and single-photon detectors.  Quantum interference has also been
%% demonstrated experimentally using electrons, neutrons, and even
%% large-scale particles such as buckyballs.
