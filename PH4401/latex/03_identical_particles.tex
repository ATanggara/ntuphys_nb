\documentclass[pra,12pt]{revtex4}
\usepackage{amsmath}
\usepackage{amssymb}
\usepackage{graphicx}
\usepackage{color}
\usepackage{mathrsfs}
\usepackage[pdfborder={0 0 0},colorlinks=true,linkcolor=blue]{hyperref}

\def\ket#1{\left|#1\right\rangle}
\def\bra#1{\left\langle#1\right|}
\def\braket#1{\left\langle#1\right\rangle}

\setlength{\parindent}{0pt}

\renewcommand{\baselinestretch}{1.0}
\setlength{\parskip}{0.07in}

\begin{document}

\begin{center}
{\Large \textbf{Chapter 3: Identical Particles}}
\end{center}

\section{Particle exchange symmetry}

We saw, in the previous chapter, how the principles of quantum
mechanics apply to systems of multiple particles.  That discussion
omitted an important feature of multi-particle systems, namely the
fact that particles of the same type are fundamentally
indistinguishable from each other.  As it turns out,
indistinguishability is a strong constraint on the form of the
multi-particle quantum states.  Looking into this will ultimately lead
us toward a fundamental re-interpretation of what ``particles'' are.

Suppose we have two particles of the same type, e.g.~two electrons.
It is a fact of Nature that all electrons have identical physical
properties: the same mass, same charge, same total spin, etc.  As a
consequence, the single-particle Hilbert spaces of the two electrons
must be mathematically identical.  Let us denote this space by
$\mathscr{H}^{(1)}$.  For a two-electron system, the Hilbert space is
a tensor product of two single-electron Hilbert spaces, denoted by
$$\mathscr{H}^{(2)} = \mathscr{H}^{(1)} \otimes \mathscr{H}^{(1)}.$$
Moreover, any Hamiltonian acting on such a system must affect the two
electrons in a symmetrical way.  An example of such a Hamiltonian is
$$\hat{H} = \frac{1}{2m_e} \Big(|\hat{\mathbf{p}}_1|^2 + |\hat{\mathbf{p}}_2|^2\Big) + \frac{e^2}{4\pi\varepsilon_0|\hat{\mathbf{r}}_1 - \hat{\mathbf{r}}_2|},$$
which contains non-relativistic kinetic energies for the two
electrons, as well as the potential energy for the Coulomb interaction
between the electrons.  The operators $\hat{\mathbf{p}}_1$ and
$\hat{\mathbf{r}}_1$ act on electron 1, while $\hat{\mathbf{p}}_2$ and
$\hat{\mathbf{r}}_2$ act on electron 2.  Evidently, this sort of
two-electron Hamiltonian is invariant under an interchange of the
operators acting on the two electrons (i.e., $\hat{\mathbf{p}}_1
\leftrightarrow \hat{\mathbf{p}}_2$ and $\hat{\mathbf{r}}_1
\leftrightarrow \hat{\mathbf{r}}_2$).  This can be regarded as a
kind of symmetry, called \textbf{exchange symmetry}.

As we know, symmetries of quantum mechanical systems can be
represented by unitary operators that commute with the Hamiltonian.
Exchange symmetry can likewise be represented by an operator
$\hat{P}$, defined as follows: let $\{|\mu_i\rangle\}$ be a basis set
for the single-electron Hilbert space $\mathscr{H}^{(1)}$; then
$\hat{P}$ interchanges the basis states for the two electrons:
$$\begin{aligned}P \Big (\sum_{ij} \psi_{ij} |\mu_i\rangle|\mu_j\rangle \Big)
\;&\equiv\;  \sum_{ij} \psi_{ij} |\mu_j\rangle|\mu_i\rangle \\&= \sum_{ij} \psi_{ji} |\mu_i\rangle|\mu_j\rangle \;\;\;\textrm{(interchanging $i\leftrightarrow j$ in the double sum)}\end{aligned}$$
This ``exchange operator'' has the following properties:
\begin{enumerate}
\item $\hat{P}^2 = \hat{I},$ where $\hat{I}$ is the identity operator.

\item $\hat{P}$ is linear, unitary, and Hermitian (see
  \hyperref[ex:1]{Exercise 1}).
  
\item The effect of $\hat{P}$ does not depend on the choice
  of basis (see \hyperref[ex:1]{Exercise 1}).

\item $\hat{P}$ commutes with the above Hamiltonian $\hat{H}$, or
  indeed any Hamiltonian where the single-particle operators appear in
  a symmetrical manner (see \hyperref[ex:2]{Exercise 2}).
\end{enumerate}

According to Noether's theorem, any symmetry implies a conservation
law.  In the case of exchange symmetry, $\hat{P}$ is both Hermitian
\textit{and} unitary, so we can take the conserved quantity to be the
eigenvalue of $\hat{P}$ itself.  We call this eigenvalue, $p$, the
\textbf{exchange parity}.  Given that $\hat{P}^2 = \hat{I}$, there are
just two possibilities:
$$\hat{P} |\psi\rangle = p|\psi\rangle \;\;\;\Rightarrow\;\;\; p = \begin{cases}+1 & \textrm{(``symmetric\;state'')}, \;\;\textrm{or} \\ -1 & \textrm{(``antisymmetric\;state'').}\end{cases}$$
Since $\hat{P}$ commutes with $\hat{H}$, if the system starts out in
one eigenstate of $\hat{P}$ with parity $p$, it always retains the
same definite parity $p$ under time evolution via Schr\"odinger's
equation.

The concept of exchange parity generalizes to systems of more than two
particles.  In a system of $N$ particles, we can define a set of
exchange operators $\hat{P}_{ij}$, where $i,j\in\{1,2,\dots,N\}$ and
$i\ne j$, such that $\hat{P}_{ij}$ exchanges particle $i$ and particle
$j$.  If the particles are identical, the Hamiltonian must commute
with \textit{all} the exchange operators---i.e., the parities ($\pm
1$) are all separately conserved.

We now invoke the following postulates:
\begin{enumerate}
\item A multi-particle state of identical particles is an eigenstate
  of every $\hat{P}_{ij}$.

\item For each $\hat{P}_{ij}$, the exchange parity $p_{ij}$ has the
  same value: i.e., all $+1$ or all $-1$.

\item The exchange parity $p_{ij}$ depends \textit{uniquely} on the
  type of particle involved.
\end{enumerate}
The above statements are \textit{not} logical results of our
discussion thus far!  Rather, they are empirical facts of Nature.  For
now, we must assume that they are true, and explore the consequences.

Particles that have symmetric states ($p_{ij} = +1$) are called
\textbf{bosons}.  The elementary particles that ``carry'' the
fundamental forces are all bosons: these are the photons (elementary
particles of light, which carry the electromagnetic force), gluons
(which carry the strong nuclear force that binds protons and neutrons
together), and $W$ and $Z$ bosons (which carry the weak nuclear force
that is responsible for beta decay).  Other bosons include particles
that carry non-fundamental forces, such as phonons (particles of
sound), as well as certain composite particles such as alpha particles
(helium-4 nuclei).

Particles that have antisymmetric states ($p_{ij} = -1$) are called
\textbf{fermions}.  The elementary particles of ``matter'' are all
fermions: electrons, muons, tauons, quarks, neutrinos, and their
anti-particles (positrons, anti-neutrinos, etc.).  Certain composite
particles are also fermions, the most notable being protons and
neutrons, each of which is composed of three quarks.

We will discuss later, in Section~\ref{sec:spinstats}, what determines
if a particle is a fermion or a boson.

\section{Bosons}

A state of $N$ bosons must be symmetric under every possible exchange
operator:
$$\hat{P}_{ij}\; |\psi\rangle = |\psi\rangle \;\;\; \forall\, i, j \in\{1,\dots,N\},\;\; i\ne j.$$

There is a standard way to construct multi-particle states obeying
this symmetry condition, which is based on the ``occupancy'' of the
single-particle states.  To illustrate the procedure, consider a
two-boson system ($N = 2$).  Suppose both bosons occupy the same
single-particle state, $|\alpha\rangle \in \mathscr{H}^{(1)}$; then
the two-boson state is simply
$$|\alpha,\alpha\rangle = |\alpha\rangle  |\alpha\rangle.$$
It is apparent that this satisfies $\hat{P}_{12}
|\alpha,\alpha\rangle = |\alpha,\alpha\rangle$, as required.

Now suppose the two bosons occupy \textit{different} single-particle
states, $|\alpha\rangle$ and $|\beta\rangle$.  It would be wrong to
write the two-boson state as $|\alpha\rangle |\beta\rangle$, because
the particles would not be symmetric under exchange.  Instead, we
construct the multi-particle state
$$|\alpha,\beta\rangle = \frac{1}{\sqrt{2}} \Big( |\alpha\rangle  |\beta\rangle + |\beta\rangle  |\alpha\rangle\Big).$$
This has the appropriate exchange symmetry:
$$\hat{P}_{12}\,|\alpha,\beta\rangle \;=\; \frac{1}{\sqrt{2}} \Big( |\beta\rangle  |\alpha\rangle + |\alpha\rangle  |\beta\rangle\Big) \;=\; |\alpha,\beta\rangle.$$

This construction is generalizable to arbitrary $N$.  Denote the $N$
occupied single-particle states by
$$|\phi_1\rangle, \, |\phi_2\rangle, \, |\phi_3\rangle, \, \dots, |\phi_N\rangle.$$
Here, we label the single-particle states with numerical subscripts,
but these states might not be unique; for example, we could have
$|\phi_1\rangle = |\phi_2\rangle = |\alpha\rangle$, meaning there is a
single-particle state $|\alpha\rangle$ occupied by two particles.  Now, the
$N$-boson state can be written as
$$|\phi_1,\phi_2,\dots,\phi_N\rangle = \mathcal{N} \sum_p \Big(|\phi_{p(1)}\rangle  |\phi_{p(2)}\rangle  |\phi_{p(3)}\rangle  \cdots  |\phi_{p(N)}\rangle\Big).$$
The sum is taken over each of the $N!$ permutations acting on
$\{1,2,\dots,N\}$.  For each permutation $p$, we let $p(j)$ denote the
integer that $j$ is permuted into.  The prefactor $\mathcal{N}$ is a
normalization constant, and it can be shown that its appropriate value
is
$$\mathcal{N} = \sqrt{\frac{n_a!n_b!\cdots}{N!}},$$
where $n_\mu$ denotes the number of particles in state $\mu$, and $N =
n_\alpha + n_\beta + \cdots$ is the total number of particles.  The proof of
this is left as an exercise (\hyperref[ex:boson_norm]{Exercise 3}).

To see that the above $N$-particle state has the right exchange
symmetry, let us apply an arbitrary exchange operator $\hat{P}_{ij}$:
$$\begin{aligned}\hat{P}_{ij}|\phi_1,\phi_2,\dots,\phi_N\rangle &= \mathcal{N} \sum_p \hat{P}_{ij} \Big(\cdots  |\phi_{p(i)}\rangle  \cdots  |\phi_{p(j)}\rangle\cdots\Big) \\&= \mathcal{N} \sum_p \Big(\cdots  |\phi_{p(j)}\rangle  \cdots  |\phi_{p(i)}\rangle\cdots\Big).\end{aligned}$$
In each term of the sum, the states $i$ and $j$ are interchanged.  But
since the sum runs through all permutations of the states, the result
is the same with or without the exchange, so we still end up with
$|\phi_1,\phi_2,\dots,\phi_N\rangle$.  The multi-particle state is
therefore symmetric under every exchange operation.

As an example, consider a three-boson system with two particles in
state $|\alpha\rangle$, and one particle in state $|\beta\rangle$.  To express
the three-particle state, define $\{|\phi_1\rangle, |\phi_2\rangle,
|\phi_3\rangle\}$ such that $|\phi_1\rangle = |\phi_2\rangle =
|\alpha\rangle$ and $|\phi_3\rangle = |\beta\rangle$.  Then
$$\begin{aligned}|\phi_1,\phi_2,\phi_3\rangle &= \mathcal{N} \Big( \;\;
|\phi_1\rangle|\phi_2\rangle|\phi_3\rangle +
|\phi_2\rangle|\phi_3\rangle|\phi_1\rangle +
|\phi_3\rangle|\phi_1\rangle|\phi_2\rangle \\ &\qquad\;\, +
|\phi_1\rangle|\phi_3\rangle|\phi_2\rangle +
|\phi_3\rangle|\phi_2\rangle|\phi_1\rangle +
|\phi_2\rangle|\phi_1\rangle|\phi_3\rangle\Big) \\
&= 2\mathcal{N} \Big(
|\alpha\rangle|\alpha\rangle|\beta\rangle +
|\alpha\rangle|\beta\rangle|\alpha\rangle +
|\beta\rangle|\alpha\rangle|\alpha\rangle\Big).
\end{aligned}$$
The three distinct exchange symmetry operators have the following effects:
$$\begin{aligned}\hat{P}_{12}|\phi_1,\phi_2,\phi_3\rangle &= 2\mathcal{N} \Big(
|\alpha\rangle|\alpha\rangle|\beta\rangle +
|\beta\rangle|\alpha\rangle|\alpha\rangle +
|\alpha\rangle|\beta\rangle|\alpha\rangle\Big) = |\phi_1,\phi_2,\phi_3\rangle \\
\hat{P}_{23}|\phi_1,\phi_2,\phi_3\rangle &= 2\mathcal{N} \Big(
|\alpha\rangle|\beta\rangle|\alpha\rangle +
|\alpha\rangle|\alpha\rangle|\beta\rangle +
|\beta\rangle|\alpha\rangle|\alpha\rangle\Big) = |\phi_1,\phi_2,\phi_3\rangle\\
\hat{P}_{13}|\phi_1,\phi_2,\phi_3\rangle &= 2\mathcal{N} \Big(
|\beta\rangle|\alpha\rangle|\alpha\rangle +
|\alpha\rangle|\beta\rangle|\alpha\rangle +
|\alpha\rangle|\alpha\rangle|\beta\rangle\Big) = |\phi_1,\phi_2,\phi_3\rangle.
\end{aligned}$$

\section{Fermions}

A state of $N$ fermions must be antisymmetric under every possible
exchange operator:
$$\hat{P}_{ij}\; |\psi\rangle = -|\psi\rangle \;\;\; \forall\, i,j\in\{1,\dots,N\}, \; i\ne j.$$
Similar to the bosonic case, we can explicitly construct multi-fermion
states based on the occupancy of single-particle states.  Let us
consider the $N=2$ case first.  If the fermions occupy states
$|\alpha\rangle$ and $|\beta\rangle$, the appropriate two-particle
state is
$$|a,b\rangle = \frac{1}{\sqrt{2}} \Big(|\alpha\rangle|\beta\rangle - |\beta\rangle|\alpha\rangle\Big).$$
We can easily check that
$$\hat{P}_{12} |a,b\rangle = \frac{1}{\sqrt{2}} \Big(|\beta\rangle|\alpha\rangle - |\alpha\rangle|\beta\rangle\Big) = - |\alpha,\beta\rangle.$$
This construction breaks down if we try letting $|\alpha\rangle$ and
$|\beta\rangle$ be the same state: in that case, the two terms would
cancel each other out to give the zero vector, which is not a valid
quantum state (as states must be described by vectors of unit norm).
This leads to the \textbf{Pauli exclusion principle}, which says that
fermions cannot occupy the same state.  Another way of stating Pauli's
principle is that any single-particle state is either unoccupied, or
occupied by exactly one fermion.

For general $N$, let the occupied single-particle states be
$\{|\phi_1\rangle, |\phi_2\rangle,\dots,|\phi_N\rangle\}$.  Then the
appropriate $N$-fermion state is
$$|\phi_1,\dots,\phi_N\rangle = \frac{1}{\sqrt{N!}} \sum_p s(p)\, |\phi_{p(1)}\rangle |\phi_{p(2)}\rangle \cdots |\phi_{p(N)}\rangle.$$
The $1/\sqrt{N!}$ prefactor is a normalization constant (which you can
verify for yourself).  The expression involves a sum over every
permutation $p$ of the sequence $\{1,2,\dots,N\}$.  However, it is
important to note that each term in the sum has a coefficient $s(p)$,
denoting the \textbf{parity} of the permutation.  The parity of a
permutation $p$ is defined as $+1$ if $p$ involves an even number of
exchanges starting from the sequence $\{1,2,\dots,N\}$, and $-1$ if
$p$ involves an odd number of exchanges.

A couple of examples will make these definitions clearer.  For $N=2$,
the sequence $\{1,2\}$ has $2! = 2$ permutations:
$$\begin{aligned}p_1 : \{1,2\} &\rightarrow \{1,2\}, \;\;\;s(p_1) = +1 \\ p_2 : \{1,2\} &\rightarrow \{2,1\}, \;\;\;s(p_2) = -1.\end{aligned}$$
For $N=3$, the sequence $\{1,2,3\}$ has $3!=6$ permutations:
$$\begin{aligned}
  p_1 : \{1,2,3\} &\rightarrow \{1,2,3\}, \;\;\;s(p_1) = +1 \\
  p_2 : \{1,2,3\} &\rightarrow \{2,1,3\}, \;\;\;s(p_2) = -1 \\
  p_3 : \{1,2,3\} &\rightarrow \{2,3,1\}, \;\;\;s(p_3) = +1 \\
  p_4 : \{1,2,3\} &\rightarrow \{3,2,1\}, \;\;\;s(p_4) = -1 \\
  p_5 : \{1,2,3\} &\rightarrow \{3,1,2\}, \;\;\;s(p_5) = +1 \\
  p_6 : \{1,2,3\} &\rightarrow \{1,3,2\}, \;\;\;s(p_6) = -1.\end{aligned}$$
The permutations can be generated by exchanging pairs of sequence
elements; each time we perform such an exchange, the sign of $s(p)$
is reversed.

We can now see why the above $N$-particle state describes fermions.
Let us apply $\hat{P}_{ij}$ to it:
$$\begin{aligned}\hat{P}_{ij}|\phi_1,\dots,\phi_N\rangle &= \frac{1}{\sqrt{N!}} \sum_p s(p)\, \hat{P}_{ij} \big[\cdots |\phi_{p(i)}\rangle \cdots |\phi_{p(j)}\rangle \cdots\big] \\&= \frac{1}{\sqrt{N!}} \sum_p s(p)\, \big[\cdots |\phi_{p(j)}\rangle \cdots |\phi_{p(i)}\rangle \cdots\big].\end{aligned}$$
Consider each term in the above sum.  Here, the single-particle state
for $p(i)$ and $p(j)$ have exchanged places.  The resulting term must
be a match for another one of the terms in
$|\phi_1,\dots,\phi_N\rangle$, because the sum runs over all possible
permutations---\textit{except} for one thing: the coefficient $s(p)$
must have an \textit{opposite} sign, since the two permutations are
related by an exchange.  It follows that
$\hat{P}_{ij}|\phi_1,\dots,\phi_N\rangle = -
|\phi_1,\dots,\phi_N\rangle$, as desired for fermions.

If the occupied single-particle states are non-unique, this
construction breaks down, since the terms in the sum over $p$ cancel
out pairwise.  This is the manifestation of the Pauli exclusion
principle for $N$ fermions.

\section{Second quantization}

In the usual tensor product notation, it quite cumbersome to deal with
symmetric and antisymmetric states.  There is a more convenient
formalism, called \textbf{second quantization}, which we will now
introduce.  (The reason it is called ``second quantization'' will be
explained later; it is a bad name, but one we are stuck with for
historical reasons.)

We begin with the observation that a convenient way to specify states
of multiple identical particles is to give the occupation numbers of
the single-particle states.  This is called the \textbf{occupation
  number representation}.  To do this, we first enumerate the
single-particle states $\{|\alpha\rangle, |\beta\rangle, |\gamma\rangle, |\delta\rangle,
\cdots\}$, which form an orthonormal basis set for the single-particle
Hilbert space $\mathscr{H}^{(1)}$.  For a multi-particle state, we can
specify how many particles are in state $|\alpha\rangle$, how many are in
state $|\beta\rangle$, etc.  The corresponding multi-particle state is
then \textit{defined} as the appropriate symmetric (for bosons) or
antisymmetric (for fermions) tensor product state.

For instance, $|0,2,0,0,\dots\rangle$ specifies a two-particle state
in which both particles in state $|\beta\rangle$.  Assuming that the
particles are bosons (fermions cannot share the same state), then
$$|0,2,0,0,\dots\rangle \equiv |\beta\rangle|\beta\rangle.$$

To take another example, $|1,1,1,0,0,\dots\rangle$ specifies a
three-particle state where there is one particle each occupying
$|\alpha\rangle$, $|\beta\rangle$, and $|\gamma\rangle$.  If the particles are
bosons,
$$|1,1,1,0,0,\dots\rangle \equiv \frac{1}{\sqrt{6}}\Big(|\alpha\rangle|\beta\rangle|\gamma\rangle + |\gamma\rangle|\alpha\rangle|\beta\rangle + |\beta\rangle|\gamma\rangle|\alpha\rangle + |\alpha\rangle|\gamma\rangle|\beta\rangle + |\beta\rangle|\alpha\rangle|\gamma\rangle + |\gamma\rangle|\beta\rangle|\alpha\rangle \Big).$$
And if the particles are fermions,
$$|1,1,1,0,0,\dots\rangle \equiv \frac{1}{\sqrt{6}} \Big(|\alpha\rangle|\beta\rangle|\gamma\rangle + |\gamma\rangle|\alpha\rangle|\beta\rangle + |\beta\rangle|\gamma\rangle|\alpha\rangle - |\alpha\rangle|\gamma\rangle|\beta\rangle - |\beta\rangle|\alpha\rangle|\gamma\rangle - |\gamma\rangle|\beta\rangle|\alpha\rangle\Big).$$

At this point, we have to make a technical detour regarding the
Hilbert spaces that these state vectors reside in.  The state
$|0,2,0,0,\dots\rangle$ is a bosonic two-particle state, which is a
vector in the two-particle Hilbert space $\mathscr{H}^{(2)} =
\mathscr{H}^{(1)}\otimes \mathscr{H}^{(1)}$.  However,
$\mathscr{H}^{(2)}$ also contains two-particle states that are not
symmetric under exchange, which is not allowed for bosons.  Thus, it
would be more rigorous for us to narrow the Hilbert space to the space
of state vectors that are symmetric under exchange.  We denote this
reduced space by $\mathscr{H}^{(2)}_S$.

Likewise, $|1,1,1,0,\dots\rangle$ is a three-particle state lying in
$\mathscr{H}^{(3)}$.  If the particles are bosons, we can narrow the
space to $\mathscr{H}^{(3)}_S$; whereas if the particles are fermions,
we can narrow it to the space of three-particle states that are
antisymmetric under exchange, denoted by $\mathscr{H}^{(3)}_A$.  Thus,
$|1,1,1,0,\dots\rangle \in \mathscr{H}^{(3)}_{S/A}$, where the
subscript $S/A$ depends on whether we are dealing with symmetric
states ($S$) or antisymmetric states ($A$).


In order to make the occupation number representation more convenient
to work with, let us define an extended Hilbert space that is the
space of all possible bosonic/fermionic states, \textit{for every
  possible particle number}.  This is called the \textbf{Fock space}.
In the formal language of linear algebra, it can be written as
$$\mathscr{H}_{S/A}^F = \mathscr{H}^{(0)} \oplus \mathscr{H}^{(1)} \oplus \mathscr{H}^{(2)}_{S/A} \oplus \mathscr{H}^{(3)}_{S/A} \oplus \mathscr{H}^{(4)}_{S/A} \oplus \cdots$$
Here, $\oplus$ represents the \textbf{direct sum} operation, which
combines vector spaces by directly grouping their basis vectors into a
larger basis set; if $\mathscr{H}_1$ has dimension $d_1$ and
$\mathscr{H}_2$ has dimension $d_2$, then
$\mathscr{H}_1\oplus\mathscr{H}_2$ has dimension $d_1+d_2$.  (By
contrast, the space $\mathscr{H}_1\otimes\mathscr{H}_2$, defined via
the tensor product, has dimension $d_1d_2$.)  Once again, the
subscript $S/A$ depends on whether we are dealing with bosons ($S$) or
fermions ($A$).

The upshot is that any kind of boson or fermion state that we can
write down in the occupation number representation,
$|n_\alpha,n_\beta,n_\gamma,\dots\rangle$, is guaranteed to lie in
$\mathscr{H}^{F}_{S/A}$.  Moreover, these states form a complete basis
set for $\mathscr{H}^{F}_{S/A}$.

You may have noticed that the definition of the Fock space included
$\mathscr{H}^{(0)}$, the ``Hilbert space of 0 particles''.  This
Hilbert space contains only one distinct state vector, denoted by
$$|\varnothing\rangle \equiv |0,0,0,0,\dots\rangle.$$
This is called the \textbf{vacuum state}, and it represents a state in
which there are literally no particles.  Note that it has the usual
normalization for state vectors,
$\langle\varnothing|\varnothing\rangle = 1$; it is therefore
\textit{not} the same thing as a vector of magnitude zero.

The vacuum state might seem pointless, but we shall soon see that it
plays a very important role in the formalism for manipulating
multi-particle states.

\subsection{Second quantization for bosons}

After this lengthy prelude, we are ready to introduce the formalism of
second quantization.  Let us first concentrate on the bosonic case.  

We define an operator called the \textbf{particle creation operator},
which is denoted by $\hat{a}_\mu^\dagger$ and acts in the following way:
$$\hat{a}_\mu^\dagger \big|n_\alpha, n_\beta, \dots, n_\mu, \dots\big\rangle = \sqrt{n_\mu+1} \; \big|n_\alpha, n_\beta, \dots, n_\mu + 1, \dots\big\rangle. $$
In this definition, there is one particle creation operator for each
state in the single-particle basis
$\{|\alpha\rangle,|\beta\rangle,\dots\}$.  Each creation operator is
defined as an operator acting on state vectors in the Fock space
$\mathscr{H}^F_S$, and has the effect of incrementing the occupation
number of its single-particle state by one.  The prefactor of
$\sqrt{n_\mu+1}$ is defined for later convenience.

In particular, applying a creation operator to the vacuum state yields
a single-particle state:
$$\begin{aligned}\hat{a}_\mu^\dagger |\varnothing\rangle \, &= \, |0,\dots,0, 1, 0, 0, \dots\rangle. \\[-1ex] &\qquad\qquad\quad\;\;\rotatebox[origin=c]{90}{$\Rsh$}\,\mu
\end{aligned}$$

The Hermitian conjugate of the creation operator, denoted by $\hat{a}_\mu$,
is called the \textbf{particle annihilation operator}.  To observe its
effects, let us first take the Hermitian conjugate of the equation
which defines $\hat{a}_\mu^\dagger$:
$$\big\langle n_\alpha, n_\beta, \dots\big| \hat{a}_\mu = \sqrt{n_\mu+1} \; \big\langle n_\alpha, n_\beta, \dots, n_\mu + 1, \dots\big|. $$
Right-multiplying by another occupation number state
$|n_\alpha',n_\beta',\dots\rangle$ results in
$$\begin{aligned}\big\langle n_\alpha, n_\beta, \dots \big| \hat{a}_\mu \big|n_\alpha',n_\beta',\dots\big\rangle &= \sqrt{n_\mu+1} \; \big\langle \dots, n_\mu + 1, \dots\big| \dots, n_\mu',\dots\big\rangle \\&= \sqrt{n_\mu+1}\; \delta^{n_\alpha}_{n_\alpha'}\; \delta^{n_\beta}_{n_\beta'} \cdots \delta^{n_\mu+1}_{n_\mu'} \dots \\  &= \sqrt{n_\mu'}\; \delta^{n_\alpha}_{n_\alpha'}\; \delta^{n_\beta}_{n_\beta'} \cdots \delta^{n_\mu+1}_{n_\mu'}\end{aligned}$$
From this, we can deduce that
$$\hat{a}_\mu \big|n_\alpha, n_\beta, \dots, n_\mu, \dots\big\rangle = \begin{cases} \sqrt{n_\mu} \; \big|n_\alpha, n_\beta, \dots, n_\mu - 1, \dots\big\rangle, & \mathrm{if}\; n_\mu > 0 \\ 0, & \mathrm{if}\; n_\mu = 0.\end{cases} $$
In other words, the annihilation operator reduces the occupation
number of one of the single-particle states by one (hence its name).
As a special exception, if the given single-particle state is
unoccupied ($n_\mu = 0$), then applying $\hat{a}_\mu$ results in a zero
vector (note that this is \textit{not} the same thing as the vacuum
state $|\varnothing\rangle$).

The creation and annihilation operators obey the following extremely
important commutation relations:
$$\begin{aligned}\,[\hat{a}_\mu,\hat{a}_\nu] = [\hat{a}_\mu^\dagger,\hat{a}_\nu^\dagger] &= 0, \\ \,[\hat{a}_\mu,\hat{a}_\nu^\dagger] &= \delta_{\mu\nu}\end{aligned}\qquad\textrm{for all}\;\mu,\nu.$$
These can be derived by taking the matrix elements with respect to the
occupation number basis.  We will go through the derivation of the
last commutation relation; the others are left as an exercise
(\hyperref[ex:boson_commutators]{Exercise 4}).

To prove that $[\hat{a}_\mu,\hat{a}_\nu^\dagger] = \delta_{\mu\nu}$, let us first
consider the case where the creation and annihilation operators act on
the same single-particle state:
$$\begin{aligned}\big\langle n_\alpha, n_\beta, \dots \big| \hat{a}_\mu \hat{a}_\mu^\dagger \big| n_\alpha', n_\beta'\dots\big\rangle &= \sqrt{(n_\mu+1)(n_\mu'+1)}\; \big\langle \dots, n_\mu+1, \dots \big| \dots, n_\mu'+1, \dots\big\rangle \\ &= \sqrt{(n_\mu+1)(n_\mu'+1)} \delta^{n_\alpha}_{n_\alpha'} \; \delta^{n_\beta}_{n_\beta'} \; \cdots \delta^{n_\mu+1}_{n_\mu'+1}\cdots \\ &= (n_\mu+1) \delta^{n_\alpha}_{n_\alpha'} \; \delta^{n_\beta}_{n_\beta'} \; \cdots \delta^{n_\mu}_{n_\mu'}\cdots \\ \big\langle n_\alpha, n_\beta, \dots \big| \hat{a}_\mu^\dagger \hat{a}_\mu \big| n_\alpha', n_\beta'\dots\big\rangle &= \sqrt{n_\mu n_\mu'}\; \big\langle \dots, n_\mu-1, \dots \big| \dots, n_\mu'-1, \dots\big\rangle \\&= \sqrt{n_\mu n_\mu'} \delta^{n_\alpha}_{n_\alpha'} \; \delta^{n_\beta}_{n_\beta'} \; \cdots \delta^{n_\mu-1}_{n_\mu'-1}\cdots \\ &= n_\mu \delta^{n_\alpha}_{n_\alpha'} \; \delta^{n_\beta}_{n_\beta'} \; \cdots \delta^{n_\mu}_{n_\mu'}\cdots \end{aligned}$$
In the second equation, we were a bit sloppy in handling the $n_\mu =
0$ and $n_\mu' = 0$ cases, but you can check for yourself that the
result on the last line turns out to be correct.  Upon taking the
difference of the two equations, we get
$$\big\langle n_\alpha, n_\beta, \dots \big| \left(\hat{a}_\mu \hat{a}_\mu^\dagger - \hat{a}_\mu^\dagger \hat{a}_\mu\right) \big| n_\alpha', n_\beta'\dots\big\rangle = \delta^{n_\alpha}_{n_\alpha'} \; \delta^{n_\beta}_{n_\beta'} \; \cdots \delta^{n_\mu}_{n_\mu'}\cdots = \big\langle n_\alpha, n_\beta, \dots \big| n_\alpha', n_\beta'\dots\big\rangle.$$
Since the occupation number states form a basis set for
$\mathscr{H}^F_S$, we conclude that
$$\hat{a}_\mu \hat{a}_\mu^\dagger - \hat{a}_\mu^\dagger \hat{a}_\mu = \hat{I}.$$
Next, consider the case where $\mu \ne \nu$:
$$\begin{aligned}\big\langle n_\alpha, \dots \big| \hat{a}_\mu \hat{a}_\nu^\dagger \big| n_\alpha', \dots\big\rangle &= \sqrt{(n_\mu+1)(n_\nu'+1)}\, \langle \dots, n_\mu+1, \dots, n_\nu, \dots | \dots, n_\mu, \dots, n_\mu'+1, \dots\rangle \\ &= \sqrt{n_\mu' n_\nu} \;\, \delta^{n_\alpha}_{n_\alpha'} \; \cdots \delta^{n_\mu+1}_{n_\mu'} \cdots \delta^{n_\nu}_{n_\nu' + 1}\cdots \\ \big\langle n_\alpha, \dots \big| \hat{a}_\nu^\dagger \hat{a}_\mu \big| n_\alpha', \dots\big\rangle &= \sqrt{n_\mu' n_\nu}\, \langle \dots, n_\mu, \dots,n_\nu-1,\dots | \dots, n_\mu'-1, \dots, n_\nu'\dots\rangle \\&= \sqrt{n_\mu' n_\nu} \delta^{n_\alpha}_{n_\alpha'} \; \cdots \delta^{n_\mu}_{n_\mu'-1}\cdots \delta^{n_\nu-1}_{n_\nu'} \cdots \\ &= \sqrt{n_\mu' n_\nu} \delta^{n_\alpha}_{n_\alpha'} \; \cdots \delta^{n_\mu+1}_{n_\mu'}\cdots \delta^{n_\nu}_{n_\nu'+1} \cdots\end{aligned}$$
Hence,
$$\hat{a}_\mu \hat{a}_\nu^\dagger - \hat{a}_\nu^\dagger \hat{a}_\mu = 0 \;\;\;\mathrm{for}\;\;\mu\ne\nu.$$
Combining these two results gives the desired commutation relation,
$[\hat{a}_\mu, \hat{a}_\nu^\dagger] = \delta_{\mu\nu}$.  Another useful result which
emerges from the first part of the proof is that
$$\big\langle n_\alpha, n_\beta, \dots \big| \hat{a}_\mu^\dagger \hat{a}_\mu \big| n_\alpha', n_\beta'\dots\big\rangle = n_\mu \big\langle n_\alpha, n_\beta, \dots \big| n_\alpha', n_\beta'\dots\big\rangle.$$
Hence, we can define the Hermitian operator
$$\hat{n}_\mu \equiv \hat{a}_\mu^\dagger \hat{a}_\mu,$$
whose eigenvalue is the occupation number of single-particle state $\mu$.

If you are familiar with the method of creation/annihilation operators
for solving the quantum harmonic oscillator, you will have noticed the
striking similarity with the particle creation/annihilation operators
for bosons.  This is no mere coincidence.  We will examine the
relationship between harmonic oscillators and bosons in the next
chapter.

\subsection{Second quantization for fermions}



\section{Measurements and state collapse}

\section{Quantum field theory}
\label{sec:qft}

\section{The spin-statistics connection}
\label{sec:spinstats}

\section*{Exercises}

\begin{enumerate}
\item Consider a system of two identical particles.  Each
  single-particle Hilbert space $\mathscr{H}^{(1)}$ is spanned by a
  basis $\{|\mu_i\}$.  The exchange operator is defined on
  $\mathscr{H}^{(2)} = \mathscr{H}^{(1)} \otimes \mathscr{H}^{(1)}$ by
$$P \Big (\sum_{ij} \psi_{ij} |\mu_i\rangle|\mu_j\rangle \Big)
  \;\equiv\;  \sum_{ij} \psi_{ij} |\mu_j\rangle|\mu_i\rangle.$$
  Prove that $\hat{P}$ is linear, unitary, and Hermitian.  Moreover,
  prove that the operation is basis-independent: i.e., given any other
  basis $\{\nu_j\}$ that spans $\mathscr{H}^{(1)}$, it is likewise
  true that
$$P \Big (\sum_{ij} \varphi_{ij} |\nu_i\rangle|\nu_j\rangle \Big)
  \;=\;  \sum_{ij} \varphi_{ij} |\nu_j\rangle|\nu_i\rangle.$$
  \label{ex:1}

\item
  Prove that the exchange operator commutes with the Hamiltonian
$$\hat{H} = - \frac{\hbar^2}{2m_e} \Big(\nabla_1^2 + \nabla^2_2\Big) + \frac{e^2}{4\pi\varepsilon_0|\mathbf{r}_1 - \mathbf{r}_2|}.$$ \label{ex:2}

\item
  An $N$-boson state can be written as
$$|\phi_1,\phi_2,\dots,\phi_N\rangle = \mathcal{N} \sum_p \Big(|\phi_{p(1)}\rangle  |\phi_{p(2)}\rangle  |\phi_{p(3)}\rangle  \cdots  |\phi_{p(N)}\rangle\Big).$$
  Prove that the normalization constant is
$$\mathcal{N} = \sqrt{\frac{n_a!n_b!\cdots}{N!}},$$
  where $n_\mu$ denotes the number of particles in state $\mu$.
  \label{ex:boson_norm}

\item
  Prove that for bosonic creation and annihilation operators, $[\hat{a}_\mu,\hat{a}_\nu] = [\hat{a}_\mu^\dagger,\hat{a}_\nu^\dagger] = 0$.
  \label{ex:boson_commutators}
  
\end{enumerate}

\section*{Further Reading}

%% \begin{itemize}
%% \item
%% \end{itemize}

\end{document}


%% For decades after the discovery of quantum mechanics, the quantum
%% double-slit experiment was just a ``thought experiment'', meant to
%% illustrate the features of quantum mechanics that had been uncovered
%% by other, more complicated experiments.  Nowadays, the most convenient
%% way to do the experiment is with light, using single-photon sources
%% and single-photon detectors.  Quantum interference has also been
%% demonstrated experimentally using electrons, neutrons, and even
%% large-scale particles such as buckyballs.
