\documentclass[pra,11pt]{revtex4}
\usepackage{amsmath}
\usepackage{amssymb}
\usepackage{graphicx}
\usepackage{color}
\def\ket#1{\left|#1\right\rangle}
\def\bra#1{\left\langle#1\right|}
\def\braket#1{\left\langle#1\right\rangle}

\setlength{\parindent}{0pt}

\renewcommand{\baselinestretch}{1.0}
\setlength{\parskip}{0.07in}

\begin{document}

\section{Bound states and extended states}

One of the most interesting features of wavefunctions in infinite
space is that they come in two distinct types: (i) \textbf{extended
  states}, which are spread out over the entire space, and (ii)
\textbf{bound states}, which are localized to a particular region.
Both types can co-exist in a single system.

A simple model which contains both bound and extended states is the
\textbf{1D finite square well}.  Consider the Hamiltonian
$$\hat{H} = \frac{\hat{p}^2}{2m} - V_0 \,\Theta(a -|\hat{x}|),$$
where $\hat{x}$ and $\hat{p}$ are the 1D position and momentum
operators, $m$ is the particle mass, $V_0$ and $a$ are positive real
parameters, and $\Theta$ denotes the Heaviside step function (which
gives 1 if the input is positive, and 0 otherwise).  The potential
function describes a square well of depth $V_0$ and width $2a$.
Inside the well region ($|x| < a$), the potential is $-V_0$; outside,
it is zero.

\textcolor{red}{[Fig.]}

We are interested in solutions to the time-independent Schr\"odinger
equation for this Hamiltonian.  The solution procedure will not be
discussed in detail here; if interested, refer to Appendix A.
Instead, let's zoom in on certain key facts.

First of all, note that when solving the Schr\"odinger wave equation,
it's necessary for us to specify the boundary conditions at infinity.
What choice we make affects whether we get a bound state or extended
state.  A bound state has a wavefunction that diminishes exponentially
as $x \rightarrow \pm\infty$: in either external region (i.e., $x <
-a$ or $x > a$), it satisfies
$$-\frac{\hbar^2}{2m}\,\frac{d^2\psi}{dx^2} = E \psi(x),$$
subject to the boundary conditions
$$\psi(x) \overset{x\rightarrow\pm\infty}{\sim} e^{\mp\kappa x}, \;\;\;\mathrm{Re}(\kappa) > 0.$$
Therefore, the solutions in the external regions take the form
$$\psi(x) = c_\pm\, e^{\mp\kappa x}, \;\;\mathrm{where}\;\, -\frac{\hbar^2\kappa^2}{2m} = E, \;\; c_\pm \in \mathbb{C}.$$
Since $E$ is real, it follows that $\kappa$ is real, and hence $E <
0$.  Also, the variational principle implies that $E \ge -V_0$, so we
conclude that bound state energies are restricted to the range $-V_0
\le E < 0$.  It can furthermore be shown that the bound state energies
are \textit{discrete}; similar to a particle-in-a-box problem, the
energy spacing decreases with increasing $a$, but so long as $a$ is
finite the spacing is larger than zero.

A bound state wavefunction takes the form
$$\psi(x) = \begin{cases} c_-\, e^{\kappa x}, & \;\;\;x < -a\\ (\mathrm{something}) , & -a < x < a\\ c_+\, e^{-\kappa x} , & \;\;\,x > a.\end{cases}$$
The formula for $\psi(x)$ inside the well don't concern us right now.
The main point is that since $|\psi(x)|^2$ vanishes exponentially as
$x \rightarrow \pm \infty$, the wavefunction can always be
normalized:
$$\int_{-\infty}^\infty |\psi(x)|^2\, dx\; =\; 1.$$

The situation is quite different for an extended state.  Its
wavefunction does not vanish exponentially at infinity, but instead
takes the form
$$\psi(x) = \begin{cases} \alpha_-\, e^{ik x} + \beta_-\, e^{-ik x}, & \;\;\;x < -a\\ (\mathrm{something}) , & -a < x < a\\ \alpha_+\, e^{ik x} + \beta_+\, e^{-ik x} , & \;\;\,x > a.\end{cases}$$
Within each external region, $\psi(x)$ consists of a superposition of
left-moving and right-moving plane waves.  The coefficients
$\alpha_\pm$ and $\beta_\pm$ are related by a scattering matrix, as we
saw in the previous chapter.  In order to satisfy Schr\"odinger's
equation, the wavenumber $k$ must satisfy
$$\frac{\hbar^2k^2}{2m} = E,$$
which means that extended states occur only for energies $E \ge 0$.
In fact, the solutions form a continuum, meaning that we can find
extended states for \textit{every} $E \ge 0$.  Since $|\psi(x)|^2$
does not diminish exponentially at infinity, $\int_{-\infty}^\infty
|\psi(x)|^2\, dx$ cannot converge to a finite value, except in the
trivial case where the integrand scales to zero.  Hence, the
wavefunction of an extended state has no finite normalization.  (Note:
we formally exclude the case of wavefunctions that blow up at
infinity, rather than going to a constant magnitude.  To understand
why, and for a more rigorous discussion of how extended states are
normalized, see Appendix B.)

The above results are summarized in the following figure:

\textcolor{red}{[Fig.]}

%% For $E \ge 0$, there is a continuum of extended states, which do not
%% have any finite normalization.  For $-V_0 < E < 0$, which is the
%% energy range ``inside'' the potential well, there exists a discrete
%% and finite set of bound states, whose wavefunctions vanish
%% exponentially outside the well and are therefore normalizable.

Many of the lessons drawn from this square well model can be
generalized to more complicated potential wells, so long as (i) the
potential at infinity is some finite value $V_{\textrm{ext}}$ and (ii)
the potential describes a ``well'' (i.e., it has some minimum value
$V_{\mathrm{min}} < V_{\textrm{ext}}$).  For energies $E \ge
V_{\textrm{ext}}$, there will exist a continuum of extended
(non-normalizable) states.  And for $V_{\mathrm{min}} < E <
V_{\textrm{ext}}$, there can exist a discrete set of bound
(normalizable) states.

There are, however, two important provisos.  The first is that the set
of bound states might be empty: it is possible for a potential well to
be ``too shallow'' to support a bound state.  This is observable in
the simple square well model, which turns out to have no bound state
if \textcolor{red}{[???]} (see Appendix A).  An intuitive explanation
is that all quantum wavefunctions possess some ``zero-point'' energy,
and if this exceeds the well depth then no bound state can form.  The
second proviso is that in certain cases, it turns out to be possible
to have bound states with energy $E \ge V_{\textrm{ext}}$.  These are
called ``bound states in the continuum'', since they lie in the same
energy range as the continuum of extended states.  We will ignore this
phenomenon, since it happens only under special circumstances that lie
outside the scope of the present discussion.

\section{Quasi-bound states and scattering resonances}

The bound states and extended states of the finite square well are
extremely different from each other.  But under certain circumstances,
extended states can take on characteristics that make them very
similar to bound states.  These are called \textbf{quasi-bound
  states}, and as we shall see, they play an especially important role
in scattering experiments.

An example of a potential function that supports quasi-bound states is
shown below.  The potential in the external regions is
$V_{\textrm{ext}} = 0$.  There is a ``wall'' of potential $V_1$, in
which we embed a central well of potential $V_0$.

\textcolor{red}{[Fig.]}

Since $V_0, V_1 > V_{\mathrm{ext}}$, we know from the preceding
discussion that the system does not support any bound states, only
extended states.

However, there's something intriguing about the energy range $V_0 < E
< V_1$, corresponding to the central well.  Suppose that we set the
external potential as $V_{\mathrm{ext}} = V_1$ rather than
$V_{\mathrm{ext}} = 0$.  Then the central well could support bound
states, with wavefunctions that are exponentially localized to the
central well region.  Since this scenario differs from the actual
scenario only in the value of $V_{\mathrm{ext}}$, and the bound state
wavefunctions are exponentially suppressed in the external region
anyway, it follows that the bound states in the $V_{\mathrm{ext}} =
V_1$ system are \textit{approximately} energy eigenstates of the
$V_{\mathrm{ext}} = 0$ system.

Turning this argument around, we expect there to exist certain
extended states in the $V_{\mathrm{ext}} = 0$ system whose
wavefunctions look a lot like bound states.  To see if this is true,
we calculate the scattering amplitudes $f_\pm$ (defined in the
previous chapter).  The results are shown below:



Green's function analysis

\section{Decay rate of a quasi-bound state: Fermi's Golden Rule}

\section{The decay spectrum of beta particles}


\appendix
\section{Transfer matrix method for solving the Schr\"odinger equation}


\section{Normalization of bound states and extended states}

In this Appendix, we take a closer look at where bound and extended
states come from, and how they are normalized.  Imagine that instead
of taking $x \in (-\infty,\infty)$, we enclose the system in a very
large but finite box, so that $x \in [-L,L]$ where $L \gg a$.  We
impose periodic boundary conditions on the box walls, $\psi(-L) \equiv
\psi(L)$, and solve the Schr\"odinger equation for wavefunctions
normalizable within the box, such that $\int_{-L}^{L} |\psi(x)|^2 dx =
1$.  For any finite $L$, there is an infinite but discrete set of
solutions, with energies in the range $-V_0 <E < \infty$.

We now examine how the solutions vary upon increasing $L$.

\textcolor{red}{[Fig.]}





\section{Further Reading}

\begin{itemize}
\item Bransden \& Joachain, \S4.4, 9.2--9.3, 13.4
\item Sakurai, \S5.6, 7.7--7.8

\end{itemize}


\end{document}


%% For decades after the discovery of quantum mechanics, the quantum
%% double-slit experiment was just a ``thought experiment'', meant to
%% illustrate the features of quantum mechanics that had been uncovered
%% by other, more complicated experiments.  Nowadays, the most convenient
%% way to do the experiment is with light, using single-photon sources
%% and single-photon detectors.  Quantum interference has also been
%% demonstrated experimentally using electrons, neutrons, and even
%% large-scale particles such as buckyballs.
