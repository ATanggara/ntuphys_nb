\documentclass[11pt,a4paper]{article}
\usepackage{amsmath}
\usepackage{amssymb}
\usepackage{graphicx}
\usepackage{color}
\usepackage{fancyhdr}
\usepackage[margin=3.5cm]{geometry}
\usepackage{framed}
\usepackage{enumerate}
\usepackage{textcomp}
\usepackage{fancyvrb}

\usepackage[colorlinks=true,urlcolor=blue,citecolor=black,linkcolor=black]{hyperref}
\def\ket#1{\left|#1\right\rangle}
\def\bra#1{\left\langle#1\right|}
\def\braket#1{\left\langle#1\right\rangle}

\fancyhf{}
\lhead{NTU/SPMS/PH2198--PH2199 Lab}
\rhead{Error Analysis}
\lfoot{\tiny\textcopyright \;2018 Nanyang Technological University.  Released under \href{http://creativecommons.org/licenses/by-sa/4.0/}{CC BY-SA 4.0}.}
\rfoot{\thepage}
\pagestyle{fancy}

\definecolor{dgreen}{RGB}{70,128,13}

\begin{document}

\begin{center}
\textbf{Division of Physics\;\,\&\;Applied Physics}

\textbf{PH2198/2198---Physics Laboratory IIA/IIB}

\vskip 0.05in

\underline{\Huge Writing a Good Lab Report}
\end{center}

\section{What is---and isn't---a proper lab report}

The lab reports that you are expected to prepare in this class are
shorter and simpler versions of the reports that working scientists
use to communicate their findings.  Such reports are expected to be
accurate, complete, concise, and readable.

\begin{itemize}
\item \textit{Accurate}---The contents must be as unambiguous and
  correct as possible.  ``Correct'', in this context, means what is
  written in the report matches what happened during the experiment;
  it does not mean ``getting the right answer''.  It is OK to have
  results that are uncertain or not matching expectations, provided
  these uncertainties and deviations are pointed out and/or explained.

\item \textit{Complete}---The report must include all information a
  reader would reasonably need in order to ``get the point''.  For
  example, if you use a symbol in an equation, figure, or text, that
  symbol needs to be defined in a caption or the main text.  If you
  show a figure of experimental results, the main text should describe
  how the data were obtained, and what conclusions the reader ought to
  draw from the figure.

\item \textit{Concise}---Do not clutter the report with useless
  information.  Assume that the reader is scientifically literate, at
  the level of a fellow undergraduate physics student.  For example,
  you need not explain the theory of classical mechanics before
  talking about forces.  You should omit details that a reasonable
  reader would consider trivial (e.g., the fact that the voltage on a
  power supply can be increased by turning a dial clockwise).

\item \textit{Readable}---The text in the lab report must form a
  coherent, readable narrative.  The main text should form proper
  paragraphs.  Do not write your report in Q\&A (question-and-answer)
  form, and do not organize your report as a mere collection of
  bullet-point lists.  The description of your experimental procedures
  shouldn't be a list of steps lifted from the lab manual; you must
  re-write that information, incorporating details about the intention
  of the various steps, what happened during the experiment, etc.
\end{itemize}

\noindent
These criteria are easy to remember if you bear in mind the goal of a
lab report, which is to communicate a set of scientific findings to
interested parties.  Readers will be happy if the information in the
report is presented clearly and accurately, and contains all the
details that they need to understand and appreciate the results.  They
will be annoyed if you write about irrelevancies, or give your report
in an inscrutable format like a bullet-point list.

\section{Sections of the report}

Your lab report should be divided into distinct sections.  This course
does not impose a standard set or sequence of sections.  However, a
typical report might be divided as follows:

\begin{enumerate}
\item \textit{Introduction}---Here, you can describe the purpose of
  the experiment, in your own words and according to your own
  understanding.  Do not lift the text from the lab manual (that is
  plagiarism).  This introduction need not be long; in many cases, one
  or two paragraphs is enough.

\item \textit{Procedure}---Here, you can describe the experimental
  set-up and procedure.  When describing the set-up, do not follow the
  lab manual in giving a list of individual pieces of equipment and
  instructions for how to assemble them.  You should instead describe
  the apparatus as a whole (preferably with a diagram), and how it
  works and why.

  When describing the experimental procedure, do not simply lift the
  instructions from the lab manual.  The write-up should take the form
  of a coherent narrative, and include information not present in the
  manual.

\item \textit{Results}---This section should contain \textit{both}
  complete paragraphs of text \textit{and} figures/tables.  It should
  be possible, by reading through the text paragraphs (without looking
  at the figures), to understand what results you obtained.  Every
  figure/table must be referenced and discussed somewhere in the text
  (e.g., ``\textit{The dependence of the sample temperature on applied
    voltage is plotted in Fig.~3.  From the data, we observe an
    approximately linear relationship...}'').

  You should include all necessary details about how you analyzed your
  data.  This includes error analysis: i.e., derivations for the
  uncertainty estimates.  If the analyses are too long, you may
  relegate the details to an appendix.

\item \textit{Conclusions}---This section should briefly summarize
  what scientific conclusions you drew from the experiment.  If you
  obtained results that are unexpected, or seem unreliable, you can
  discuss possible explanations.  It is OK for this section to be as
  short as a single paragraph.

\item \textit{Appendices}---You can use appendices for detailed
  derivations, tables of raw data, etc., to avoid cluttering up the
  main text.

\item \textit{References}---If you referred to any other works in the
  text of the lab report, list them here.  References should be
  numbered, and appear in the same order they are mentioned in the
  text.  The text should cite the references by number.
\end{enumerate}

\noindent
You don't need to stick religiously to the above outline.  For
example, instead of \textit{Procedure} and \textit{Results} section,
in some cases it might make sense to divide the experiment into
several parts, and combine the experimental procedure and results in
each part.







\section{Figures}

\begin{itemize}
\item All text in figures must be legible.  A common mistake is to
  have text that is too small to read.

\item Axes must be clearly labeled, including units.

\item If a graph has multiple sets of curves/data points, they must
  be clearly distinguishable and properly labeled.

\item Data points should have error bars (preferably in both
  directions), unless there's a good reason otherwise.

\item Captions must be present, and must describe the figure clearly.
\end{itemize}


\section{Raw data}

\end{document}
